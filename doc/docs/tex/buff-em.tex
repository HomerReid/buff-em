\documentclass[letterpaper]{article}

\usepackage[english]{babel}
\usepackage{graphicx}
\usepackage{color}
\usepackage{dsfont}
\usepackage{bbm}
\usepackage{amsmath}
\usepackage{amssymb}
\usepackage{float}
\usepackage{psfrag}
\usepackage{mathdots}
\usepackage{algorithm}
\usepackage{algorithmic}
\usepackage{listings}
\usepackage{fancybox}
\usepackage{fancyvrb}
\usepackage{arydshln}
\usepackage{verbatim}

%--------------------------------------------------
%- primed variables 
%--------------------------------------------------
\newcommand\kp{k^\prime}
\newcommand\rp{r^\prime}
\newcommand\kk{k^\prime}

%--------------------------------------------------
%- boldface greek letters 
%--------------------------------------------------
\newcommand\vbphi{\mathbf{\phi}}
\newcommand\vbPhi{\mathbf{\Phi}}
\newcommand{\vbDelta}{\boldsymbol{\Delta}}
\newcommand{\vbLambda}{\boldsymbol{\Lambda}}
\newcommand{\vbGamma}{\boldsymbol{\Gamma}}
\newcommand\vbtheta{\vec{\theta}}
\newcommand\vbbeta{\vec{\beta}}
\newcommand\vbsigma{\vec{\sigma}}
\newcommand{\vbrho}{\boldsymbol{\rho}}
\newcommand\bsigma{\overline{\sigma}}

%--------------------------------------------------
%- field theory stuff 
%--------------------------------------------------
\newcommand\ub{\overline{u}}
\newcommand\ubar{\overline{u}}
\newcommand\vbar{\overline{v}}
\newcommand\psibar{\overline{\psi}}
\newcommand\ps{\FMSlash{p}}
\newcommand\ks{\FMSlash{k}}
\newcommand\qs{\FMSlash{q}}
\newcommand\ls{\FMSlash{l}}
\newcommand\ds{\FMSlash{\partial}}
\newcommand\pps{\FMSlash{p}^\prime}
\newcommand\kps{\FMSlash{k}^\prime}
\newcommand\MM{\mathcal{M}}
\newcommand\btu{\bigtriangleup}

%--------------------------------------------------
%- colors -----------------------------------------
%--------------------------------------------------
\newcommand{\red}[1]{\textcolor{red}{#1}}
\newcommand{\blue}[1]{\textcolor{blue}{#1}}
\newcommand{\green}[1]{\textcolor{green}{#1}}
\newcommand{\atan}{\text{atan}}
\definecolor{cyan}{rgb}{0.0,1.00,1.00}
%\newcommand{\cyan}[1]{\textcolor{cyan}{#1}}
\definecolor{lightblue}{rgb}{0.0,0.00,0.75}
\newcommand{\lightblue}[1]{\textcolor{lightblue}{#1}}

%--------------------------------------------------
% general commands
%--------------------------------------------------
\newcommand{\ccdot}{\, \cdot \,}
\newcommand{\eps}{\epsilon}
\newcommand{\ez}{\epsilon_0}
\newcommand{\unit}[1]{\, \hbox{#1} \, }
\newcommand{\qtext}[1]{\quad \hbox{#1} \quad }
\newcommand{\curl}[0]{\nabla \times \vb}
\newcommand{\divv}[0]{\nabla \cdot \vb}
\newcommand{\uv}[1]{\vb{\hat{#1}}}
\newcommand{\lab}{<\!} 
\newcommand{\rab}{\!>} 
\newcommand{\bra}[1]{\left<#1\right|}
\newcommand{\ket}[1]{\left|#1\right>}
\newcommand{\ketb}[1]{\big|#1\big>}
\newcommand{\ketB}[1]{\Big|#1\Big>}
\newcommand{\inp}[2]{\left<#1\right|\left.#2\right>}
\newcommand{\Inp}[2]{\big<#1\big|\big.#2\big>}
\newcommand{\INP}[2]{\Big<#1\Big|\Big.#2\Big>}
\newcommand{\Inpb}[2]{\big<#1\big|\big.#2\big>}
\newcommand{\inpB}[2]{\Big<#1\Big|\Big.#2\Big>}
\newcommand{\inpv}[2]{\left<\vp #1\right|\left.#2\right>}
\newcommand{\expval}[1]{\left< #1 \right>}
\newcommand{\expvalv}[1]{\left<\vp #1 \right>}
\newcommand{\exptwo}[3]{\left<#1\right|#2\left|#3\right>} 
\newcommand{\ExpTwo}[3]{\big<#1\big|#2\big|#3\big>} 
\newcommand{\EXPTWO}[3]{\Big<#1\Big|#2\Big|#3\Big>} 
\newcommand{\exptwob}[3]{\big<#1\big|#2\big|#3\big>} 
\newcommand{\exptwoB}[3]{\Big<#1\Big|#2\Big|#3\Big>} 
\newcommand{\exptwov}[3]{\left<\vp#1\right|#2\left|\vp#3\right>} 
\newcommand{\exptwoi}[3]{\left<\vi#1\right|#2\left|\vi#3\right>} 
\newcommand{\union}{\cup}
\newcommand\unitmatrix{\mathds{1}}
\newcommand\Tr{\hbox{Tr }}
\newcommand\tbtm[4]{\left(\begin{array}{ll}#1 & #2 \\ #3 & #4\end{array}\right)}
\newcommand\sups[1]{^{\hbox{\scriptsize{#1}}}}
\newcommand\supt[1]{^{\hbox{\tiny{#1}}}}
\newcommand\subs[1]{_{\hbox{\scriptsize{#1}}}}
\newcommand\subt[1]{_{\hbox{\tiny{#1}}}}
\newcommand{\nn}{\nonumber \\}
\newcommand{\vb}[1]{\mathbf{#1}}
\newcommand{\eq}[1]{\begin{equation} #1 \end{equation}}
\newcommand{\numeq}[2]{\begin{equation} #2 \label{#1} \end{equation}}
\newcommand{\BE}{\begin{equation}}
\newcommand{\EE}{\end{equation}}
\newcommand{\pard}[2]{\frac{\partial #1}{\partial #2}}
\newcommand{\pardn}[3]{\frac{\partial^{#1} #2}{\partial #3^{#1}}}
\newcommand{\pf}[2]{\left(\frac{#1}{#2}\right)}
\newcommand{\vp}{\vphantom\sum}
\newcommand{\vi}{\vphantom{\sum_{-\infty^\infty}}}
\newcommand{\evalat}[2]{\left. #1 \right|_{#2}}
\newcommand{\evalatl}[2]{\left| #1 \right|_{#2}}
\newcommand{\pp}{{\prime\prime}}
\newcommand{\vbhat}[1]{\vb{\hat #1}}
\newcommand{\vbhatt}[1]{\boldsymbol{\widehat #1}}
\newcommand{\wh}{\widehat}
\newcommand{\mc}[1]{\mathcal{#1}}
\newcommand{\bmc}[1]{\boldsymbol{\mathcal{#1}}}
\newcommand{\mb}[1]{\mathbb{#1}}
\newcommand{\primedsum}{\sideset{}{'}{\sum}}

\graphicspath{{figures/}}
\definecolor{shadecolor}{rgb}{0.85,0.85,0.85}

%--------------------------------------------------
%-- \text with a built-in font size specifier   
%--------------------------------------------------
\newcommand\ts[1]{\text{\scriptsize{#1}}}
\newcommand\ty[1]{\text{\tiny{#1}}}

\newcommand{\prob}[3]{\section*{Problem #1}} 

%--------------------------------------------------
%- shaded verbatim for code inclusions
%--------------------------------------------------
\definecolor{lightgrey}{rgb}{0.75,0.75,0.75}
\newenvironment{verbcode}{\VerbatimEnvironment% 
  \noindent
  %      {\columnwidth-\leftmargin-\rightmargin-2\fboxsep-2\fboxrule-4pt} 
  \begin{Sbox} 
  \begin{minipage}{\linewidth-2\fboxsep-2\fboxrule-4pt}    
  \begin{Verbatim}
}{% 
  \end{Verbatim}  
  \end{minipage}   
  \end{Sbox} 
  \fcolorbox{black}{lightgrey}{\textcolor{black}{\TheSbox}}
} 


\usepackage{cite}

\graphicspath{{figures/}}
\newcommand{\TInv}{\rotatebox[origin=c]{180}{$T$}}
\newcommand{\vbTInv}{\rotatebox[origin=c]{180}{$\vb T$}}
\newcommand{\vbTInvSmall}{\rotatebox[origin=c]{180}{\scriptsize{$\vb T$}}}
\newcommand{\bbTInv}{\rotatebox[origin=c]{180}{$\mathbb{T}$}}
\newcommand{\bbVInv}{\rotatebox[origin=c]{180}{$\mathbb{V}$}}
\newcommand{\vbSigma}{\boldsymbol{\Sigma}}
\newcommand{\fd}{^{\text{\tiny{F}}\dagger}}

% 'IEM' stands for 'imaginary part of epsilon matrix'
\newcommand{\vbIEM}{\boldsymbol{\Sigma}}
\newcommand{\IEM}{\Sigma}

\newcommand{\citeasnoun}[1]{Ref.~\citen{#1}}
\newcommand{\citeasnouns}[1]{Refs.~\citen{#1}}

%------------------------------------------------------------
%------------------------------------------------------------
%- Special commands for this document -----------------------
%------------------------------------------------------------
%------------------------------------------------------------
\newcommand{\vbxi}{\boldsymbol{\xi}}
\newcommand{\im}{\text{Im }}
\newcommand{\vbeps}{\boldsymbol{\epsilon}}

%------------------------------------------------------------
%------------------------------------------------------------
%- Document header  -----------------------------------------
%------------------------------------------------------------
%------------------------------------------------------------
\title { {\sc buff-em}: A Volume-Integral Solver Suite \\
         for Classical Scattering and \\
         Fluctuational Electrodynamics}
\author {Homer Reid}
\date {June 3, 2014}

%------------------------------------------------------------
%------------------------------------------------------------
%- Start of actual document
%------------------------------------------------------------
%------------------------------------------------------------

\begin{document}
\pagestyle{myheadings}
\markright{Homer Reid: \texttt{buff-em} }
\maketitle

\begin{abstract}
The $\mb T$-matrix approach to fluctuational electrodynamics,
pioneered by the MIT Casimir theory group led by Professors Kardar
and Jaffe, has yielded a bountiful smorgasbord of analytical
formulas expressing quantities in fluctuational 
electrodynamics---including equilibrium and non-equilibrium
Casimir forces and thermal heat-transfer rates for compact 
and extended bodies---in terms of the $\mb T$-matrices of the 
bodies in question\cite{Rahi2009, Krueger2012}.
Although the analytical insight afforded by these formulas
is immensely valuable,
their practical application has typically been restricted
to the small catalog of highly symmetric bodies---such as 
homogeneous spheres---for which $\mb T$-matrix element may
be computed in closed form.

In its earliest incarnation, {\sc buff-em} was born as
an attempt to extend the $\mb T$-matrix approach to a more
general class of bodies by computing $\mb T$-matrices
numerically. However, in the course of implementing
these calculations I discovered that the $\mb T$-matrix
formalism may in fact be understood as simply a disguised
version of the well-known volume-integral-equation (VIE) 
approach to computational electromagnetism, and that
in implementing a numerical tool for computing
$\mb T$-matrices one is in fact implementing a VIE solver.

The {\sc buff-em} suite\footnote{{\sc buff-em} stands
for \textbf{bu}lk \textbf{f}ield \textbf{f}ormulation of
\textbf{e}lectro\textbf{m}agnetism.} consists of a
core library ({\sc libbuff}) implementing this solver---using
SWG basis functions\cite{SWG1984}---together with 
application modules for classical scattering 
({\sc buff-scatter}) and non-equilibrium fluctuational 
electrodynamics ({\sc buff-neq}).
\end{abstract}

%%%%%%%%%%%%%%%%%%%%%%%%%%%%%%%%%%%%%%%%%%%%%%%%%%%%%%%%%%%%%%%%%%%%%%
%%%%%%%%%%%%%%%%%%%%%%%%%%%%%%%%%%%%%%%%%%%%%%%%%%%%%%%%%%%%%%%%%%%%%%
%%%%%%%%%%%%%%%%%%%%%%%%%%%%%%%%%%%%%%%%%%%%%%%%%%%%%%%%%%%%%%%%%%%%%%
\newpage
\tableofcontents

%%%%%%%%%%%%%%%%%%%%%%%%%%%%%%%%%%%%%%%%%%%%%%%%%%%%%%%%%%%%%%%%%%%%%%
%%%%%%%%%%%%%%%%%%%%%%%%%%%%%%%%%%%%%%%%%%%%%%%%%%%%%%%%%%%%%%%%%%%%%%
%%%%%%%%%%%%%%%%%%%%%%%%%%%%%%%%%%%%%%%%%%%%%%%%%%%%%%%%%%%%%%%%%%%%%%
\newpage
\addcontentsline{toc}{section}{\large Part I: Theory}
\section{VIE formulation of classical EM scattering}

In this section I rederive the volume-integral-equation (VIE)
approach to classical electromagnetic scattering. This formulation
is standard, but I here describe it using terminology and
symbols that emphasize the connection to the $\mathbb{T}$-matrix
scattering approach used by the Kardar-Jaffe group.

%=================================================
%=================================================
%=================================================
\subsection{Continuous VIE formulation}

Consider a material body with relative permittivity tensor
$\vbeps(\vb x)$ lying in vacuum and irradiated by monochromatic
sources which may lie inside or outside the body. 
Let the electric field due to the sources be $\vb E\sups{inc}$ 
and let $\vb J\supt{I}(\vb x)$ be the induced volume current
density throughout the bulk of the body. 
(We work at frequency $\omega$ and assume time dependence
of all field and currents $\propto e^{-i\omega t}$.)
The total electric field at any point is a sum of
``incident'' and ``scattered'' contributions:\footnote{We
put the terms ``incident'' and ``scattered'' in quotes
to remind readers that the ``incident''-field sources may
in fact lie \textit{inside} the body; in this case it is 
not quite right to refer to the field they produce as 
being ``incident'' on the body, but the terminology is 
convenient nonetheless.}
%====================================================================%
\begin{align}
 \vb E\sups{tot} &= \vb E\sups{inc} + \vb E\sups{scat}
\\
                 &= \vb E\sups{inc} + ikZ_0 \mathbb{G}*\vb J\supt{I}
\label{ETot}
\end{align}
%====================================================================%
where $k=\omega/c$ is the (free-space) wavenumber,
$Z_0=\sqrt{\mu_0/\epsilon_0}$ is the impedance of free 
space\footnote{Here and throughout we consistently eliminate
all reference to the free-space permittivity and
permeability constants $\epsilon_0,\mu_0$ in favor of
$Z_0=\sqrt{\mu_0/\epsilon_0}$ and $c=1/\sqrt{\epsilon_0 \mu_0}.$
For example, we write the combinations 
$\{\epsilon_0\omega, \mu_0\omega\}$ respectively in the form 
$\{\frac{k}{Z_0}, kZ_0\}$ where
$k=\omega/c$ is the free space wavelength.},
$*$ denotes convolution, and $\mathbb{G}$ is the free-space
dyadic Green's function:
%====================================================================%
\begin{align}
  \mathbb{G}_{ij}(\vb x, \vb x^\prime) 
&=
   \Big(\vb \delta_{ij} - \frac{1}{k^2} \partial_i \partial_j \Big)
   \frac{e^{ikr}}{4\pi r}
   \qquad (r\equiv |\vb r|=|\vb x-\vb x^\prime|)
\\
%--------------------------------------------------------------------%
&=
\frac{e^{ikr}}{4\pi k^2 r^3}
  \Big[ f_1(ikr) \delta_{ij} + f_2(ikr) \frac{r_i r_j}{r^2} \Big]
\label{HelmholtzDyadic} 
\\
%--------------------------------------------------------------------%
  f_1(x) &\equiv -1 + x -x^2, \qquad
  f_2(x) \equiv  3 - 3x + x^2.
\nonumber
\end{align}
%====================================================================%
On the other hand, the induced current is related to the total field
according to
%====================================================================%
\begin{align}
 \vb J\supt{I}(\vb x) 
&= -i\frac{k}{Z_0} \big[\vbeps(\vb x) - \vb 1\big]\cdot \vb E\sups{tot}(\vb x)
\nn
&\equiv -\frac{1}{ikZ_0} \mathbb{V}(\vb x) \cdot \vb E\sups{tot}(\vb x)
\label{JFromETot}
\end{align}
%====================================================================%
where $\mathbb{V}\equiv k^2\big[\vb 1 - \vbeps(\vb x)\big]$ is
sometimes~\cite{Rahi2009} known as the ``potential.''
At points not in free space, i.e. points at which 
$\vbeps(\vb x)\ne \vb 1$, we can invert this equation to read
%====================================================================%
\begin{align}
-ikZ_0 \bbVInv(\vb x) \cdot \vb J\supt{I}(\vb x) &= \vb E\sups{tot}(\vb x)
\nonumber
\intertext{where $\bbVInv\equiv \mathbb{V}^{-1}.$ Now insert (\ref{ETot}):}
-ikZ_0 \bbVInv(\vb x) \cdot \vb J\supt{I}(\vb x) 
&= \vb E\sups{inc} + ik Z_0 \mathbb{G} \star \vb J\supt{I} \\
\intertext{Rearranging and writing out the convolution, 
           we obtain a volume-integral equation for $\vb J\supt{I}$:} 
  -ikZ_0\left[   \bbVInv(\vb x) \cdot \vb J\supt{I}(\vb x)
              + \int \mathbb{G}(\vb x, \vb x^\prime) 
                \cdot \vb J\supt{I}(\vb x^\prime) \,d\vb x^\prime
        \right]
&= \vb E\sups{inc}(\vb x).
\\
\intertext{In what follows it will be convenient to think of the LHS
           here as the convolution of just a single operator with $\vb J\supt{I}$:}
 -ikZ_0
  \int \bbTInv(\vb x, \vb x^\prime) \cdot \vb J\supt{I}(\vb x^\prime) \,d\vb x^\prime
&= \vb E\sups{inc}(\vb x).
\label{ContinuousVIE}
\end{align}
where
\numeq{TOperator}
{
  \bbTInv(\vb x, \vb x^\prime) 
  =
  \bbVInv(\vb x) \delta(\vb x-\vb x^\prime)+ \mathbb{G}(\vb x,\vb x^\prime).
}
(The symbol $\bbTInv$ is pronounced ``tee-inverse'' or ``eet''.)

%=================================================
%=================================================
%=================================================
\subsection{Discretized VIE formulation}

Now let $\vb b_\alpha$ be some convenient set of $N$ vector-valued 
basis functions and approximate the induced current in the form
%====================================================================%
\numeq{JExpansion}
 { \vb J\supt{I}(\vb x) \approx \sum_{\alpha} j\supt{I}_\alpha \vb b_\alpha(\vb x). }
%====================================================================%
Insert into (\ref{ContinuousVIE}) and ``test'' both sides
with the elements of the set $\{\vb b_\alpha\}$ to obtain a
discretized version of (\ref{ContinuousVIE}) in the form of an
$N\times N$ linear system:
%====================================================================%
\numeq{VIESystem}
{
 \vbTInv \cdot \vb j\supt{I} = \vb v
}
%====================================================================%
where the elements of the vector $\vb j\supt{I}$ are the expansion 
coefficients in (\ref{JExpansion}) and the elements of 
$\vbTInv$ and $\vb v$ are 
%====================================================================%
\numeq{Elements}
{
 \TInv_{\alpha\beta}=\exptwoB{\vb b_\alpha}{\bbTInv}{\vb b_\beta},
 \qquad
 v_{\alpha}=-\frac{1}{ikZ_0} \inpB{\vb E\sups{inc}}{\vb b_\beta}.
}
%====================================================================%

\subsubsection*{VIE matrix for geometries containing multiple bodies}

Equation (\ref{VIESystem}) applies to the case in which we 
have only a single material body. For a geometry involving
$N$ separate bodies, this is generalized to read
%====================================================================%
\numeq{VIESystem2}
{
 \underbrace{
 \left(\begin{array}{cccc}
 \vbTInv_1  & \vb G_{12}  & \cdots & \vb G_{1N} \\
 \vb G_{21} & \vbTInv_{2} & \cdots & \vb G_{2N} \\ 
 \vdots     & \vdots      & \ddots & \vdots     \\ 
 \vb G_{N1} & \vb G_{N2}  & \cdots & \vbTInv_N 
 \end{array}\right)
            }_{\vb M}
%--------------------------------------------------------------------%
 \underbrace{
\left(\begin{array}{c} 
 \vphantom{\vbTInv_1} \vb j_1\supt{I} \\
 \vphantom{\vbTInv_1} \vb j_2\supt{I} \\
 \vphantom{\ddots}    \vdots  \\
 \vphantom{\vbTInv_1} \vb j_N\supt{I} \\
\end{array}\right)
            }_{\vb j}
%--------------------------------------------------------------------%
 =
 \underbrace{
\left(\begin{array}{c} 
 \vphantom{\vbTInv_1} \vb v_1 \\
 \vphantom{\vbTInv_1} \vb v_2 \\
 \vphantom{\ddots}    \vdots  \\
 \vphantom{\vbTInv_1} \vb v_N \\
\end{array}\right).
            }_{\vb v}
}
%====================================================================%
where $\vb M$ is the VIE matrix for the composite system.
Here the diagonal blocks involve just the matrix elements
of the inverse $\mathbb{T}$-operator [defined by equation (\ref{TOperator})
with the $\bbVInv$ operator appropriate for the $n$th body]
while the off-diagonal blocks involve only the matrix elements of 
the $\mathbb{G}$ operator. 

The matrices $\vb M$ and $\vb v$ in equation (\ref{VIESystem2})
are the quantities computed by the \texttt{AssembleVIEMatrix()}
and \texttt{AssembleRHSVector()} methods of the 
\texttt{SWGGeometry} class in {\sc buff-em}.

%=================================================
%=================================================
%=================================================
\subsection*{Computation of scattered fields}

For a geometry irradiated by incident fields 
$\{\vb E, \vb H\}\sups{inc}$, the total fields at 
$\vb x$ are
%====================================================================%
\begin{align}
\vb E\sups{tot}(\vb x)
&= \vb E\sups{inc}(\vb x) + \vb E\sups{scat}(\vb x) 
\\
\vb H\sups{tot}(\vb x)
&= \vb H\sups{inc}(\vb x) + \vb H\sups{scat}(\vb x) 
\label{EHTotIncScat}
\\
%--------------------------------------------------------------------%
\vb E\sups{scat}(\vb x)
&=\sum_\alpha j_\alpha \vb E_\alpha(\vb x) 
\\
\vb H\sups{scat}(\vb x)
&=\sum_\alpha j_\alpha \vb H_\alpha(\vb x)
%--------------------------------------------------------------------%
\intertext{where $\{\vb E, \vb H\}_\alpha(\vb x)$ are the fields due 
to basis function $\vb b_\alpha$ populated with unit strength:}
%--------------------------------------------------------------------%
\vb E_{\alpha}(\vb x)
&\equiv
   ikZ_0 \int_{\sup \vb b_\alpha} \mb G(\vb x, \vb x^\prime)
   \vb b_\alpha(\vb x^\prime) d\vb x^\prime
\\
\vb H_{\alpha}(\vb x)
&\equiv
   -ik \int_{\sup \vb b_\alpha} \mb C(\vb x, \vb x^\prime)
   \vb b_\alpha(\vb x^\prime) d\vb x^\prime.
\end{align}
%====================================================================%
Scattered and total fields are computed by the
\texttt{GetFields()} method of the \texttt{SWGGeometry} class
in the {\sc buff-em} core library.

%%%%%%%%%%%%%%%%%%%%%%%%%%%%%%%%%%%%%%%%%%%%%%%%%%%%%%%%%%%%%%%%%%%%%%
%%%%%%%%%%%%%%%%%%%%%%%%%%%%%%%%%%%%%%%%%%%%%%%%%%%%%%%%%%%%%%%%%%%%%%
%%%%%%%%%%%%%%%%%%%%%%%%%%%%%%%%%%%%%%%%%%%%%%%%%%%%%%%%%%%%%%%%%%%%%%
\newpage
\section{Computation of power, force, and torque}

As is true for surface-integral solvers like {\sc scuff-em},
in volume-integral solvers like {\sc buff-em} there are
multiple distinct ways of computing the power, force, and
torque (PFT) on a body.

%=================================================
%=================================================
%=================================================
\subsection{Displaced surface-integral (DSI) PFT}

A first approach is to evaluate surface integrals of the
Poynting vector and Maxwell stress tensor over a closed
bounding surface $\mc S$ containing the body but displaced
from its surface:
%%%%%%%%%%%%%%%%%%%%%%%%%%%%%%%%%%%%%%%%%%%%%%%%%%%%%%%%%%%%%%%%%%%%%%
\begin{align*}
 P\sups{abs}
&=
\frac{1}{2}\oint_{\mc S}
   \bmc F^\dagger(\vb x)
   \bmc N\supt{P}\Big(\vbhat{n}(\vb x)\Big)
   \bmc F        (\vb x) \, dA
\\
 \vb F \cdot \vbhat{u} 
&=
\frac{1}{2}\oint_{\mc S}
   \bmc F^\dagger(\vb x)
   \bmc N\supt{F}\Big(\vbhat{n}(\vb x), \vbhat{u} \Big)
   \bmc F        (\vb x) \, dA
\\
 \bmc T \cdot \vbhat{u} 
&=
\frac{1}{2}\oint_{\mc S}
   \bmc F^\dagger(\vb x)
   \bmc N\supt{T}\Big(\vbhat{n}(\vb x), \vbhat{u} \Big)
   \bmc F        (\vb x) \, dA
\end{align*}
%%%%%%%%%%%%%%%%%%%%%%%%%%%%%%%%%%%%%%%%%%%%%%%%%%%%%%%%%%%%%%%%%%%%%%
Here $\bmc F={\vb E \choose \vb H}$ is the 6-vector of fields
computed from (\ref{EHTotIncScat}), and the $\bmc N$ matrices
are certain constant $6\times 6$ matrices whose entries depend on
the surface normal $\vbhat{n}$ to $\mc S$ at $\vb x$.

%=================================================
%=================================================
%=================================================
\subsection{JDEPFT}

An alternative to the surface-integral method of the 
previous section is to compute the power, force and torque 
on a body using a volume-integral approach:
%====================================================================%
\begin{subequations}
\begin{align}
 P\sups{abs}&=\frac{1}{2}\text{Re }\int \vb J^* \cdot \vb E \, dV 
\\
 F_i&=\frac{1}{2\omega}\text{Im }\int \vb J^* \cdot \partial_i \vb E \, dV
\\
 \mc T_i
 &=\frac{1}{2\omega}\text{Im }\int 
 \Big[ \vb J^* \times \vb E + \vb J^* \cdot \partial_{\theta_i} \vb E\Big]dV
\end{align}
\label{JDEPFT}%
\end{subequations}%
%====================================================================%
where $\vb J$ is the induced current and $\vb E$ is the total field.
Equation (\ref{JDEPFT}a) is just the usual Joule heating, while
Equations (\ref{JDEPFT}b,c) follow from Lorenz-force considerations
and are derived in Appendix \ref{ForceFormulaAppendix}.
I call equations (\ref{JDEPFT}) the ``J dot E'' or JDE approach
to power, force, and torque computation. [The second term in 
(\ref{JDEPFT}c) is typically small and will be neglected below

For implementation purposes, it is convenient to separate the total
field $\vb E$ in equations (\ref{JDEPFT}) into incident and scattered
portions and to write
%====================================================================%
\begin{align}
   \{P\sups{abs}, F_i, \mc T_i\}
=  \{P\sups{abs}, F_i, \mc T_i\}\sups{JI}
+  \{P\sups{abs}, F_i, \mc T_i\}\sups{JJ}
\end{align}
%====================================================================%
where the ``JI'' terms involve the interaction of $\vb J$ with
the incident field $\vb E\sups{inc}$ alone, while the ``JJ'' terms
involve the interaction of $\vb J$ with itself.

The JI contribution to the power is 
%====================================================================%
\begin{subequations}
\begin{align}
P\sups{JI}
&= \frac{1}{2}\text{Re }\int \vb J^* \cdot \vb E\sups{inc} \, dV 
\\
&= \frac{1}{2}\text{Re }\sum_{\alpha} j^*_\alpha
   \INP{\vb b_\alpha}{\vb E\sups{inc}}
\\
\intertext{and similarly}
F_i\sups{JI}
&= \frac{1}{2\omega}\text{Im }\sum_{\alpha} j^*_\alpha
   \INP{\vb b_\alpha}{\partial_i \vb E\sups{inc}}
\\
\mc T_i\sups{JI}
&= \frac{1}{2\omega}\text{Im }\sum_{\alpha} j^*_\alpha
   \Big\langle \vb b_\alpha \times \vb E\sups{inc}\Big\rangle.
\end{align}
\label{JDEPFTTerm1}
\end{subequations}
%====================================================================%
The three-dimensional integrals involved in the matrix elements
in (\ref{JDEPFTTerm1}) are nonsingular and evaluated in {\sc buff-em}
by low-order numerical cubature (Appendix \ref{VolumeIntegralAppendix}).

On the other hand, the JJ contributions to the power, force, and torque
involve \textit{six}-dimensional integrals:
%====================================================================%
\begin{subequations}
\begin{align}
   P\sups{JJ}
&= \frac{1}{2}\text{Re }\int \vb J^*(\vb x) \cdot \vb E\sups{scat}(\vb x)\, d\vb x
\intertext{Using $\vb E\sups{scat}=ik\mb G \star \vb J$, this becomes}
&= \frac{kZ_0}{2}
   \text{Re }
   \iint J_i^*(\vb x)
         \Big( i\mb G_{ij}(\vb x, \vb x^\prime) \Big)
         J_j(\vb x^\prime)\, d\vb x \, d\vb x^\prime
\intertext{and similarly}
F_i     &=\frac{Z_0}{2c}\text{Im }
 \iint J^*_j \Big(i\partial_i \mathbb{G}_{jk}\Big) J_k
 \, d\vb x \, d\vb x^\prime
\\
\mc T_i &=\frac{Z_0}{2c}\text{Im }\varepsilon_{ijk}
 \iint J^*_j \Big(i\mathbb{G}_{k\ell}\Big) J_\ell 
 \, d\vb x \, d\vb x^\prime
\end{align}%
\label{JDEPFTTerm2a}%
\end{subequations}%
%====================================================================%
Although equations (\ref{JDEPFTTerm2a}) appear to be singular
integrals, this appearance is misleading, as is demonstrated
by the following rewriting, which follows from Onsager 
reciprocity [$\mb G_{ij}(\vb x, \vb y)=\mb G_{ji}(\vb y, \vb x)$]:
%====================================================================%
\begin{subequations}
\begin{align}
  P\sups{JJ}&=-\frac{kZ_0}{2}
 \iint \Big[ \text{Re }\big( J_i^* J_j \big)
             \text{Im }\mathbb G_{ij}
       \Big] \, dV dV^\prime
\\
 F_i\sups{JJ}&=-\frac{Z_0}{2c}
 \iint \Big[ \text{Im }\big( J_j^* J_k \big)
             \text{Im }\partial_i \mathbb G_{jk}
       \Big] \, dV dV^\prime
\\
 \mc T_i\sups{JJ}&=\frac{Z_0}{2c} \varepsilon_{ijk}
 \iint \Big[ \text{Im }\big( J_j^* J_\ell \big)
             \text{Im }\mathbb G_{k\ell}
       \Big] \, dV dV^\prime
\end{align}%
\label{JDEPFTTerm2b}
\end{subequations}
%====================================================================%
These equations involve only the imaginary part of $\mb G$,
which is \textit{non-singular}; indeed, in the short-distance
limit one finds immediately from (\ref{HelmholtzDyadic}) that
%====================================================================%
\numeq{ImGExpansion}
{
\text{Im }\mathbb{G}_{ij}(\vb r)
 = \frac{k}{6\pi}\delta_{ij}
  -\frac{k^3 r^2}{30\pi}
   \Big[\delta_{ij} - \frac{1}{2}\frac{r_i r_j}{r^2}\Big]
  +\frac{k^5 r^4}{560\pi}
  \Big[\delta_{ij} - \frac{2}{3}\frac{r_i r_j}{r^2}\Big]
  +O(r^6)
}
%====================================================================%
The discretized versions of equations (\ref{JDEPFTTerm2b}) read, upon
accounting for simplifying symmetries,
%====================================================================%
\begin{subequations}
\begin{align}
  P\sups{JJ}
&=-k Z_0 \primedsum_{\beta\ge \alpha}
  \Big( \text{Re } j_\alpha^* j_\beta \Big)
  \EXPTWO{\vb b_\alpha}{\text{Im }\mb G}{\vb b_\beta}
\\
  F_i\sups{JJ}
&=-\frac{Z_0}{ c} \sum_{\beta>\alpha}
   \Big( \text{Im } j_\alpha^* j_\beta \Big)
   \EXPTWO{\vb b_\alpha}{\text{Im }\partial_i \mb G}{\vb b_\beta}
\\
  \mc T_i\sups{JJ}
&=-\frac{Z_0}{c} \sum_{\beta>\alpha}
   \Big( \text{Im } j_\alpha^* j_\beta \Big)
   \varepsilon_{ijk}
   \Big\langle b_{\alpha j} \Big(\text{Im } \mb G_{k\ell}\Big) b_{\beta \ell}
   \Big\rangle 
\end{align}
\label{JDEPFTTerm2c}%
\end{subequations}
%====================================================================%
where the primed sum in (\ref{JDEPFTTerm2c}) indicates that 
summands with $\alpha=\beta$ are to be weighted by $\frac{1}{2}$.
The matrix elements in (\ref{JDEPFTTerm2c}) involve 
nonsingular 6-dimensional integrals which are evaluated
in {\sc buff-em} by simple numerical cubature.

\subsection*{Multipole expansion of PFT quantities}

Inserting the short-distance expansion (\ref{ImGExpansion})
into (\ref{JDEPFTTerm2b}a) and keeping only terms of lowest
order in $k$ yields
%====================================================================%
\begin{align}
  P\sups{JJ}
&\approx
-\frac{k^2 Z_0}{12 \pi}
  \underbrace{ \int J_i^*(\vb x) d\vb x }_{i\omega p^*_i}
  \underbrace{ \int J_i  (\vb x^\prime) d\vb x^\prime }_{-i\omega p_i}
\\
&=-\frac{c^2 k^4 Z_0}{12 \pi}|\vb p|^2
\label{PMultipole}
\end{align}
%====================================================================%
where $\vb p=-\frac{1}{i\omega} \int \vb J dV$ is the
dipole moment of the induced current distribution.
Note that (\ref{PMultipole}) is minus the usual expression
for the total power radiated by a point dipole radiator.

Proceeding similarly for the force, from (\ref{ImGExpansion})
one first finds
%====================================================================%
$$ \partial_i G_{jk} = \frac{k^3}{60\pi}
   \Big( r_j \delta_{ik} + r_k \delta_{ij} - 4r_i \delta_{jk} \Big)
   + O(k^5)
$$
%====================================================================%
whereupon (\ref{JDEPFTTerm2b}b) reads
%====================================================================%
\begin{align}
  F_i\sups{JJ}
&\approx
-\frac{k^3 Z_0}{120 \pi c}
  \text{Im }
  \int \bigg\{  J^*_j(\vb x)(\vb x-\vb x^\prime)_j J_i(\vb x^\prime)
               +J^*_i(\vb x)(\vb x-\vb x^\prime)_j J_j(\vb x^\prime)
\nn
&\hspace{2in}
              -4J^*_j(\vb x)(\vb x-\vb x^\prime)_i J_j(\vb x^\prime)
       \bigg\}\,d\vb x \, d\vb x^\prime
\nn
&= -\frac{k^3 Z_0}{60 \pi c}\text{Im }
   \Big[  \mc M^*_{jj} \mc M_i  + \mc M^*_{ij} \mc M_j
        -4\mc M^*_{ji} \mc M_j \Big]
\label{ForceMultipole1}
\end{align}
where I defined 
%====================================================================%
$$ \mc M_{i}  \equiv \int J_i(\vb x)     \, dV, \qquad
   \mc M_{ij} \equiv \int J_i(\vb x) x_j \, dV.
$$
%====================================================================%
The quantity $\mc M_i$ is related to the electric dipole moment $\vb p$ by 
$$\vb p_i = -\frac{1}{i\omega} \mc M_i.$$
On the other hand, $\mc M_{ij}$ is related to the magnetic
dipole and electric quadrupole moments; basically, 
the magnetic dipole moment $\vb m$ 
is the antisymmetric part of $\mc M_{ij}$,
while the electric quadrupole moment $Q_{ij}$ 
is the symmetric part.
From standard definitions in e.g. Jackson one finds
%====================================================================%
\begin{align*}
 m_i 
&= \frac{1}{2}\varepsilon_{ijk}\int x_i J_j(\vb x) dV 
\\
&= \frac{1}{2}\varepsilon_{ijk}\mc M_{ji}
\\
&= -\frac{1}{2}\varepsilon_{ijk}\mc M_{ij}
\\
 Q_{ij} 
&=
 -\frac{1}{i\omega} 
 \int \Big\{ 3J_i x_j + 3x_i J_j - 2J_k x_k \delta_{ij} \Big\} \, dV
\\
&=
 -\frac{1}{i\omega}
  \Big[ 3\mc M_{ij} + 3\mc M_{ji} - 2\delta_{ij} \mc M_{kk} \Big].
\end{align*}
%====================================================================%
Using these definitions in (\ref{ForceMultipole1}) and performing
some algebra, one finds the lowest-order terms in the multipole 
expansion of the self-force:
%====================================================================%
\begin{align}
  F_i\sups{JJ}
\label{ForceMultipole1}
&\approx
\frac{k^4 Z_0}{12\pi}\text{Re}\Big( \vb m^* \times \vb p \Big)_i
+
\frac{c k^5 Z_0}{120\pi}\text{Im}\Big( \vb Q^* \vb p \Big)_i
\end{align}
%====================================================================%
The quantity in the second term involves the matrix-vector product 
of the 3$\times $3 matrix $\vb Q^*$ with the 3-vector $\vb p$.

Finally, for the torque, inserting (\ref{ImGExpansion}) into
(\ref{JDEPFTTerm2b}b) yields
%====================================================================%
\begin{align}
  \mc T_i\sups{JJ}
&\approx
-\frac{k Z_0}{12 \pi c}\varepsilon_{ijk}
  \text{Im }\left\{ 
\left[\int \vb J^*_j dV\right] \left[\int \vb J_k dV\right]
            \right\}
\\
&= -\frac{c k^3 Z_0}{6 \pi}
    \Big(\text{Re }\vb p \times \text{Im }\vb p\Big)_i.
\end{align}
%====================================================================%

%=================================================
%=================================================
%=================================================
\subsection{OPFT}

From equation (\ref{JFromtETot}), the total field $\vb E$ may be 
expressed in terms of $\vb J$ according to 
%====================================================================%
\numeq{EFromJ}{\vb E = -ikZ_0 \mb V^{-1} \vb J.}
%====================================================================%
Using this in equation (\ref{JDEPFT}) yields
%====================================================================%
\begin{subequations}
\begin{align}
 P &= \frac{kZ_0}{2}\text{Im }\int \vb J^* \cdot \mb V^{-1} \cdot \vb J \, dV
\\
%--------------------------------------------------------------------%
 F_i &= -\frac{Z_0}{2c}
 \text{Re } 
 \int \vb J^* \cdot \partial_i \Big[\mb V^{-1} \cdot \vb J \Big] \,dV
\\
%--------------------------------------------------------------------%
 \mc T_i
&=-\frac{Z_0}{2c}\text{Re }\int
  \vb J^* \times \Big[ \mb V^{-1} \cdot \vb J\Big]\, dV
\end{align}
\label{OPFT}%
\end{subequations}%
%====================================================================%
The discretized versions of these formulas involve only overlap
integrals between SWG basis functions, which vanish unless the 
pairs of basis functions share one or more common tetrahedra.
Thus they amount to vector-matrix-vector products with 
highly sparse matrices and are thus, in principle, the most
computationally efficient technique for computing PFTs;
However, equation (\ref{EFromJ}) is only approximately satisfied
in a numerical solver, so 

I refer to (\ref{OPFT}) as the ``overlap PFT'' (OPFT) formulas.

%%%%%%%%%%%%%%%%%%%%%%%%%%%%%%%%%%%%%%%%%%%%%%%%%%%%%%%%%%%%%%%%%%%%%%
%%%%%%%%%%%%%%%%%%%%%%%%%%%%%%%%%%%%%%%%%%%%%%%%%%%%%%%%%%%%%%%%%%%%%%
%%%%%%%%%%%%%%%%%%%%%%%%%%%%%%%%%%%%%%%%%%%%%%%%%%%%%%%%%%%%%%%%%%%%%%
\newpage
\section{FVC approach to fluctuation-induced phenomena}

In this section I consider a collection of one or 
more material bodies $\{\mc B_n\}$, at various temperatures
$\{T_n\}$ and embedded in an environment at
temperature $T\subs{env}$, and derive a sequence of concise
matrix-trace formulas expressing thermally and quantum-mechanically 
averaged heat-transfer rates, forces, and torques on the bodies 
in terms of the $\vb T$ and $\vb G$ matrices discussed in the 
previous section. Because the resulting energy and momentum
transfers may be viewed as arising from fluctuations in volume 
currents in the bodies, I term this the 
``fluctuating volume-current'' (FVC) approach to fluctuation
physics.

The derivation proceeds in two steps.

\begin{enumerate}
%--------------------------------------------------------------------%
\item I first consider a fixed, deterministic volume electric 
current distribution $\vb J\supt{F}(\vb x)$---confined to the interiors
of our material bodies but otherwise arbitrary---and use
the VIE formalism of the previous section to derive compact 
expressions for the rates of energy and momentum absorption 
by the bodies. These expressions will be quadratic (bilinear) 
functions of $\vb J\supt{F}$. (The F superscript stands
for ``free''; it distinguishes the fixed, externally-imposed
current $\vb J\supt{F}$ from the \textit{induced} current
$\vb J\supt{I}$ to which it gives rise.)
%--------------------------------------------------------------------%
\item I then average over thermal and quantum-mechanical
fluctuations of $\vb J\supt{F}$ to derive temperature-dependent
mean heat-transfer rates and forces on the bodies.
%--------------------------------------------------------------------%
\end{enumerate}
In what follows I will go back and forth somewhat freely between
continuous operator/field notation 
[involving symbols like $\mathbb{G}$ and $\vb E(\vb x)$] and 
discretized matrix/vector notation 
(involving symbols like $\vb G$ and $\vb e$). For a precise
dictionary of the correspondence, see Appendix \ref{CorrespondenceAppendix}.

%%%%%%%%%%%%%%%%%%%%%%%%%%%%%%%%%%%%%%%%%%%%%%%%%%%%%%%%%%%%%%%%%%%%%%
%%%%%%%%%%%%%%%%%%%%%%%%%%%%%%%%%%%%%%%%%%%%%%%%%%%%%%%%%%%%%%%%%%%%%%
%%%%%%%%%%%%%%%%%%%%%%%%%%%%%%%%%%%%%%%%%%%%%%%%%%%%%%%%%%%%%%%%%%%%%%
\subsection{Energy and momentum transfer from volume-current bilinears}

Consider a collection of material bodies $\{\mc B_n\}$ and a 
fixed, deterministic volume current distribution $\vb J\supt{F}(\vb x)$
that is nonzero only inside the bodies.
(We work at a fixed frequency $\omega$ with all fields and currents 
varying in time like $e^{-i\omega t}.$)
In this section we derive formulas expressing time-average rates 
of energy and momentum absorption by the bodies as bilinear
functions of $\vb J\supt{F}.$

\subsubsection*{Induced currents from free currents}

The free current distribution $\vb J\supt{F}(\vb x)$ excites an 
induced current distribution $\vb J\supt{I}(\vb x)$ which we can 
determine using the VIE techniques of the previous section. 
Indeed, taking $\vb J\supt{F}$ as the source of the incident field 
in a scattering problem, we have
%====================================================================%
\begin{align}
 \vb E\sups{inc}(\vb r) &= ikZ_0 \mathbb{G} \star \vb J\supt{F}
\intertext{and the RHS vector of the discretized VIE system, equation
           (\ref{VIESystem2}), reads}
 \vb v                  &= -\vb G \, \vb j\supt{F}
\label{ScatProbRHS}
\end{align}
%====================================================================%
where, for a geometry consisting of $N$ bodies, the vectors and
matrices have an $N$-fold block structure:
%====================================================================%
\renewcommand{\arraystretch}{1.5}
$$ \vb v 
   = 
   \left(\begin{array}{c} 
     \vphantom{\vb j\supt{F}_1} \vb v_1 \\
     \vphantom{\vb j\supt{F}_2} \vb v_2 \\
     \vdots \\
     \vphantom{\vb j\supt{F}_N} \vb v_N
   \end{array}\right),
%--------------------------------------------------------------------%
\qquad  
   \vb G = \left(\begin{array}{cccc}
    \vphantom{\vb j\supt{F}_1} \vb G_{11} & \vb G_{12}  & \cdots & \vb G_{1N} \\
    \vphantom{\vb j\supt{F}_1} \vb G_{21} & \vb G_{22}  & \cdots & \vb G_{2N} \\
    \vdots                          & \vdots      & \ddots & \vdots     \\ 
    \vphantom{\vb j\supt{F}_1} \vb G_{N1} & \vb G_{N2}  & \cdots & \vb G_{NN}
   \end{array}\right), 
\qquad
%--------------------------------------------------------------------%
   \vb j\supt{F} 
   = 
   \left(\begin{array}{c} 
     \vb j\supt{F}_1 \\
     \vb j\supt{F}_2 \\ 
     \vdots    \\
     \vb j\supt{F}_N
   \end{array}\right).
$$
\renewcommand{\arraystretch}{1.0}
%====================================================================%
In particular, the $n$th subblock of $\vb j\supt{F}$ is the
projection of the free current distribution\footnote{Note that 
$\vb J_n\supt{F}(\vb x)$ is just the restriction of $\vb J\supt{F}$ 
to the interior of $\mc B_n$.} in body $\mc B_n$, 
$\vb J_n\supt{F}(\vb x)$, onto the subset of basis functions whose 
support lies in body $n$:
%====================================================================%
\numeq{jnF}
{ j_{n\alpha}\supt{F} 
  = 
  \int \vb b_{n\alpha}(\vb r) \cdot \vb J_n\supt{F}(\vb r) \, d\vb r.
}
%====================================================================%
Now taking equation (\ref{ScatProbRHS}) to be the RHS of the VIE scattering
problem (\ref{VIESystem2}), we obtain an expression for the induced
currents in terms of the free currents, 
%====================================================================%
\begin{align}
  \vb M \, \vb j\supt{I} &= - \vb G \vb j\supt{F} 
\\
\intertext {or}
  \vb j\supt{I} &= -\vb W \vb G \vb j\supt{F} 
\label{jiFromjf}
\end{align}
%====================================================================%
where $\vb W=\vb M^{-1}$ is the inverse of the VIE matrix 
defined by (\ref{VIESystem2}).

The \textit{total} current is 
\begin{align}
 \vb j &= \vb j\supt{F} + \vb j\supt{I}
\nn
  &= \Big[ \vb 1 - \vb W \vb G\Big] \vb j\supt{F}.
\label{jTot}
\end{align}

\subsubsection*{Fields from free currents}

The $\vb E$-field at an arbitrary point in space (either
inside or outside a body) is then simply the sum of 
contributions from fixed and induced currents:
%====================================================================%
\begin{align}
 \vb E &= ikZ_0 \mathbb{G} \star (\vb J\supt{F} + \vb J\supt{I})
\label{EConvolution} \\
\intertext{or, in discretized form (Appendix \ref{CorrespondenceAppendix}),}
 \vb e &= ikZ_0 \vb G (\vb j\supt{F} + \vb j\supt{I}) 
\nn
       &= ikZ_0 \vb G \Big( \vb 1 - \vb W \vb G \Big) \vb j\supt{F}
\label{efromjf}
\end{align}
%====================================================================%
where in going to the last line I used (\ref{jiFromjf}).

In what follows I will also need the the quantity 
$\partial_i \vb E(\vb r),$ i.e. the derivative of $\vb E$ with
respect to the evaluation point. Differentiating both sides of
(\ref{EConvolution}), we see that the derivative operates
on the first argument of $\mathbb{G}(\vb r, \vb r^\prime)$
and leaves everything else on the RHS untouched; thus we find
simply 
%====================================================================%
\numeq{diefromfj}
{\partial_i \vb e = ikZ_0 \Big[\partial_i \vb G\Big] (\vb j\supt{F} + \vb j\supt{I})}
%====================================================================%
where the matrix elements of the quantity in square brackets are
$$ \Big[ \partial_i \vb G\Big]_{\alpha\beta}
  = \int \int \vb b_\alpha(\vb r) 
              \left[\pard{}{\vb r_i} \mathbb{G}(\vb r, \vb r^\prime)\right]
              \vb b_\beta(\vb r^\prime) \, d\vb r \, d\vb r^\prime.
$$

\subsubsection*{Power absorption}

The time-average rate at which body $\mc B_n$ absorbs power from
the source distribution $\vb J\supt{F}$ is obtained by integrating
the quantity $\frac{1}{2}\text{Re }\vb J^* \cdot \vb E$ over the 
interior of $\mc B_n$; here $\vb J$ is the \textit{total} current 
in $\mc B_n$, consisting of both free and induced contributions:
%====================================================================%
\begin{align}
 P_n(\omega) 
 &= 
 \frac{1}{2}\text{Re }\int_{\mc B_n} \vb J^*(\vb r) \cdot \vb E(\vb r) \, d\vb r
\label{PowerVolumeIntegral} \\
 &= \frac{1}{2}\text{Re }\vb j_n^\dagger \cdot \vb e_n \nonumber
\intertext{where we used equation (\ref{VolIntToDotProd}). 
           The $n$ subscript on vectors picks out the subblock
           corresponding to body $n$; using the projection matrices
           $\vb P_n$ defined by (\ref{ProjectionMatrix}), we 
           could equivalently write this in the form}
 &= \frac{1}{2}\text{Re }\vb j^\dagger \vb P_n \vb e \nonumber
\intertext{Now insert equations (\ref{jTot}) and (\ref{efromjf}):}
 &= \frac{1}{2}\text{Re }\left\{ ikZ_0 \, 
    \vb j\fd \Big[\vb 1-\vb G^\dagger \vb W^\dagger \Big]
    \vb P_n
    \vb G \Big[\vb 1 - \vb W \vb G\Big] \vb j\supt{F} \right\}
\nn
 &= -\frac{kZ_0}{2}\text{ Im Tr }\left\{ 
    \Big[\vb 1-\vb G^\dagger \vb W^\dagger \Big]
    \vb P_n
    \vb G \Big[\vb 1 - \vb W \vb G\Big] 
    \Big[\vb j\supt{F} \vb j\fd \Big]
    \right\}.
\label{DeterministicPowerTraceFormula}
\end{align}
%====================================================================%
As advertised, this expression depends quadratically on
$\vb J\supt{F}$, as witness the appearance of the outer matrix
product $\vb j\supt{F} \vb j\fd$.

%=================================================
%=================================================
%=================================================
\subsubsection*{Momentum absorption}

The time-average rate at which body $\mc B_n$ absorbs $i$-directed
\textit{momentum} from the source distribution $\vb J\supt{F}$---that
is, the $i$-directed force on the body---may be expressed as a
volume-integral expression very similar to that of
(\ref{PowerVolumeIntegral})
but with 
 \textbf{(i)} ``Re'' replaced by ``Im,''
 \textbf{(ii)} $\vb E$ replaced by $\partial_i \vb E$, 
and 
 \textbf{(iii)} an extra factor of $\omega$ in the denominator:
%====================================================================%
\begin{align*}
 T_n(\omega) 
 &= 
 \frac{1}{2\omega}
 \text{Im }\int_{\mc B_n} \vb J^*(\vb r) \cdot \partial_i \vb E(\vb r) d\vb r.
\intertext{(This expression is quoted in~\citeasnoun{Kruger2012}; I
            also provide a quick derivation in
            Appendix \ref{ForceFormulaAppendix}.)
            Going over to the discrete world, we have}
 &= 
 \frac{1}{2\omega}
 \text{Im }\vb j^\dagger \vb P_n (\partial_i \vb e)
\intertext{Insert (\ref{jTot}) and (\ref{efromjf}):}
 &= 
\frac{1}{2\omega}\text{Im }
    \left\{ ikZ_0 \, 
    \vb j\fd \Big[\vb 1-\vb G^\dagger \vb W^\dagger \Big]
    \vb P_n
    (\partial_i \vb G) \Big[\vb 1 - \vb W \vb G\Big] \vb j\supt{F} \right\}
\nn
&= \frac{kZ_0}{2\omega}\text{Re Tr}
    \left\{ 
    \Big[\vb 1-\vb G^\dagger \vb W^\dagger \Big]
    \vb P_n
    (\partial_i \vb G) \Big[\vb 1 - \vb W \vb G\Big] \vb j\supt{F} \vb j\fd
   \right\}
\end{align*}
%====================================================================%

%%%%%%%%%%%%%%%%%%%%%%%%%%%%%%%%%%%%%%%%%%%%%%%%%%%%%%%%%%%%%%%%%%%%%%
%%%%%%%%%%%%%%%%%%%%%%%%%%%%%%%%%%%%%%%%%%%%%%%%%%%%%%%%%%%%%%%%%%%%%%
%%%%%%%%%%%%%%%%%%%%%%%%%%%%%%%%%%%%%%%%%%%%%%%%%%%%%%%%%%%%%%%%%%%%%%
\subsection{Statistical averages of volume-current bilinears}

The classical, deterministic expressions derived above for 
time-average quantities $Q$ (where $Q$ is a power, force,
or torque) all take the form
%====================================================================%
\numeq{QofW}
{Q(w) 
   \propto 
   \Tr \Big\{ \vb Q(\omega) \cdot \big(\vb j\supt{F} \vb j\fd\big) \Big\}
}
%====================================================================%
where $\vb Q(\omega)$ is a frequency-dependent matrix. The
statistical \textit{average} of such quantities is performed
by averaging over all possible free current distributions
$\vb J\supt{F}(\vb r)$, which amounts to computing the statistical
average of the matrix $\vb j\supt{F} \vb j\fd$:
%====================================================================%
\begin{align*}
 \big\langle Q(\omega) \big\rangle
   \propto 
   \Tr \Big\{ \vb Q(\omega) \cdot 
              \big\langle \vb j\supt{F} \vb j\fd\big\rangle_{\omega} 
       \Big\}.
\end{align*}
This quantity represents just the contribution of frequency-$\omega$
fluctuations to the average power, force or torque (PFT); the  
\textit{total} PFT is given by integrating over all frequencies,
$ Q=\int_{0}^\infty \, \langle Q(\omega) \rangle \,d\omega.$

%====================================================================%
The elements of the matrix $\vb j\supt{F}\vb j\fd$ are 
%====================================================================%
\begin{align}
 \Big[ \vb j\supt{F} \vb j\fd \Big]_{\alpha\beta}
&=\int \int     b_{\alpha i}(\vb r) 
                J\supt{F}_i(\vb r) J^{\text{\tiny{F}}*}_j(\vb r^\prime)
                b_{\beta j}(\vb r^\prime)
  \, d\vb r \, d\vb r^\prime
%--------------------------------------------------------------------%
\intertext{Now perform the statistical average. The only quantities
on the RHS that experience averaging are the factors of $J$ in the
integrand:}
%--------------------------------------------------------------------%
 \Big\langle \vb j\supt{F} \vb j\fd \Big\rangle_{\alpha\beta}
&=\int \int     b_{\alpha i}(\vb r) 
 \Big\langle
                J\supt{F}_i(\vb r) J^{\text{\tiny{F}}*}_j(\vb r^\prime)
 \Big\rangle
                b_{\beta j}(\vb r^\prime)
  \, d\vb r \, d\vb r^\prime.
\label{jjd1}
\end{align}
We now make use of the 
fluctuation-dissipation theorem in the form of the Rytov correlation
function,\footnote{How do the \textit{units} of this equation work?
To answer this question, I think it's easiest to multiply both
sides by $Z_0$ to bring a factor of impedance to the LHS.
Since $J$ is the Fourier transform of a volume current density, it
has units of 
$\frac{\text{current}}{\text{length}^2}\cdot \frac{1}{\text{frequency}}$.
and the LHS then has units of 
$\frac{\text{impedance} \cdot \text{current}^2 \cdot \text{time}^2}
      {\text{length}^4}
 =\frac{\text{energy}\cdot\text{time}}
      {\text{length}^4}
$ where we used that $\text{impedance}\cdot\text{current}^2=\text{power}$
(for example, recall the $\sim I^2 R$ dependence of Joule heating)
and $\text{power}\cdot\text{time}=\text{energy}$. Meanwhile, on
the RHS, the dimensionful factors remaining after multiplying by 
$Z_0$ are $\Theta$ (energy) and $k\delta(\vb r-\vb r^\prime)$ 
(length$^{-4}$) so everything works out modulo a factor of inverse 
frequency on the RHS, which I think must be coming from a factor
like $\delta(\omega-\omega^\prime)$ that is implicit somewhere.}
%====================================================================%
\numeq{Rytov}
{ \big\langle J\supt{F}_i(\vb r) J^{\text{\tiny{F}}*}_j(\vb r^\prime)
  \big\rangle
  = \frac{2k}{\pi Z_0} \Theta(T) \delta(\vb r-\vb r^\prime) 
    \text{Im }\epsilon_{ij}(\vb r).
}
%====================================================================%
Inserting this into (\ref{jjd1}), we find
%====================================================================%
$$
 \Big\langle \vb j\supt{F} \vb j\fd \Big\rangle_{\alpha\beta}
 = \frac{2k}{\pi Z_0}
     \int \Theta\big[T(\vb r)\big]
          b_{\alpha i}(\vb r) 
          \Big[\text{Im }\epsilon_{ij}(\vb r) \Big]
          \vb b_{\beta  j}(\vb r)
  \, d\vb r. 
$$
%====================================================================%
For the situation we consider here---involving $N$ material bodies,
throughout the interior of which the temperature is constant---the
full matrix takes the form 
%====================================================================%
\numeq{jjdMatrix}
{
 \Big\langle \vb j\supt{F} \vb j\fd \Big\rangle
 =\frac{2k}{\pi Z_0}
  \left(\begin{array}{cccc}
  \Theta(T_1) \vbIEM_1 & 0 & \cdots & 0 \\ 
  0 & \Theta(T_2) \vbIEM_2 & \cdots & 0 \\ 
  \vdots & \vdots & \ddots & \vdots   \\
  0 & 0 & \cdots & \Theta(T_N) \vbIEM_N \\
  \end{array}\right)
}
%====================================================================%
where $\vbIEM_n$ is just the matrix of basis-function overlaps
with the imaginary part of the relative dielectric function
for the $n$th body, i.e.
%====================================================================%
\numeq{IEM}
{ \Big[\vbIEM_n\Big]_{\alpha\beta} = 
   \int b_{\alpha i}(\vb r) \Big[ \text{Im } \vbeps_{n}(\vb r)\Big]_{ij}
        b_{\beta j}(\vb r) d\vb r.
}
%====================================================================%
Here $\vbeps_n$ is the dielectric tensor for body $\mc B_n.$
For a basis of localized functions, this matrix is highly sparse
(for the SWG basis discussed below it contains just 7 nonzero elements 
per row). Moreover, numerical evaluation of the matrix elements of 
$\vbIEM$ is essentially costless, particularly compared to the cost of 
computing matrix elements of $\mathbb{G}$;
it involves just a single three-dimensional numerical cubature
and may be carried out simultaneously with computation of the 
matrix elements of the $\bbVInv$ operator needed to assemble the 
VIE matrix.

Inserting (\ref{jjdMatrix}) into ... 

The total heat transfer to, and the total $i$-directed 
force and torque on, a destination body $\mc B_d$ are given by
%====================================================================%
\begin{align*}
 H_d    = \int_0^\infty \mc H_d(\omega) d\Omega,         \qquad 
 F_{di} = \int_0^\infty \mc F_{di}(\omega) d\Omega, \qquad 
 T_{di} = \int_0^\infty \mc T_{di}(\omega) d\Omega
\end{align*}
%====================================================================%
where the contribution of each frequency $\omega$ may be written
as the sum of equilibrium contributions plus non-equilibrium
contributions from all other bodies acting as sources, with 
each contribution expressed as a thermal/statistical factor 
times a generalized flux: 
%====================================================================%
\begin{align*}
 \mc H_d(\omega)
 &= \sum_{s} 
    \Big[ \Theta\big(\omega,T_s) - \Theta\big(\omega, T\subs{env}\big) \Big]
    \Phi\sups{energy}_{s\to d}(\omega)
\\
 \mc F_{di}(\omega)
 &= F_{di}\sups{eq}\big( T\subs{env} \big)
   +\sum_{s} 
    \Big[ \Theta\big(\omega,T_s) - \Theta\big(\omega, T\subs{env}\big) \Big]
    \Phi\sups{lin mom}_{s\to d}(\omega)
\\
 \mc T_{di}(\omega)
 &= T_{di}\sups{eq}\big( T\subs{env} \big)
   +\sum_{s} 
    \Big[ \Theta\big(\omega,T_s) - \Theta\big(\omega, T\subs{env}\big) \Big]
    \Phi\sups{ang mom}_{s\to d}(\omega)
\end{align*}
%====================================================================%
(For the heat transfer there is no equilibrium contribution, as there
is no net exchange of energy between equal-temperature bodies. There
\textit{is} a net transfer of \textit{momentum}, whereupon the force and
torque expressions do contain equilibrium contributions.)
%====================================================================%
\begin{align*}
 \Phi\sups{energy}_{s\to d}(\omega)
&=-\frac{k^2}{\pi}\text{Im Tr }
   \left\{ 
           \Big[ \vb G (\vb 1 - \vb W \vb G)\Big]_{ds}
           \vbSigma_s
           \Big[\vb 1 - \vb G^\dagger \vb W^\dagger\Big]_{sd}
   \right\}
\\
 \Phi\sups{lin mom}_{s\to d}(\omega)
&= \frac{k^2}{\pi \omega}\text{Re Tr }
   \left\{ 
           \Big[ \big(\partial_i \vb G\big) (\vb 1 - \vb W \vb G)\Big]_{ds}
           \vbSigma_s
           \Big[\vb 1 - \vb G^\dagger \vb W^\dagger\Big]_{sd}
   \right\}
\\
 \Phi\sups{ang mom}_{s\to d}(\omega)
&= \frac{k^2}{\pi \omega}\text{Re Tr }
   \left\{ 
           \Big[ \big(\partial_\theta \vb G\big) (\vb 1 - \vb W \vb G)\Big]_{ds}
           \vbSigma_s
           \Big[\vb 1 - \vb G^\dagger \vb W^\dagger\Big]_{sd}
   \right\}
\end{align*}
%====================================================================%
The trace we compute in these expressions has the form
%====================================================================%
\begin{align*}
\text{Tr }\Big[ \left(\vb X \vb A\right)_{ds}
                \vbSigma_s
                \big(\vb A^\dagger\big)_{sd} 
          \Big]
&= \sum_{ijk} \left(\vb X \vb A\right)_{ds, ij}
              \Sigma_{s,jk}
              \Big[\vb A^\dagger_{sd}\Big]_{ki}
\\
&= \sum_{ijk} \big(\vb X \vb A\big)_{di; sj}
              \Sigma_{s,jk}
              A^*_{di;sk}
\end{align*}
%====================================================================%
where 
%====================================================================%
$$\vb A = \Big( \vb 1 - \vb W \vb G\Big)$$
%====================================================================%
and
%====================================================================%
$$ \vb X=\{\vb G, \partial_i \vb G, \partial_\theta \vb G\}.$$
%====================================================================%
%For example, suppose in a four-object geometry we want the 
%force on object 2 due to sources in object 3.
%%====================================================================%
%$$ \left(\begin{array}{cccc}
%   \vb A^\dagger_{11} & \vb A^\dagger_{12} & \vb A^\dagger_{13} & \vb A^\dagger_{14} \\
%   \vb A^\dagger_{21} & \vb A^\dagger_{22} & \vb A^\dagger_{23} & \vb A^\dagger_{24} \\
%   \vb A^\dagger_{31} & \vb A^\dagger_{32} & \vb A^\dagger_{33} & \vb A^\dagger_{34} \\
%   \vb A^\dagger_{41} & \vb A^\dagger_{42} & \vb A^\dagger_{43} & \vb A^\dagger_{44}
%   \end{array}\right)
%%--------------------------------------------------------------------%
%   \left(\begin{array}{cccc}
%   \vb X_{11} & \vb X_{12} & \vb X_{13} & \vb X_{14} \\
%   \red{\vb X_{21}} & \red{\vb X_{22}} & \red{\vb X_{23}} & \red{\vb X_{24}} \\
%   \vb X_{31} & \vb X_{32} & \vb X_{33} & \vb X_{34} \\
%   \vb X_{41} & \vb X_{42} & \vb X_{43} & \vb X_{44}
%   \end{array}\right)
%%--------------------------------------------------------------------%
%   \left(\begin{array}{cccc}
%   \vb A_{11} & \vb A_{12} & \red{\vb A_{13}} & \vb A_{14} \\
%   \vb A_{21} & \vb A_{22} & \red{\vb A_{23}} & \vb A_{24} \\
%   \vb A_{31} & \vb A_{32} & \red{\vb A_{33}} & \vb A_{34} \\
%   \vb A_{41} & \vb A_{42} & \red{\vb A_{43}} & \vb A_{44}
%   \end{array}\right)
%$$
%%====================================================================%

%%%%%%%%%%%%%%%%%%%%%%%%%%%%%%%%%%%%%%%%%%%%%%%%%%%%%%%%%%%%%%%%%%%%%%
%%%%%%%%%%%%%%%%%%%%%%%%%%%%%%%%%%%%%%%%%%%%%%%%%%%%%%%%%%%%%%%%%%%%%%
%%%%%%%%%%%%%%%%%%%%%%%%%%%%%%%%%%%%%%%%%%%%%%%%%%%%%%%%%%%%%%%%%%%%%%
\addcontentsline{toc}{section}{\large Part II: Implementation}
\newpage
\section{SWG Basis Functions}

SWG basis functions are the three-dimensional analog of
RWG basis functions. They are defined on pairs of adjacent tetrahedra:
%====================================================================%
$$ \vb b_\alpha(\vb x) = 
  \begin{cases}
   \displaystyle{
    +\frac{A_\alpha}{3V_\alpha^+}(\vb x - \vb Q_\alpha^+), 
                }
    \qquad &\vb x\in \mc P_\alpha^+
\\[5pt]
   \displaystyle{
  -\frac{A\alpha}{3V_\alpha^-}(\vb x - \vb Q_\alpha^-), 
                } 
    \qquad &\vb x\in \mc P_\alpha^-
  \end{cases}
$$
%====================================================================%
where $\mc P_\alpha^\pm$ are the two tetrahedra associated with basis
function $\alpha$, $V_\alpha^\pm$ are their volumes, $\vb Q_\alpha\pm$
are the source/sink vertices, and $A_\alpha$ is the area of the 
triangular face shared by $\mc P_\alpha^\pm$. (I denote tetrahedra 
by $\mc P$, which stands for ``pyramid,'' to avoid confusion with
the symbol $\mc T$, which stands for ``triangle'' in my memos on
RWG basis functions.)

The divergence of the SWG basis function is 
%====================================================================%
$$ \nabla \cdot \vb b_\alpha(\vb x) = 
    \pm \frac{A_\alpha}{V_\alpha^\pm}, \qquad \vb x\in \mc P_\alpha^\pm.
$$

%%%%%%%%%%%%%%%%%%%%%%%%%%%%%%%%%%%%%%%%%%%%%%%%%%%%%%%%%%%%%%%%%%%%%%
%%%%%%%%%%%%%%%%%%%%%%%%%%%%%%%%%%%%%%%%%%%%%%%%%%%%%%%%%%%%%%%%%%%%%%
%%%%%%%%%%%%%%%%%%%%%%%%%%%%%%%%%%%%%%%%%%%%%%%%%%%%%%%%%%%%%%%%%%%%%%
\newpage
\section{SWG Matrix Elements of the $\mathbb{G}$ operator}

%=================================================
%=================================================
%=================================================
\subsection{Distant case: Volume-integral method}

The $\mathbb{G}$-matrix element between two SWG basis functions
is
%====================================================================%
\numeq{bGb}
{
 \exptwoB{\vb b_\alpha}{\mathbb G}{\vb b_\beta}
 =\int_{\sup \vb b_\alpha} \, d\vb x_\alpha \,
  \int_{\sup \vb b_\beta} \, d\vb x_\beta\,
   b_{\alpha i}(\vb x_\alpha)
   \mathbb G_{ij}(\vb R_0 + \overline{\vb x}_\alpha - \overline{\vb x}_\beta)
   b_{\beta j}(\vb x_\beta)
}
%====================================================================%
where 
%====================================================================%
$$ \overline{\vb x}_\alpha \equiv \vb x_\alpha - \vb x_{\alpha 0}, 
   \qquad 
   \overline{\vb x}_\beta \equiv \vb x_\beta - \vb x_{\beta 0}, 
   \qquad 
   \vb R_0 = \vb x_{\alpha 0}-\vb x_{\beta 0}
$$
%====================================================================%
and $\vb x_{\alpha 0}, \vb x_{\beta 0}$ are the centroids of
the basis functions.

When the two basis functions are well separated 
(i.e. $|\vb R_0| \gg |\overline{\vb x}_\alpha|, |\overline{\vb x}_\beta|$),
we may compute (\ref{bGb}) to sufficient accuracy using a volume-integral
method: 
%====================================================================%
\numeq{bGb}
{
 \exptwoB{\vb b_\alpha}{\mathbb G}{\vb b_\beta}
 =\sum \pm \int_{\mc P_\alpha^\pm } d\vb x_\alpha
           \int_{\mc P_\beta^\pm  } d\vb x_\beta
           \left[ \vb b_\alpha \cdot \vb b_\beta
                 -\frac{9}{k^2}
           \right] \Phi\Big( \big|\vb R_0 + \overline{\vb x}_\alpha
                                - \overline{\vb x}_\beta
                             \big|\Big)
}
%====================================================================%
where $\Phi(r) = \frac{e^{ik|\vb r|}}{4\pi |\vb r|}.$
in which the 6-dimensional integration over each of the
four pairs of tetrahedra is carried out by simple low-order
numerical cubature, as discussed in Appendix \ref{VolumeIntegralAppendix}.

%=================================================
%=================================================
%=================================================
\subsection{Nearby case: Desingularization}

When the two basis functions have one or more common vertices,
the integral (\ref{bGb}) is singular. In this case,
{\sc buff-em} follows the standard desingularization strategy:
the first few singular terms in the small-$r$ expansion of
$\Phi(r)$ are subtracted off, leaving \textbf{(a)} a 
desingularized version of (\ref{bGb}) which is evaluated by 
low-order cubature, and \textbf{(b)} a collection of 
singular but frequency-independent integrals.

All singular integrals computed by {\sc buff-em} for
an object described by a given tetrahedral mesh are
automatically stored in a binary data file named 
\texttt{Mesh.cache}, (where \texttt{Mesh.vmsh} is 
the name of the file from which the mesh was read). 
Te {\sc buff-em} core library will automatically
look for this file when it needs singular integrals;
if the file is not found, the singular integrals are 
computed on the fly and automatically stored in
the file \texttt{Mesh.cache}. (Thus, the caching of
singular integrals is more transparent to the user
than is the case in {\sc scuff-em}.)

%=================================================
%=================================================
%=================================================
\subsection{Evaluation of singular integrals: Taylor-Duffy Method}

Singular 6-dimensional integrals in {\sc buff-em} are evaluated
by a complicated technique obtained by generalizing the
Taylor-Duffy method for triangles to the case of tetrahedra.
 
%%%%%%%%%%%%%%%%%%%%%%%%%%%%%%%%%%%%%%%%%%%%%%%%%%%%%%%%%%%%%%%%%%%%%%
%%%%%%%%%%%%%%%%%%%%%%%%%%%%%%%%%%%%%%%%%%%%%%%%%%%%%%%%%%%%%%%%%%%%%%
%%%%%%%%%%%%%%%%%%%%%%%%%%%%%%%%%%%%%%%%%%%%%%%%%%%%%%%%%%%%%%%%%%%%%%
\appendix 

%%%%%%%%%%%%%%%%%%%%%%%%%%%%%%%%%%%%%%%%%%%%%%%%%%%%%%%%%%%%%%%%%%%%%%
%%%%%%%%%%%%%%%%%%%%%%%%%%%%%%%%%%%%%%%%%%%%%%%%%%%%%%%%%%%%%%%%%%%%%%
%%%%%%%%%%%%%%%%%%%%%%%%%%%%%%%%%%%%%%%%%%%%%%%%%%%%%%%%%%%%%%%%%%%%%%
\newpage
\section{Dictionary of the operator--matrix correspondence}
\label{CorrespondenceAppendix}

Infinite-dimensional position-space basis 
$\Longleftrightarrow$ 
$N\subt{B}$-dimensional basis of discrete expansion functions
$\{\vb b_\alpha(\vb x)\}.$ 

\subsection*{Notation}

\begin{itemize}
  \item We use blackboard-bold symbols for position-space 
        operators: $\mathbb{G}, \mathbb{T}$.
  \item We use upper-case bold letters for 
        discrete-basis matrices: $\vb T, \vb G, \vb W.$
        The elements of these matrices are, e.g.
        %====================================================================%
        $$ G_{\alpha\beta} \equiv \int \int 
           \vb b_\alpha(\vb r) \mathbb{G}(\vb r, \vb r^\prime)
           \vb b_\beta(\vb r^\prime) \, d\vb r \, d\vb r^\prime
        $$
        %====================================================================%
  \item (At the risk of confusion with the previous item) 
        We use upper-case bold letters for position-space
        vectors: $\vb J(\vb x), \vb E(\vb x).$
  \item We use lower-case bold letters for
        discrete-basis vectors: $\vb j, \vb e.$ The elements
        of these vectors are e.g.
        %====================================================================%
        $$ e_{\alpha} 
           \equiv 
           \int \vb b_\alpha(\vb r) \cdot \vb E(\vb r) d\vb r.
        $$
        %====================================================================%
\end{itemize}

%====================================================================%
\renewcommand{\arraystretch}{1.5}
$$\begin{array}{|c|c|}\hline
   \textbf{Continuous} 
 & \textbf{Discrete} 
\\\hline
%--------------------------------------------------------------------%
  \mathbb{G}(\vb r, \vb r^\prime)
& 
  \vb G\text{ matrix, with elements }
  U_{\alpha\beta} 
  = \int \int \vb b_\alpha(\vb r) \mathbb{G}(\vb r, \vb r^\prime)
              \vb b_\beta(\vb r^\prime) \, d\vb r \, d\vb r^\prime
\\\hline
%--------------------------------------------------------------------%
  \vb{E}(\vb r)
&
  \vb e\text{ vector, with elements }
  e_{\alpha} 
  = \int \vb b_\alpha(\vb r) \cdot \vb {E}(\vb r) \, d\vb r
\\\hline
%--------------------------------------------------------------------%
\end{array}$$
%====================================================================%

\subsection*{Approximate completeness relation}

One way to conceptualize the transition from the continuous
to the discrete is to suppose that the basis functions 
$\{\vb b_\alpha(\vb x)\}$ satisfy an approximate completeness relation
of the form 
%====================================================================%
\begin{subequations}
\begin{align}
\sum_{\alpha} b_{\alpha i}(\vb r) b_{\alpha j}(\vb r^\prime)
&\approx \delta_{ij} \delta(\vb r-\vb r^\prime)
\intertext{or}
\sum_\alpha \int b_{\alpha i}(\vb r) b_{\alpha j}(\vb r^\prime) \, d\vb r^\prime
&=\delta_{ij}
\end{align}
\label{SWGCompleteness}
\end{subequations}
%====================================================================%

\subsection*{Convolutions becomes matrix-vector products}
For example, consider the continuous version of the equation
relating the total current to the total field:
%====================================================================%
\begin{align}
  \vb E &= ikZ_0 \mathbb{G} \star \vb J 
\intertext{or}
  E_i(\vb r) &= ikZ_0 \int \mathbb{G}_{ij}(\vb r, \vb r^\prime) 
                J_j(\vb r^\prime) 
                d\vb r^\prime.
\intertext{Insert (\ref{SWGCompleteness}a) [in the form 
           $\sum b_{\beta j}(\vb r^\prime) 
                 b_{\beta k}(\vb r^{\prime\prime})
                 =\delta_{jk}\delta(\vb r^\prime-\vb r^{\prime\prime})
           $]between $\mathbb{G}$ and $\vb J$ on the RHS:}
  E_i(\vb r) &= ikZ_0 \sum_{\beta }
                \left[
                \int \mathbb{G}_{ij}(\vb r, \vb r^\prime) 
                     b_{\beta j}(\vb r^\prime)
                     d\vb r^\prime
                \right]
                \underbrace{\left[
                \int b_{\beta k}(\vb r^{\prime\prime}) 
                     J_k(\vb r^{\prime\prime})
                     d\vb r^{\prime\prime}
                            \right]}_{\vb j_\beta}
\intertext{As it stands this equation exists in a sort of 
           hybrid continuous-discrete form. 
           Now multiply both sides by $\vb b_{\alpha}(\vb r)$ 
           and integrate over $\vb r$ to find}
 e_\alpha &= ikZ_0 G_{\alpha\beta} j_\beta
\intertext{or}
 \vb e &= ikZ_0 \vb G\, \vb j.
\end{align}
%====================================================================%

%=================================================
%=================================================
%=================================================
\subsection*{Volume integrals become dot products}

Consider, for example, the integral
%====================================================================%
\begin{align}
\Big\langle \vb J \cdot \vb E \Big\rangle_{\mc B_n}
 &\equiv
 \int_{\mc B_n} \vb J^*(\vb r) \cdot \vb E(\vb r) \, d\vb r
\nonumber
\intertext{Rewrite this in the seemingly pedantic form}
 &= 
 \int_{\mc B_n} \int_{\mc B_n} 
 J_i^*(\vb r) \delta_{ij} \delta(\vb r-\vb r^\prime)
                E_j(\vb r^\prime) \, d\vb r \, d\vb r^\prime
\nonumber
\intertext{Now insert (\ref{SWGCompleteness}a):}
 &= 
 \int_{\mc B_n} \int_{\mc B_n} 
  J_i^*(\vb r) 
  \left[ \sum_{\alpha} b_{\alpha i}(\vb r) b_{\alpha j}(\vb r^\prime)\right]
  E_j(\vb r^\prime) \, d\vb r \, d\vb r^\prime
\nn
 &= 
 \sum_{\alpha}
 \underbrace{\Big[
 \int_{\mc B_n} b_{\alpha i}(\vb r)J_i^*(\vb r)  \, d\vb r 
             \Big]
            }_{j_{n\alpha}^*}
 \underbrace{\Big[
 \int_{\mc B_n} b_{\alpha j}(\vb r^\prime) E_j(\vb r^\prime) \, d\vb r^\prime 
             \Big]
            }_{e_{n\alpha}}
 &= \vb j_n^\dagger \vb e_n.
\label{VolIntToDotProd}
\end{align}
%====================================================================%

%%%%%%%%%%%%%%%%%%%%%%%%%%%%%%%%%%%%%%%%%%%%%%%%%%%%%%%%%%%%%%%%%%%%%%
%%%%%%%%%%%%%%%%%%%%%%%%%%%%%%%%%%%%%%%%%%%%%%%%%%%%%%%%%%%%%%%%%%%%%%
%%%%%%%%%%%%%%%%%%%%%%%%%%%%%%%%%%%%%%%%%%%%%%%%%%%%%%%%%%%%%%%%%%%%%%
\newpage
\section{Derivation of volume integrals for the force and torque}
\label{ForceFormulaAppendix}

Consider a body in which exists both a (deterministic) 
total volume current distribution $\vb J(\vb x)$ and
electric and magnetic fields $\{\vb E, \vb H\}(\vb x).$
The time-average force experienced by the currents in 
an infinitesimal volume $dV$ is
%====================================================================%
\begin{align}
d\vb F &= \frac{1}{2}\text{Re }
 \Big[ \rho^* \vb E + \mu_0 \vb J^* \times \vb H \Big] \, dV
\nn
\intertext{Use $\rho=\frac{1}{i\omega}(\nabla \cdot \vb J)$
           and $\vb H=\frac{1}{i\omega\mu_0}\nabla \times \vb E$:}
 &=
 \frac{1}{2}\text{Re }\left\{
 \frac{1}{i\omega}
 \Big[ -(\nabla \cdot \vb J^*) \vb E 
       + \vb J^* \times (\nabla \times \vb E) 
 \Big] \,
                      \right\} dV
\nonumber
\intertext{or}
dF_i &= -\frac{1}{2\omega}\text{Im }
 \Big[
 (\partial_j J^*_j) E_i - 
  \underbrace{ \varepsilon_{ijk}
               \varepsilon_{k\ell m}
             }_{\delta_{i\ell}\delta_{jm} - \delta_{im}\delta_{j\ell}}
  J^*_j \partial_\ell E_m
 \Big] \, dV
\nn
&= -\frac{1}{2\omega}\text{Im }
 \Big[ (\partial_j J^*_j) E_i
       - J^*_j \partial_i E_j
       + J^*_j \partial_j E_i
 \Big] \, dV.
\label{dFi}
\intertext{The total force is given by integrating over the volume:}
F_i
&= -\frac{1}{2\omega}\text{Im } \int_{\mc B_n}
 \Big[ (\partial_j J^*_j) E_i
       - J^*_j \partial_i E_j
       + J^*_j \partial_j E_i
 \Big] dV
\label{Fi}
\end{align}
%====================================================================%
The first and third terms here together read
%====================================================================%
\numeq{Argument}
{
 \int \partial_j \big(J_j^* E_i\big ) dV 
 = \int \nabla \cdot (E_i \, \vb J^*) \, dV
 = \oint E_i \vb J^* \cdot d\vb A = 0
}
%====================================================================%
because $\vb J\cdot \vbhat{n}=0$ at the surface of the object 
(no current flows from the body into space). 
Thus only the middle term in (\ref{Fi}) is nonvanishing,
and we find simply
%====================================================================%
\numeq{ForceVolumeIntegral}
{ F_i = \frac{1}{2\omega}\text{Im } \int_{\mc B_n}
         J^*_j \partial_i E_j \, dV 
      = \frac{1}{2\omega}\text{Im } 
         \int_{\mc B_n} \vb J^* \cdot \partial_i \vb E \, dV.
}
%====================================================================%

\subsubsection*{Torque}

The contribution of currents in $dV$ to the \textit{torque} 
about an origin $\vb r_0$ is given by
%====================================================================%
\begin{align} 
 d\bmc T &= (\vb r - \vb r_0) \times d \vb F
\nonumber
\intertext{or, in components,}
 d\mc T_i &= \varepsilon_{ijk} (\vb r - \vb r_0)_j d \vb F_k.
\nonumber
\intertext{Insert (\ref{dFi}):}
&= -\frac{1}{2\omega}\text{Im}
 \left\{
 \varepsilon_{ijk} (\vb r - \vb r_0)_j
 \Big[ (\partial_\ell J^*_\ell) E_k
       - J^*_\ell \partial_k E_\ell
       + J^*_\ell \partial_\ell E_k
 \Big]\right\} \, dV 
\label{dTorque}
\end{align} 
%====================================================================%
%= 20150417
%====================================================================%
The volume integral of the second term reads
%====================================================================%
$$ \mc T_i^{(2)} = +\frac{1}{2\omega}
   \text{Im }\int \vb J^* \cdot \partial_{\theta_i} \vb E \, dV.
$$
where the symbol $\partial_{\theta_i} \vb E$ denotes the derivative
of $\vb E(\vb r)$ with respect to an infinitesimal rotation of the
point $\vb r$ about the $i$th coordinate axis with origin $\vb r_0.$

The volume integral of the first+third terms in (\ref{dTorque}) is
%====================================================================%
\begin{align*}
\mc T_i^{(1+3)} 
 &= -\frac{1}{2\omega}\text{Im }\int \epsilon_{ijk} (\vb r-\vb r_0)_j
       \partial_\ell \big(J_\ell^* E_k\big) dV
\\
 &= -\frac{1}{2\omega}\text{Im }\underbrace{
      \oint \epsilon_{ijk} (\vb r-\vb r_0)_j
       E_k \, \vb J\cdot d\vb A}_{=0}
    \,+\,\frac{1}{2\omega}\text{Im }\int \epsilon_{ijk} \delta_{\ell j}
       J_\ell^* E_k dV
\\
 &= +\frac{1}{2\omega}\text{Im }
      \int (\vb J^* \times \vb E)_i \, dV
\end{align*}
%====================================================================%
where the surface integral in the second line 
vanishes by the argument of (\ref{Argument}).

Adding the two contributions, I find
%====================================================================%
\numeq{TorqueVolumeIntegral}
{  \mc T_i
 = \frac{1}{2\omega}\text{Im } \int_{\mc B_n}
   \Big[ \vb J^* \cdot \partial_{\theta_i} \vb E
         \, + \, 
         \big(\vb J^* \times \vb E\big)_i
   \Big]\,dV.
}

%%%%%%%%%%%%%%%%%%%%%%%%%%%%%%%%%%%%%%%%%%%%%%%%%%%%%%%%%%%%%%%%%%%%%% 
%%%%%%%%%%%%%%%%%%%%%%%%%%%%%%%%%%%%%%%%%%%%%%%%%%%%%%%%%%%%%%%%%%%%%%
%%%%%%%%%%%%%%%%%%%%%%%%%%%%%%%%%%%%%%%%%%%%%%%%%%%%%%%%%%%%%%%%%%%%%%
\newpage
\section{From Rytov to Johnson-Nyquist}
\label{RytovToJohnson}

For those of us who learned about noise in resistors
before learning about Casimir forces and radiative heat transfer
in nanoparticles,
it's useful to relate the abstract and possibly mysterious 
notion of the Rytov correlation function to the concrete and 
familiar concept of Johnson-Nyquist noise.
(Even for those who need no help with fluctuation-dissipation 
ideas, this exercise is useful for pinning down factors of $2\pi$ and
other normalization effluvia.)

\subsection*{Johnson-Nyquist Noise}

In elementary circuit theory we are taught that, at temperature
$T$, a resistor exhibits a mean-square power of 
%====================================================================%
$$ \big\langle P \big\rangle = 4 kT \Delta f$$
%====================================================================%
(where $\Delta f$ is the effective measurement bandwidth in Hertz,
usually determined by low-pass and high-pass filters in the circuit).
If the resistance of the resistor is $R$, then the mean-square voltage 
across its terminals and the mean-square current flowing through it are
%====================================================================%
\begin{align}
 \big \langle V^2 \big\rangle 
 &= \big\langle P \big \rangle R = 4k T R\, \Delta f, \qquad 
\nn
 \big \langle I^2 \big\rangle 
 &= \frac{1}{R} \big\langle P \big \rangle 
  = \frac{4k T}{R} \Delta f
\label{MeanSquareCurrent}
\end{align}
%====================================================================%
We would now like to understand equation (\ref{MeanSquareCurrent})
on the basis of the Rytov correlation function.

%=================================================
%=================================================
%=================================================
\subsection*{Macroscopic current noise from microscopic current-density 
             fluctuations}

To this end, consider a resistor consisting of a homogeneous cylinder of
length $L$ and cross-sectional area $A$ with relative dielectric
function
%====================================================================%
\numeq{EpsilonSigma}
{
 \epsilon(\omega)
 \quad=\quad
 \epsilon^\prime(\omega) + i\epsilon^{\prime\prime}(\omega)
 \quad=\quad
 \epsilon^\prime(\omega) + i\frac{\sigma}{\epsilon_0 \omega}
}
%====================================================================%
where $\sigma$ is the microscopic conductivity in units of
mho$\cdot$meters [one mho = 1 inverse ohm  (1 $\Omega^{-1}$) = 
1 siemen].\footnote{The \textit{absolute} permittivity
of the object is $\epsilon_0\epsilon^\prime + i\frac{\sigma}{\omega}$. 
To check that the imaginary part of (\ref{EpsilonSigma}) is indeed
dimensionless, note that $\epsilon_0=\frac{1}{Z_0 c}$ where
$Z_0\approx 377\, \Omega$ is the impedance of free space
and $c$ is the speed of light; thus the units of the imaginary
part of (\ref{EpsilonSigma}) are 
$$ \Big[\frac{\sigma}{\epsilon_0 \omega}\Big] 
   =\frac{[\text{mhos$\cdot$meters}]}
         {[\text{mhos$\cdot$seconds$\cdot$meters}][\text{seconds}^{-1}]}
   =\text{dimensionless.}
$$}
The microscopic resistivity, with units of ohms/meter, is
$\rho=1/\sigma$. The total resistance of the resistor is
%====================================================================%
$$ R=\frac{L}{A}\rho = \frac{L}{A\sigma}.$$
%====================================================================%
We align the cylinder axis with the $z$-axis and break up 
coordinates into transverse and longitudinal components according to
$\vb x=(\vbrho, z)$.

The instantaneous current density at a point in the cylinder
may be expressed as a Fourier synthesis:
%====================================================================%
\begin{align}
 \vb J(\vb x, t) 
&= \int \vb J(\vb x,\omega) e^{-i\omega t} \, d\omega
\label{FourierSynthesis}
\intertext{where $\vb J(\vb x,\omega)$ is the Fourier transform of 
           the instantaneous current density $\vb J(\vb x,t)$:}
 \vb J(\vb x, \omega) 
&= \frac{1}{2\pi} \int \vb J(\vb x,t) e^{+i\omega t} \, dt.
\label{FourierAnalysis}
\end{align}
%====================================================================%
The instantaneous \textit{current} passing through a cross-sectional
plane at height $z$ is 
%====================================================================%
$$ I(z,t)=\int I(z,\omega) e^{-i\omega t} \, d\omega$$
%====================================================================%
where $I(z,\omega)$ is given by integrating $\vb J$ over the cross-sectional
plane:
%====================================================================%
\begin{align*}
 I(z,\omega)&=\int_A \vb J(\vbrho, z,\omega)\cdot \vbhat{n} \, d\vbrho
\intertext{or, in our specific geometry in which the cross section is
everywhere normal to the $z$ direction,}
 I(z,\omega)&=\int_A J_z(\vbrho, z,\omega) \, d\vbrho.
\end{align*}
%====================================================================%
Over a time interval of length $\tau$, the average of the product
of $I(t,z)$ and $I(t,z^\prime)$ is
%====================================================================%
\begin{align} 
\big\langle I(z)I(z^\prime)\big\rangle
&=\frac{1}{\tau}\int_0^\tau I(z,t) I(z^\prime, t) \, dt
\nn
&=\frac{1}{\tau}\int_0^\tau \, dt \,
  \int \, d\omega \, 
  \int \, d\omega^\prime \, 
  I(z,\omega) I(z,\omega^\prime)
  e^{-i(\omega + \omega^\prime)t}
\nn
&=\frac{1}{\tau}\int_0^\tau \, dt \,
  \int \, d\omega \,
  \int \, d\omega^\prime \,
  \int_A \, d\vbrho \,
  \int_A \, d\vbrho^\prime \,
  J_z(\vbrho,z,\omega) J_z(\vbrho^\prime,z^\prime, \omega^\prime)
  e^{-i(\omega + \omega^\prime)t}.
\label{IZZP}
\end{align}
%====================================================================%
To proceed we need to address a question about which we haven't said
anything thus far---namely, where the current \textit{comes from}. 
In a deterministic circuit problem
we would have some fixed, known, externally applied voltage $V$
across the resistor, which would induce a current equal to $I=V/R$.
Here, on the other hand, there is no external voltage, and instead
the current arises from thermal and quantum-mechanical 
\textit{fluctuations} in the microscopic current density.
Although we can't track the instantaneous progress of these
fluctuations in time---and thus, for example, we can't write
down an expression for the instantaneous current density 
$\vb J(\vb x, t)$---we can make precise statements about
certain statistical \textit{averages} over these fluctuating
quantities. One particularly obvious statement is that the 
time-average value of any Cartesian component of $\vb J$ vanishes, 
%====================================================================%
$$\Big\langle J_i(\vb x,t)\Big\rangle=0 \qquad \text{for any } i.$$
%====================================================================%
A less obvious but even more important statement is that
the fluctuation-dissipation theorem allows us to make a
very definite prediction about the time-average value of the
\textit{product} of two cartesian components of $\vb J$. 
This equation---sometimes known as the \textit{Rytov correlation
function}---is easiest to write in the frequency domain, where
it takes the form
%====================================================================%
\begin{align}
 \Big\langle  J_i(\vb x, \omega) J_j(\vb x^\prime, \omega^\prime) \Big\rangle
&= \delta(\omega+\omega^\prime)
   \Big\langle J_i(\vb x), J_j(\vb x^\prime) \Big\rangle_\omega
\label{Rytov1}
\\[5pt]
 \Big\langle  J_i(\vb x) J_k(\vb x^\prime) \Big\rangle_{\omega}
&= \frac{2 \omega \epsilon_0}{\pi} \Theta(\omega, T) \delta(\vb r-\vb r^\prime)
   \text{Im }\epsilon_{ij}(\vb r, \omega)
\label{Rytov2}
\end{align}
%====================================================================%
Equation (\ref{Rytov1}) here is essentially the same\footnote{
My formula differs from that of Landau and Lifshitz (LL) by a 
factor of $2\pi$, which arises because of our different conventions
for the Fourier analysis and synthesis of time-domain functions:
I like to put a factor of $\frac{1}{2\pi}$ in Fourier-analysis 
equations like (\ref{FourierAnalysis}), and to omit this factor 
in Fourier-synthesis equations like (\ref{FourierSynthesis}), while
LL make the opposite choice.} as equation 
(122.4) in Landau and Lifshitz (LL), \textit{Statistical Physics Volume 1}.
In equation (\ref{Rytov2}), $\Theta(\omega, T)$ is the Bose-Einstein 
statistical factor at the local temperature near $\vb r$ 
(about which we will have more to say shortly)
and $\epsilon_{ij}(\vb r, \omega)$ is the
$(i,j)$ component of the relative permittivity tensor of the material
at point $\vb r$ and frequency $\omega$.
Inserting (\ref{Rytov1}) into (\ref{IZZP}) and using
the $\delta$ functions to perform the $\vbrho^\prime$ and $\omega^\prime$
integrations, we have 
%%%%%%%%%%%%%%%%%%%%%%%%%%%%%%%%%%%%%%%%%%%%%%%%%%%%%%%%%%%%%%%%%%%%%%
\begin{align*}
&\hspace{-0.1in}
\big\langle I(z)I(z^\prime)\big\rangle\\
&=\frac{1}{\tau}\int_0^\tau \, dt \,
  \int \, d\omega \,
  \int \, d\omega^\prime \,
  \int_A \, d\vbrho \,
  \int_A \, d\vbrho^\prime \,
  \Big\langle
  J_z(\vbrho,z,\omega) J_z(\vbrho^\prime,z^\prime, \omega^\prime)
  \Big\rangle
  e^{-i(\omega + \omega^\prime)t}
\\
&=\frac{2\epsilon_0 }{\pi \tau}\delta(z-z^\prime)
  \int_0^\tau \, dt \,
  \int \, d\omega \, \omega \, \Theta(\omega, T)
  \int_A \, d\vbrho \,
  \text{Im }\epsilon_{zz}(\vbrho, z, \omega)
\intertext{
In the present case [cf. equation (\ref{EpsilonSigma})] we have
$\text{Im }\epsilon_{ij}(\vb r,\omega)=\delta_{ij} \frac{\sigma}{\epsilon_0 \omega}$
(independent of $\vb r$), whereupon we find}
&=\frac{2\sigma }{\pi}\cdot \delta(z-z^\prime) 
  \cdot \underbrace{\frac{1}{\tau} \int_0^\tau \, dt \, \int\, d\omega \,
                    \Theta(\omega, T) 
                   }_{kT \Delta \omega}
  \underbrace{\int_A \, d\vbrho \,}_{A}
\end{align*}
%%%%%%%%%%%%%%%%%%%%%%%%%%%%%%%%%%%%%%%%%%%%%%%%%%%%%%%%%%%%%%%%%%%%%%
Here I used the high-temperature limit $\Theta(\omega, T)\approx kT$ 
(see below), Finally, averaging over the length of the resistor yields
\begin{align*}
 \big\langle I^2(z) \big\rangle
&= \frac{1}{L} \int_0^L \big\langle I(z)I(z^\prime)\big\rangle \, dz^\prime 
\\
&=\frac{2kT\Delta \omega}{\pi} \cdot \underbrace{\frac{A\sigma}{L}}_{1/R}
\\
&= \frac{4kT}{R}\Delta f
\end{align*}
where I used $\Delta \omega=2\pi\Delta f.$ This is equation 
(\ref{MeanSquareCurrent}).
%%%%%%%%%%%%%%%%%%%%%%%%%%%%%%%%%%%%%%%%%%%%%%%%%%%%%%%%%%%%%%%%%%%%%%

%=================================================
%=================================================
%=================================================
\subsection*{Limiting behavior of $\Theta(\omega, T)$}

The Bose-Einstein statistical factor $\Theta(\omega , T)$, 
which describes the average energy contained in an 
electromagnetic\footnote{Or otherwise bosonic.} 
mode of frequency $\omega$, is 
%%%%%%%%%%%%%%%%%%%%%%%%%%%%%%%%%%%%%%%%%%%%%%%%%%%%%%%%%%%%%%%%%%%%%%
\numeq{ThetaDef}
{
 \Theta(\omega, t)
 =\hbar \omega
        \left[ \frac{1}{e^{\frac{\hbar\omega}{kT}} - 1} + \frac{1}{2}\right]
=\frac{\hbar \omega}{2} \coth\pf{\hbar \omega}{2k T}
}
%%%%%%%%%%%%%%%%%%%%%%%%%%%%%%%%%%%%%%%%%%%%%%%%%%%%%%%%%%%%%%%%%%%%%%
In the high- and low-temperature limits (equivalently, the low- and 
high-frequency) limits, the statistical factor of equation (\ref{ThetaDef}) 
becomes
%%%%%%%%%%%%%%%%%%%%%%%%%%%%%%%%%%%%%%%%%%%%%%%%%%%%%%%%%%%%%%%%%%%%%%
\numeq{ThetaLimits}
{
  \Theta(\omega, t)\xrightarrow{\frac{\hbar\omega}{kT}\to 0} kT, 
  \qquad
  \Theta(\omega, t)\xrightarrow{\frac{\hbar\omega}{kT}\to \infty} \frac{\hbar \omega}{2}.
}
%%%%%%%%%%%%%%%%%%%%%%%%%%%%%%%%%%%%%%%%%%%%%%%%%%%%%%%%%%%%%%%%%%%%%%
The first case here corresponds to classical equipartition of energy: 
we have roughly $kT$ worth of energy in each mode, independent of 
frequency. The second case corresponds to quantum-mechanical zero-point
energy; we have some energy in each mode even at zero temperature.

To estimate the crossover between the high- and low-temperature
regimes, recall that room temperature ($T$=300 K) corresponds to 
an energy of $kT \approx $ 26 meV (milli-electron-volts), while 
$\hbar$ has the numerical value
%%%%%%%%%%%%%%%%%%%%%%%%%%%%%%%%%%%%%%%%%%%%%%%%%%%%%%%%%%%%%%%%%%%%%%
$$ \hbar 
   \approx 7\cdot 10^{-16} \text{eV}\cdot \text{s}
  = 7\cdot{10^{-16}} \frac{\text{eV}}{\text{rad/s}}.
$$
Thus, for a circuit at frequency $f=1$ GHz at a temperature of $T$=300 K,
we have 
%%%%%%%%%%%%%%%%%%%%%%%%%%%%%%%%%%%%%%%%%%%%%%%%%%%%%%%%%%%%%%%%%%%%%%
$$ kT = 0.026 \text{ eV } \quad \ll \quad \hbar\omega 
      = \left( 7\cdot 10^{-16} \frac{\text{eV}}{\text{rad/s}} \right)
        \left( 2\pi \cdot 10^9 \text{rad/s} \right)
      = 4 \cdot 10^{-6}\text{ eV}
$$
and thus, for ordinary circuits at ordinary temperatures, we are 
well in the high-temperature (low-frequency) regime in which
$\Theta(\omega,T)\approx kT$.

%%%%%%%%%%%%%%%%%%%%%%%%%%%%%%%%%%%%%%%%%%%%%%%%%%%%%%%%%%%%%%%%%%%%%%
%%%%%%%%%%%%%%%%%%%%%%%%%%%%%%%%%%%%%%%%%%%%%%%%%%%%%%%%%%%%%%%%%%%%%%
%%%%%%%%%%%%%%%%%%%%%%%%%%%%%%%%%%%%%%%%%%%%%%%%%%%%%%%%%%%%%%%%%%%%%%
\newpage
\section{Volume integrals involving SWG basis functions}
\label{VolumeIntegralAppendix}

Integrals over the support of SWG basis functions take the form
\numeq{SWGVolumeIntegral}
{
 \int_{\sup \vb b_\alpha} \mathcal{I}(\vb x, \vb b_\alpha) \, d\vb x
= 
  \int_{\mc P_\alpha^+} \mathcal{I}(\vb x, \vb b_\alpha) \, d\vb x
 +\int_{\mc P_\alpha^-} \mathcal{I}(\vb x, \vb b_\alpha) \, d\vb x
}
%%%%%%%%%%%%%%%%%%%%%%%%%%%%%%%%%%%%%%%%%%%%%%%%%%%%%%%%%%%%%%%%%%%%%%
Consider a tetrahedron with vertices 
$\{\vb Q, \vb V_1, \vb V_2, \vb V_3.\}$ 
A general
integral over this region takes the form 
%====================================================================%
\begin{align*}
 \int_{\mc P} \mathcal{I}(\vb x, \vb b ) \, d\vb x
&=\mc{J} \int_0^1 \, du \int_0^{1-u} \, dv \int_0^{1-u-v} dw \,
  \mathcal{I}\Big( \vb x(u,v,w), \vb b(u,v,w) \Big)
\end{align*}
%====================================================================%
%====================================================================%
\begin{align*}
 \vb x(u,v,w) 
&= 
 \vb Q + u\vb L_1 + v\vb L_2 + w\vb L_3 
\\
 \vb b(u,v,w) 
&= 
 \pm \frac{A}{3V}\bigg\{ u \vb L_1 
                        +v \vb L_2 
                        +w \vb L_3 
                 \bigg\}
\end{align*}
%====================================================================%
where
%====================================================================%
$$ \vb L_i \equiv \vb V_i - \vb Q_i $$
%====================================================================%
and the Jacobian of the transformation is 
%====================================================================%
$$\mc J=\frac{d(x,y,z)}{d(u,v,w)}
   =\det\left|\begin{array}{ccc} 
     \\
     \vb L_1 & \vb L_2 & \vb L_3 \\
     \\
     \end{array}\right|
  =6V
$$

%=================================================
%=================================================
%=================================================
\subsection*{Overlap Matrix Elements} 

\begin{align*} 
  \inpB{\vb b_\alpha}{\vb b_\beta}
&=\sum \pm \frac{A_\alpha A_\beta}{9V^2}
  \int_{\mc V} 
  \Big(\vb x-\vb Q_\alpha\Big)
  \cdot
  \Big(\vb x-\vb Q_\beta\Big)
  d\vb x
\intertext{(where the sum is over the 0, 1, or 2 tetrahedra in the 
            common support of $\{\vb b_\alpha,\vb b_\beta\}$)}
&=\sum \pm \frac{2A_\alpha A_\beta}{3V}
  \int_0^1 \, du \int_0^{1-u} \, dv \, \int_0^{1-u-v}\, dw\,
  \bigg\{ 
  \Big[u\vb L_{1\alpha} + v\vb L_{2\alpha} + w\vb L_{3\alpha}\Big]
\\
&\hspace{2in}
  \cdot \Big[u\vb L_{1\alpha} + v\vb L_{2\alpha} + w\vb L_{3\alpha} 
            + \vb Q_\alpha - \vb Q_\beta\Big]
  \bigg\}
\\
&=\sum \pm \frac{2A_\alpha A_\beta}{3V}
  \bigg\{ \frac{1}{120}
          \Big|\vb L_{1\alpha} + \vb L_{2\alpha} + \vb L_{3\alpha}
          \Big|^2
         +\frac{1}{120}\Big( |\vb L_{1\alpha}|^2 
                            +|\vb L_{2\alpha}|^2 
                            +|\vb L_{3\alpha}|^2 \Big)
\\
&\hspace{2in}
         +\frac{1}{24}
         \Big(\vb L_{1\alpha} + \vb L_{2\alpha} + \vb L_{3\alpha}\Big)
          \cdot (\vb Q_\alpha - \vb Q_\beta)
  \bigg\}
\intertext{This can be simplified by noting that 
           $\vb L_{1\alpha} + \vb L_{2\alpha} + \vb L_{3\alpha}
           =3(\vb X_{0\alpha} - \vb Q_\alpha)$
           where $\vb X_{0\alpha}$ is the centroid of 
           basis function $\vb b_\alpha$:}
&=\sum \pm \frac{2A_\alpha A_\beta}{3V}
  \Big[ \frac{3}{40}(\vb X_{0\alpha}-\vb Q_\alpha)^2 
        +\frac{1}{120}\Big( |\vb L_{1\alpha}|^2 
                           +|\vb L_{2\alpha}|^2 
                           +|\vb L_{3\alpha}|^2 \Big)
\\
&\hspace{1.5in}
         +\frac{1}{8}(\vb X_{0\alpha} - \vb Q_\alpha) 
               \cdot (\vb Q_\alpha-\vb Q_\beta)
  \Big].
\end{align*}

%=================================================
%=================================================
%=================================================
\subsection*{Dipole and Quadrupole Moments}

The dipole moment of the current distribution described by
a single unit-strength SWG basis function is 

\numeq{pAlpha}
{
   \vb p_\alpha 
   \equiv 
   \frac{i}{\omega} 
   \underbrace{\int_{\sup \vb b_\alpha} \vb b_\alpha(\vb x) \, d\vb x}
             _{\bmc J_\alpha(\vb x)}
}
%====================================================================%
The Cartesian components of $\bmc J_\alpha(\vb x)$ may be worked out 
in closed form:
%====================================================================%
\begin{align}
 \mc J_{\alpha i}(\vb x)
&=2A\int_0^1 \, du \, \int_0^{1-u} \, dv \, \int_0^{1-u-v}  \, dw \,
    (u+v+w) (\vb Q^- - \vb Q^+)_i
\nn
&=\frac{A}{4} (\vb Q^- - \vb Q^+)_i.
\label{mcJAlpha}
\end{align}
%====================================================================%
Similarly, the quadrupole moments are related to the quantity
\begin{align*}
 \mc Q_{\alpha ij}(\vb x)
&= \int_{\sup \vb b_\alpha} b_{\alpha i} (\vb x-\vb x_0)_j\, d\vb x
\intertext{where $\vb x_0=\frac{1}{3}(\vb V_1 + \vb V_2 + \vb V_3)$
           is the centroid of the basis function (which we take to 
           be the centroid of the triangle that constitutes the 
           common face).}
&=
2A\int_0^1 \, du \, \int_0^{1-u} \, dv \, \int_0^{1-u-v}  \, dw \,
 \bigg\{
\\
&\qquad\qquad 
 \hphantom{-}\,
  \bigg[ u\vb L_1^+ + v\vb L_2^+ + w\vb L_3+ \bigg]_i
  \left[ \left(u-\frac{1}{3}\right)\vb L_1^+
        +\left(v-\frac{1}{3}\right)\vb L_1^+
        +\left(w-\frac{1}{3}\right)\vb L_1^+
  \right]_j
\\
&\qquad\qquad 
 -\bigg[ u\vb L_1^- + v\vb L_2^- + w\vb L_3- \bigg]_i
  \left[ \left(u-\frac{1}{3}\right)\vb L_1^-
        +\left(v-\frac{1}{3}\right)\vb L_2^-
        +\left(w-\frac{1}{3}\right)\vb L_3^-
  \right]_j
 \bigg\}
\\
&=\frac{A}{20}
  \bigg\{ 
          Q^-_i \Big[ \vb Q^- - \vb x_0 \Big]_j
         -Q^+_i \Big[ \vb Q^+ - \vb x_0 \Big]_j
         +x_{0i}\Big[ \vb Q^+ - \vb x_0 \Big]_j
  \bigg\}
\end{align*}

%====================================================================%
%====================================================================%
%====================================================================%
\newpage
\bibliographystyle{ieeetr}
\bibliography{buff-em}

\end{document}
