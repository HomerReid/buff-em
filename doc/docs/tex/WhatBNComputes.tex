
\documentclass[letterpaper]{article}

\usepackage[english]{babel}
\usepackage{graphicx}
\usepackage{color}
\usepackage{dsfont}
\usepackage{bbm}
\usepackage{amsmath}
\usepackage{amssymb}
\usepackage{float}
\usepackage{psfrag}
\usepackage{mathdots}
\usepackage{algorithm}
\usepackage{algorithmic}
\usepackage{listings}
\usepackage{fancybox}
\usepackage{fancyvrb}
\usepackage{arydshln}
\usepackage{verbatim}

%--------------------------------------------------
%- primed variables 
%--------------------------------------------------
\newcommand\kp{k^\prime}
\newcommand\rp{r^\prime}
\newcommand\kk{k^\prime}

%--------------------------------------------------
%- boldface greek letters 
%--------------------------------------------------
\newcommand\vbphi{\mathbf{\phi}}
\newcommand\vbPhi{\mathbf{\Phi}}
\newcommand{\vbDelta}{\boldsymbol{\Delta}}
\newcommand{\vbLambda}{\boldsymbol{\Lambda}}
\newcommand{\vbGamma}{\boldsymbol{\Gamma}}
\newcommand\vbtheta{\vec{\theta}}
\newcommand\vbbeta{\vec{\beta}}
\newcommand\vbsigma{\vec{\sigma}}
\newcommand{\vbrho}{\boldsymbol{\rho}}
\newcommand\bsigma{\overline{\sigma}}

%--------------------------------------------------
%- field theory stuff 
%--------------------------------------------------
\newcommand\ub{\overline{u}}
\newcommand\ubar{\overline{u}}
\newcommand\vbar{\overline{v}}
\newcommand\psibar{\overline{\psi}}
\newcommand\ps{\FMSlash{p}}
\newcommand\ks{\FMSlash{k}}
\newcommand\qs{\FMSlash{q}}
\newcommand\ls{\FMSlash{l}}
\newcommand\ds{\FMSlash{\partial}}
\newcommand\pps{\FMSlash{p}^\prime}
\newcommand\kps{\FMSlash{k}^\prime}
\newcommand\MM{\mathcal{M}}
\newcommand\btu{\bigtriangleup}

%--------------------------------------------------
%- colors -----------------------------------------
%--------------------------------------------------
\newcommand{\red}[1]{\textcolor{red}{#1}}
\newcommand{\blue}[1]{\textcolor{blue}{#1}}
\newcommand{\green}[1]{\textcolor{green}{#1}}
\newcommand{\atan}{\text{atan}}
\definecolor{cyan}{rgb}{0.0,1.00,1.00}
%\newcommand{\cyan}[1]{\textcolor{cyan}{#1}}
\definecolor{lightblue}{rgb}{0.0,0.00,0.75}
\newcommand{\lightblue}[1]{\textcolor{lightblue}{#1}}

%--------------------------------------------------
% general commands
%--------------------------------------------------
\newcommand{\ccdot}{\, \cdot \,}
\newcommand{\eps}{\epsilon}
\newcommand{\ez}{\epsilon_0}
\newcommand{\unit}[1]{\, \hbox{#1} \, }
\newcommand{\qtext}[1]{\quad \hbox{#1} \quad }
\newcommand{\curl}[0]{\nabla \times \vb}
\newcommand{\divv}[0]{\nabla \cdot \vb}
\newcommand{\uv}[1]{\vb{\hat{#1}}}
\newcommand{\lab}{<\!} 
\newcommand{\rab}{\!>} 
\newcommand{\bra}[1]{\left<#1\right|}
\newcommand{\ket}[1]{\left|#1\right>}
\newcommand{\ketb}[1]{\big|#1\big>}
\newcommand{\ketB}[1]{\Big|#1\Big>}
\newcommand{\inp}[2]{\left<#1\right|\left.#2\right>}
\newcommand{\Inp}[2]{\big<#1\big|\big.#2\big>}
\newcommand{\INP}[2]{\Big<#1\Big|\Big.#2\Big>}
\newcommand{\Inpb}[2]{\big<#1\big|\big.#2\big>}
\newcommand{\inpB}[2]{\Big<#1\Big|\Big.#2\Big>}
\newcommand{\inpv}[2]{\left<\vp #1\right|\left.#2\right>}
\newcommand{\expval}[1]{\left< #1 \right>}
\newcommand{\expvalv}[1]{\left<\vp #1 \right>}
\newcommand{\exptwo}[3]{\left<#1\right|#2\left|#3\right>} 
\newcommand{\ExpTwo}[3]{\big<#1\big|#2\big|#3\big>} 
\newcommand{\EXPTWO}[3]{\Big<#1\Big|#2\Big|#3\Big>} 
\newcommand{\exptwob}[3]{\big<#1\big|#2\big|#3\big>} 
\newcommand{\exptwoB}[3]{\Big<#1\Big|#2\Big|#3\Big>} 
\newcommand{\exptwov}[3]{\left<\vp#1\right|#2\left|\vp#3\right>} 
\newcommand{\exptwoi}[3]{\left<\vi#1\right|#2\left|\vi#3\right>} 
\newcommand{\union}{\cup}
\newcommand\unitmatrix{\mathds{1}}
\newcommand\Tr{\hbox{Tr }}
\newcommand\tbtm[4]{\left(\begin{array}{ll}#1 & #2 \\ #3 & #4\end{array}\right)}
\newcommand\sups[1]{^{\hbox{\scriptsize{#1}}}}
\newcommand\supt[1]{^{\hbox{\tiny{#1}}}}
\newcommand\subs[1]{_{\hbox{\scriptsize{#1}}}}
\newcommand\subt[1]{_{\hbox{\tiny{#1}}}}
\newcommand{\nn}{\nonumber \\}
\newcommand{\vb}[1]{\mathbf{#1}}
\newcommand{\eq}[1]{\begin{equation} #1 \end{equation}}
\newcommand{\numeq}[2]{\begin{equation} #2 \label{#1} \end{equation}}
\newcommand{\BE}{\begin{equation}}
\newcommand{\EE}{\end{equation}}
\newcommand{\pard}[2]{\frac{\partial #1}{\partial #2}}
\newcommand{\pardn}[3]{\frac{\partial^{#1} #2}{\partial #3^{#1}}}
\newcommand{\pf}[2]{\left(\frac{#1}{#2}\right)}
\newcommand{\vp}{\vphantom\sum}
\newcommand{\vi}{\vphantom{\sum_{-\infty^\infty}}}
\newcommand{\evalat}[2]{\left. #1 \right|_{#2}}
\newcommand{\evalatl}[2]{\left| #1 \right|_{#2}}
\newcommand{\pp}{{\prime\prime}}
\newcommand{\vbhat}[1]{\vb{\hat #1}}
\newcommand{\vbhatt}[1]{\boldsymbol{\widehat #1}}
\newcommand{\mc}[1]{\mathcal{#1}}
\newcommand{\bmc}[1]{\boldsymbol{\mathcal{#1}}}
\newcommand{\mb}[1]{\mathbb{#1}}
\newcommand{\primedsum}{\sideset{}{'}{\sum}}

\graphicspath{{figures/}}
\definecolor{shadecolor}{rgb}{0.85,0.85,0.85}

%--------------------------------------------------
%-- \text with a built-in font size specifier   
%--------------------------------------------------
\newcommand\ts[1]{\text{\scriptsize{#1}}}
\newcommand\ty[1]{\text{\tiny{#1}}}

\newcommand{\prob}[3]{\section*{Problem #1}} 

%--------------------------------------------------
%- shaded verbatim for code inclusions
%--------------------------------------------------
\definecolor{lightgrey}{rgb}{0.75,0.75,0.75}
\newenvironment{verbcode}{\VerbatimEnvironment% 
  \noindent
  %      {\columnwidth-\leftmargin-\rightmargin-2\fboxsep-2\fboxrule-4pt} 
  \begin{Sbox} 
  \begin{minipage}{\linewidth-2\fboxsep-2\fboxrule-4pt}    
  \begin{Verbatim}
}{% 
  \end{Verbatim}  
  \end{minipage}   
  \end{Sbox} 
  \fcolorbox{black}{lightgrey}{\textcolor{black}{\TheSbox}}
} 


\usepackage{cite}

\graphicspath{{figures/}}
\newcommand{\TInv}{\rotatebox[origin=c]{180}{$T$}}
\newcommand{\vbTInv}{\rotatebox[origin=c]{180}{$\vb T$}}
\newcommand{\vbTInvSmall}{\rotatebox[origin=c]{180}{\scriptsize{$\vb T$}}}
\newcommand{\bbTInv}{\rotatebox[origin=c]{180}{$\mathbb{T}$}}
\newcommand{\bbVInv}{\rotatebox[origin=c]{180}{$\mathbb{V}$}}
\newcommand{\vbSigma}{\boldsymbol{\Sigma}}
\newcommand{\fd}{^{\text{\tiny{F}}\dagger}}

% 'IEM' stands for 'imaginary part of epsilon matrix'
\newcommand{\vbIEM}{\boldsymbol{\Sigma}}
\newcommand{\IEM}{\Sigma}

\newcommand{\citeasnoun}[1]{Ref.~\citen{#1}}
\newcommand{\citeasnouns}[1]{Refs.~\citen{#1}}

%------------------------------------------------------------
%------------------------------------------------------------
%- Special commands for this document -----------------------
%------------------------------------------------------------
%------------------------------------------------------------
\newcommand{\vbxi}{\boldsymbol{\xi}}
\newcommand{\im}{\text{Im }}
\newcommand{\vbeps}{\boldsymbol{\epsilon}}

%------------------------------------------------------------
%------------------------------------------------------------
%- Start of actual document
%------------------------------------------------------------
%------------------------------------------------------------

\begin{document}

In the FVC approach, the thermal average of 
a power, force, or torque quantity $Q$ is 
computed by integrating a spectral density over all 
frequencies:
%====================================================================%
\begin{align*}
 Q &= \int_0^\infty \big\langle Q\big\rangle_\omega \, d\omega
\\
 \big\langle Q \big \rangle_\omega
   &=\text{Tr }\Big[\vb Q\sups{PFT} \vb W \vb R \vb W^\dagger\Big]
\end{align*}
%====================================================================%
where
\begin{itemize}
  \item the $\vb Q\supt{PFT}$ matrix is the matrix one sandwiches
        between the vectors of volume-current coefficients to 
        obtain the power, force, or torque
  \item $\vb R$ is a matrix describing the Rytov source density, and
  \item $\vb W$ is a matrix describing the ``dressing'' of the 
        Rytov density by the polarization response of the material
        geometry.
\end{itemize}
The elements of the $\vb R$ matrix are
%====================================================================%
$$ R_{\alpha\beta}=\frac{2k}{\pi Z_0}
   \EXPTWO{\vb b_\alpha}{ \Delta\Theta(\vb x) (\vbeps(\vb x) - \vb 1)}{\vb b_\beta}
$$
%====================================================================%
where $\Delta\Theta(\vb x)$ is the difference between the Bose-Einstein
factor at the local temperature at $\vb x$ and the Bose-Einstein
factor at the temperature of the environment:
%====================================================================%
$$\Delta\Theta(\vb x) = \Theta\Big( T(\vb x), \omega \Big) 
                       -\Theta\Big( T\sups{env}, \omega \Big).
$$
%====================================================================%
It is now convenient to split the $\vb R$ matrix into separate
contributions from the various objects in the geometry:
%====================================================================%
$$ \vb R = \vb R_1 + \vb R_2 + \cdots + \vb R_{N} $$
%====================================================================%
and to extract from the $n$th term a scalar
prefactor $\wh{\Delta\Theta}_n$ that involves the 
volume average of the temperature in body $n$:
%====================================================================%
\begin{align*}
 \wh{\Delta\Theta}_n 
&\equiv \Theta\Big( \wh{T}_n, \omega \Big) - 
        \Theta\Big( T\sups{env}, \omega \Big)
\\
\wh{T}_n 
&\equiv 
   \frac{1}{\mc V_n} \int_{\mc V_n} T(\vb x) \, d\vb x.
\end{align*}
%====================================================================%
The $\vb R$ matrix then reads
%====================================================================%
$$ \vb R = 
     \wh{\Delta\Theta}_1 \, \wh{\vb R}_1
    +\wh{\Delta\Theta}_2 \, \wh{\vb R}_2
    +\cdots
    +\wh{\Delta\Theta}_N \, \wh{\vb R}_N
$$
%====================================================================%
where the elements of the $\wh{R}_n$ matrices are just the 
$\vb R$-matrix elements normalized by $\wh{\Delta\Theta}_n$:
%====================================================================%
\numeq{NormalizedRMatrix}
{
\wh R_{n;\alpha\beta}=
   \frac{2k}{\pi Z_0 \wh{\Delta\Theta}}
   \EXPTWO{\vb b_\alpha}
          { \Delta\Theta(\vb x) (\vbeps(\vb x) - \vb 1)}{\vb b_\beta}.
}
%====================================================================%
This allows us to resolve the quantity $Q$ into contributions
from thermal sources in each object:
%====================================================================%
\begin{align*}
 Q&=\sum Q_n
\\
\int_0^\infty \sum_n \Delta\Theta(T_n,\omega)\Phi_n(\omega) \,d\omega
\label{Qn}
\Phi_n(\omega)
  &=\text{Tr }\Big[\vb Q\sups{PFT} \vb W \wh{\vb R}_n \vb W^\dagger\Big].
\end{align*}
%====================================================================%
For bodies of uniform temperature, the $\Phi_n$ factor here is just
the same flux quantity computed by {\sc scuff-neq}. 

This normalization scheme does not work for objects whose average
temperature is equal to that of the environment, $\wh T_n=T\sups{env}$.
In this case we remove the factor $\wh{\Delta\Theta}$ from both
the denominator of (\ref{NormalizedRMatrix})
and the numerator of (\ref{Qn}).

\end{document}
