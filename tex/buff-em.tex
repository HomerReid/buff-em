\documentclass[letterpaper]{article}

\input{physcmds}

\graphicspath{{figures/}}
\newcommand{\TInv}{\rotatebox[origin=c]{180}{$T$}}
\newcommand{\vbTInv}{\rotatebox[origin=c]{180}{$\vb T$}}
\newcommand{\vbTInvSmall}{\rotatebox[origin=c]{180}{\scriptsize{$\vb T$}}}
\newcommand{\bbTInv}{\rotatebox[origin=c]{180}{$\mathbb{T}$}}
\newcommand{\bbVInv}{\rotatebox[origin=c]{180}{$\mathbb{V}$}}

%------------------------------------------------------------
%------------------------------------------------------------
%- Special commands for this document -----------------------
%------------------------------------------------------------
%------------------------------------------------------------
\newcommand{\vbxi}{\boldsymbol{\xi}}
\newcommand{\TT}{\mathbb{T}}
\newcommand{\GG}{\mathbb{G}}
\newcommand{\im}{\text{Im }}
\newcommand{\vbeps}{\boldsymbol{\epsilon}}
\newcommand{\vbLambda}{\boldsymbol{\Lambda}}

%------------------------------------------------------------
%------------------------------------------------------------
%- Document header  -----------------------------------------
%------------------------------------------------------------
%------------------------------------------------------------
\title { {\sc buff-em}: A Volume-Integral-Equation
         Solver Suite for Computational Fluctuation Physics}
\author {Homer Reid}
\date {June 3, 2014}

%------------------------------------------------------------
%------------------------------------------------------------
%- Start of actual document
%------------------------------------------------------------
%------------------------------------------------------------

\begin{document}
\pagestyle{myheadings}
\markright{Homer Reid: \texttt{buff-em} }
\maketitle

\tableofcontents

%%%%%%%%%%%%%%%%%%%%%%%%%%%%%%%%%%%%%%%%%%%%%%%%%%%%%%%%%%%%%%%%%%%%%%
%%%%%%%%%%%%%%%%%%%%%%%%%%%%%%%%%%%%%%%%%%%%%%%%%%%%%%%%%%%%%%%%%%%%%%
%%%%%%%%%%%%%%%%%%%%%%%%%%%%%%%%%%%%%%%%%%%%%%%%%%%%%%%%%%%%%%%%%%%%%%
\newpage
\section{SWG Basis Functions}

SWG basis functions are the three-dimensional analog of
RWG basis functions. They are defined on pairs of adjacent tetrahedra:
%====================================================================%
$$ \vb b_\alpha(\vb x) = 
  \begin{cases}
   \displaystyle{
    +\frac{A_\alpha}{3V_\alpha^+}(\vb x - \vb Q_\alpha^+), 
                }
    \qquad &\vb x\in \mc P_\alpha^+
\\[5pt]
   \displaystyle{
  -\frac{A\alpha}{3V_\alpha^-}(\vb x - \vb Q_\alpha^-), 
                } 
    \qquad &\vb x\in \mc P_\alpha^-
  \end{cases}
$$
%====================================================================%
where $\mc P_\alpha^\pm$ are the two tetrahedra associated with basis
function $\alpha$, $V_\alpha^\pm$ are their volumes, $\vb Q_\alpha\pm$
are the source/sink vertices, and $A_\alpha$ is the area of the 
triangular face shared by $\mc P_\alpha^\pm$. (I denote tetrahedra 
by $\mc P$, which stands for ``pyramid,'' to avoid confusion with
the symbol $\mc T$, which stands for ``triangle'' in my memos on
RWG basis functions.)

The divergence of the SWG basis function is 
%====================================================================%
$$ \nabla \cdot \vb b_\alpha(\vb x) = 
    \pm \frac{A_\alpha}{V_\alpha^\pm}, \qquad \vb x\in \mc P_\alpha^\pm.
$$

%%%%%%%%%%%%%%%%%%%%%%%%%%%%%%%%%%%%%%%%%%%%%%%%%%%%%%%%%%%%%%%%%%%%%%
%%%%%%%%%%%%%%%%%%%%%%%%%%%%%%%%%%%%%%%%%%%%%%%%%%%%%%%%%%%%%%%%%%%%%%
%%%%%%%%%%%%%%%%%%%%%%%%%%%%%%%%%%%%%%%%%%%%%%%%%%%%%%%%%%%%%%%%%%%%%%
\newpage
\section{Volume-Integral Scattering Formulation}

\subsection*{Continuous VIE formulation}

Consider a material body with relative permittivity tensor
$\vbeps(\vb x)$ lying in vacuum and irradiated by monochromatic
sources which may lie inside or outside the body. 
Let the electric field due to the sources be $\vb E\sups{inc}$ 
and let $\vb J(\vb x)$ be the induced volume current
density throughout the bulk of the body. 
(We work at frequency $\omega$ and assume time dependence
of all field and currents $\propto e^{-i\omega t}$.)
The total electric field at any point is a sum of
``incident'' and ``scattered'' contributions:\footnote{We
put the terms ``incident'' and ``scattered'' in quotes
to remind readers that the ``incident''-field sources may
in fact lie \textit{inside} the body; in this case it is 
not quite right to refer to the field they produce as 
being ``incident'' on the body, but the terminology is 
convenient nonetheless.}
%====================================================================%
\begin{align}
 \vb E\sups{tot} &= \vb E\sups{inc} + \vb E\sups{scat}
\\
                 &= \vb E\sups{inc} + ikZ_0 \mathbb{G}*\vb J
\label{ETot}
\end{align}
%====================================================================%
where $k=\omega/c$ is the (free-space) wavenumber,
$Z_0=\sqrt{\mu_0/\epsilon_0}$ is the impedance of free space,
$*$ denotes convolution, and $\mathbb{G}$ is the free-space
dyadic Green's function:
$$ \mathbb{G}(\vb x, \vb x^\prime) = 
   \Big(\vb 1 - \frac{1}{k^2} \nabla \times \nabla \Big)
   \frac{e^{ikr}}{4\pi r}
   \qquad (r\equiv |\vb x-\vb x^\prime|).
$$
On the other hand, the induced current is related to the total field
according to
%====================================================================%
\begin{align*}
 \vb J(\vb x) 
&= i\frac{k}{Z_0} \big[\vbeps(\vb x) - \vb 1\big]\cdot \vb E\sups{tot}(\vb x)
\\
&\equiv^* -\frac{1}{ikZ_0} \mathbb{V}(\vb x) \cdot \vb E\sups{tot}(\vb x)
\end{align*}
%====================================================================%
where $\mathbb{V}\equiv k^2\big[\vb 1 - \vbeps(\vb x)\big]$ is
sometimes known\footnote{Jamal's paper.} as the ``potential.''
At points not in free space, i.e. points at which 
$\vbeps(\vb x)\ne \vb 1$, we can invert this equation to read
%====================================================================%
\begin{align}
-ikZ_0 \bbVInv(\vb x) \cdot \vb J(\vb x) &= \vb E\sups{tot}(\vb x)
\nonumber
\intertext{where $\bbVInv\equiv \mathbb{V}^{-1}.$ Now insert (\ref{ETot}):}
-ikZ_0 \bbVInv(\vb x) \cdot \vb J(\vb x) &= \vb E\sups{inc} + ik Z_0 \mathbb{G} \star \vb J \\
\intertext{Rearranging and writing out the convolution, 
           we obtain a volume-integral equation for $\vb J$:} 
  -ikZ_0\left[   \bbVInv(\vb x) \cdot \vb J(\vb x)
              + \int \mathbb{G}(\vb x, \vb x^\prime) 
                \cdot \vb J(\vb x^\prime) \,d\vb x^\prime
        \right]
&= \vb E\sups{inc}(\vb x).
\\
\intertext{In what follows it will be convenient to think of the RHS 
           here as the convolution of just a single operator with $\vb J$:}
 -ikZ_0
  \int \bbTInv(\vb x, \vb x^\prime) \cdot \vb J(\vb x^\prime) \,d\vb x^\prime
&= \vb E\sups{inc}(\vb x).
\label{ContinuousVIE}
\end{align}
where
$$
  \bbTInv(\vb x, \vb x^\prime) 
  =
  \bbVInv(\vb x) \delta(\vb x-\vb x^\prime)+ \mathbb{G}(\vb x,\vb x^\prime).
$$
(The symbol $\bbTInv$ is pronounced ``tee-inverse'' or ``eet''.)

\subsection*{Discretized VIE formulation}

Now let $\vb b_\alpha$ be some convenient set of $N$ vector-valued 
basis functions and approximate $j$ in the form
%====================================================================%
\numeq{JExpansion}
 { \vb J(\vb x) \approx \sum_{\alpha} j_\alpha \vb b_\alpha(\vb x). }
%====================================================================%
Insert into (\ref{ContinuousVIE}) and ``test'' both sides
with the elements of the set $\{\vb b_\alpha\}$ to obtain a
discretized version of (\ref{ContinuousVIE}) in the form of an
$N\times N$ linear system:
\begin{align*}
 \vbTInv \cdot \vb j &= \vb v
\end{align*}
where the elements of the vector $\vb j$ are the expansion 
coefficients in (\ref{JExpansion}) and the elements of 
$\vbTInv$ and $\vb v$ are 
%====================================================================%
\numeq{Elements}
{
 \TInv_{\alpha\beta}=\exptwoB{\vb b_\alpha}{\bbTInv}{\vb b_\beta},
 \qquad
 v_{\alpha}=-\frac{1}{ikZ_0} \inpB{\vb E\sups{inc}}{\vb b_\beta}.
}
%====================================================================%

%%%%%%%%%%%%%%%%%%%%%%%%%%%%%%%%%%%%%%%%%%%%%%%%%%%%%%%%%%%%%%%%%%%%%%
%%%%%%%%%%%%%%%%%%%%%%%%%%%%%%%%%%%%%%%%%%%%%%%%%%%%%%%%%%%%%%%%%%%%%%
%%%%%%%%%%%%%%%%%%%%%%%%%%%%%%%%%%%%%%%%%%%%%%%%%%%%%%%%%%%%%%%%%%%%%%
\newpage
\section{SWG Matrix Elements of the $\mathbb{G}$ operator}

%=================================================
%=================================================
%=================================================
\subsection{Distant case: Dipole and Quadrupole Approximations}

The $\mathbb{G}$-matrix element between two SWG basis functions
is
%====================================================================%
\numeq{bGb}
{
 \exptwoB{\vb b_\alpha}{\mathbb G}{\vb b_\beta}
 =\int_{\sup \vb b_\alpha} \, d\vb x_\alpha \,
  \int_{\sup \vb b_\beta} \, d\vb x_\beta\,
   b_{\alpha i}(\vb x_\alpha)
   \mathbb G_{ij}(\vb R_0 + \overline{\vb x}_\alpha - \overline{\vb x}_\beta)
   b_{\beta j}(\vb x_\beta)
}
%====================================================================%
where 
%====================================================================%
$$ \overline{\vb x}_\alpha \equiv \vb x_\alpha - \vb x_{\alpha 0}, 
   \qquad 
   \overline{\vb x}_\beta \equiv \vb x_\beta - \vb x_{\beta 0}, 
   \qquad 
   \vb R_0 = \vb x_{\alpha 0}-\vb x_{\beta 0}
$$
%====================================================================%
and $\vb x_{\alpha 0}, \vb x_{\beta 0}$ are the centroids of
the basis functions.

When the two basis functions are well separated 
(i.e. $|\vb R_0| \gg |\overline{\vb x}_\alpha|, |\overline{\vb x}_\beta|$),
we may retain just the first two terms in the Taylor expansion of 
$\mathbb G_{ij}$ about $\vb R_0$:
%====================================================================%
$$ \mathbb{G}_{ij}(\vb R_0 + \overline{\vb x}_\alpha - \overline{\vb x}_\beta)
  \,\approx\, \mathbb{G}_{ij}^0
         +\overline{x}_{\alpha k} \mathbb{G}_{ijk}^0
         -\overline{x}_{\beta  k} \mathbb{G}_{ijk}^0
         +\cdots
$$
%====================================================================%
where $\mathbb{G}^0_{ij}\equiv \mathbb{G}_{ij}(\vb R_0)$ 
and $\mathbb{G}_{ijk}^0\equiv \partial_k \mathbb{G}_{ij}(\vb R_0).$ 
Equation (\ref{bGb}) then reads (with a summation convention in force)
%====================================================================%
\numeq{DQApprox}
{
 \exptwoB{\vb b_\alpha}{\mathbb G}{\vb b_\beta}
\approx 
  \mc J_{\alpha i} \mathbb{G}^0_{ij} \mc J_{\beta j}
 +\mc Q_{\alpha ik} \mathbb{G}^0_{ijk} \mc J_{\beta j}
 -\mc J_{\alpha i} \mathbb{G}^0_{ijk} \mc Q_{\beta jk}
}
%====================================================================%
Here we have introduced the first and second moments
of the current distributions of the SWG functions:
\begin{align*}
 \mc J_{\alpha i}
\equiv \int_{\sup \vb b_\alpha} b_{\alpha i}(\vb x)\,d\vb x
\\
 \mc Q_{\alpha ij}
\equiv \int_{\sup \vb b_\alpha} b_{\alpha i}(\vb x) x_j\,d\vb x
\end{align*}

%=================================================
%=================================================
%=================================================
\subsection{Nearby case: Surface-integral method}
When the basis functions are not well separated, the
dipole/quadrupole expansion method of the previous 
subsection is inaccurate. In this case it is convenient 
to follow the work of Bleszynski et al.~\cite{} by reducing 
the volume integral (\ref{bGb}) to a nonsingular 
\textit{surface} integral:

\begin{align*}
 \exptwoB{\vb b_\alpha}{\mathbb G}{\vb b_\beta}
&=-\frac{1}{4\pi ik} 
  \int_{\partial \sup \vb b_\alpha} \, \!\!\!\!\! d\vb x_\alpha \,
  \int_{\partial \sup \vb b_\alpha} \, \!\!\!\!\! d\vb x_\beta
  \Big(\vbhat{n}_\alpha \cdot \vbhat{n}_\beta\Big)
  \Big[   T_1(\vb x_\alpha, \vb x_\beta)
        + T_2(\vb x_\alpha, \vb x_\beta)
  \Big]
\end{align*}
with
\begin{align*} 
     T_1(\vb x_\alpha, \vb x_\beta) 
  &= \vb b_\alpha(\vb x_\alpha ) \cdot \vb b_\beta(\vb x_\beta)
     h(ik R)
\\
     T_2(\vb x_\alpha, \vb x_\beta) 
  &= -\frac{9A_\alpha A_\beta}{V_\alpha V_\beta k^2}
      \Big\{ 9h(ikR) + w(ikR) 
             +k^2\big[(\vb Q_\alpha - \vb Q_\beta)\cdot \vb R\big]\,p(ikR)
      \Big\}
\end{align*} 
where the scalar functions $h,w,p$ are 
%====================================================================%
\begin{align*}
  h(x) &= \frac{1}{x}\texttt{ExpRelBar}(x,2) \\
  w(x) &= \texttt{ExpRelBar}(x,3) \\
  p(x) &= \frac{1}{x^3}\Big[ x\texttt{ExpRelBar}(x,2) 
                            - \texttt{ExpRelBar}(x,3)\Big] - \frac{1}{3}
\\
       &= \frac{h(x)}{x} - \frac{w(x)}{x^3} - \frac{1}{3}.
\end{align*}
Here the function \texttt{ExpRelBar} is defined by 
\begin{align*}
 \texttt{ExpRelBar}(x,N) 
&= e^{x} - \sum_{n=0}^{N-1} \frac{x^{n}}{N!}
\\
&= \sum_{n=N}^\infty \frac{x^n}{N!}
\end{align*}
I call this \texttt{ExpRelBar} because it is similar to
a similar function known as \texttt{ExpRel} (the two
functions differ only in their normalization).

%%%%%%%%%%%%%%%%%%%%%%%%%%%%%%%%%%%%%%%%%%%%%%%%%%%%%%%%%%%%%%%%%%%%%%
%%%%%%%%%%%%%%%%%%%%%%%%%%%%%%%%%%%%%%%%%%%%%%%%%%%%%%%%%%%%%%%%%%%%%%
%%%%%%%%%%%%%%%%%%%%%%%%%%%%%%%%%%%%%%%%%%%%%%%%%%%%%%%%%%%%%%%%%%%%%%
\section{SWG Matrix Elements of the $\mathbb{T}^{-1}$ Operator}

The basic observation underlying our approach is that,
while the $\mathbb{T}$ operator is highly unwieldy in position space,
the \textit{inverse} $\mathbb{T}$ operator is quite easy to 
describe in position space---in fact, it is just equal to the 
$\mathbb{G}$ operator plus a correction which is \textit{diagonal}
in the position-space basis. 

What this means is that, numerically, it's easy to compute
matrix elements of the \textit{inverse} $\mathbb{T}$ operator.
Our approach is to compute the matrix of this operator in 
a basis of SWG functions,
then simply \textit{invert} this matrix numerically to obtain
the actual $\mathbb{T}$ matrix.

\subsection{Matrix Elements of the Inverse $\mathbb{T}$ Operator}

For nonmagnetic materials $(\mu=1)$, the inverse $\mathbb{T}$ operator 
may be expressed in position space in the form
%%%%%%%%%%%%%%%%%%%%%%%%%%%%%%%%%%%%%%%%%%%%%%%%%%%%%%%%%%%%%%%%%%%%%%
\numeq{TInverseRealSpace}
{\mathbb{T}^{-1}(\omega; \vb x, \vb x^\prime)
   =\frac{c^2 }{\omega^2} 
    \underbrace{
    \Big[1-\vbeps(\omega, \vb x)\Big]^{-1}
               }_{\vb Y(\omega,\vb x)}
    \delta(\vb x-\vb x^\prime)
    + \mathbb{G}_0(\omega; \vb x, \vb x^\prime)
}
%%%%%%%%%%%%%%%%%%%%%%%%%%%%%%%%%%%%%%%%%%%%%%%%%%%%%%%%%%%%%%%%%%%%%%
where $\vbeps$ is the frequency- and position-dependent relative
dielectric permittivity tensor.

Using (\ref{TInverseRealSpace}), the matrix elements of $\mathbb{T}^{-1}$
take the form
%====================================================================%
\begin{align}
  \exptwoB{\vb b_\alpha}{\mathbb{T}^{-1}}{\vb b_\beta}
&=
  \frac{c^2}{\omega^2}
  \int \vb b_\alpha(\vb x) \cdot \vb Y(\omega,\vb x) \cdot \vb b_\beta(\vb x)
  \, d\vb x
  +
  \exptwoB{\vb b_\alpha}{\mathbb{G}}{\vb b_\beta}
\label{TInvME}
\end{align}
%====================================================================%
where $\vb Y$ at each point is computed as a $3\times 3$ matrix 
inverse:
%====================================================================%
$$\vb Y(\omega,\vb x) \equiv \Big[1 - \epsilon(\omega,\vb x)\Big]^{-1}.$$
%====================================================================%
The first integral in (\ref{TInvME}) extends over the common support of 
basis functions $\vb b_\alpha,\vb b_\beta$. For the SWG basis, each 
function shares support with only a small number (typically 7) of other 
functions, so this term amounts to a sparse update to the second term in 
(\ref{TInverseRealSpace}), which is dense.

%%%%%%%%%%%%%%%%%%%%%%%%%%%%%%%%%%%%%%%%%%%%%%%%%%%%%%%%%%%%%%%%%%%%%%
%%%%%%%%%%%%%%%%%%%%%%%%%%%%%%%%%%%%%%%%%%%%%%%%%%%%%%%%%%%%%%%%%%%%%%
%%%%%%%%%%%%%%%%%%%%%%%%%%%%%%%%%%%%%%%%%%%%%%%%%%%%%%%%%%%%%%%%%%%%%%
\newpage
\section{$\mathbb{T}\mathbb{G}$ Formulas
         for Casimir Forces and Heat-Transfer Rates}

The Kardar/Jaffe group has derived a number of formulas expressing
Casimir forces (both equilibrium and non-equilibrium) and
rates of heat transfer among a collection of material bodies 
in terms of the $\TT$ and $\GG$ operators of the
bodies. For example:
%====================================================================%
\begin{description}
 \item[1.] \textit{Equilibrium Casimir energy between two bodies:}
 $$\mathcal{E}=\frac{\hbar c}{2\pi}\int
   \log\det\Big(\vb{1} - \GG_{12}\mathbb{T}_2\GG_{21}\mathbb{T}_1\Big)
   d\xi
 $$
(where $\xi$ indicates imaginary frequency)
%====================================================================%
 \item[2.] \textit{Rate of heat radiation from a single body:}
 $$H=\frac{2}{\pi}
   \int \,\Theta(\omega) 
   \text{Tr }
   \Big[   \text{Im }\GG \, \text{Im }\TT 
         - \text{Im }\GG \, \TT \,
           \text{Im }\GG\, \TT^\dagger 
   \Big] \, d\omega
 $$
(where $\Theta(\omega)\equiv \frac{\hbar\omega}{e^{\hbar\omega/kT} - 1}$
 is the Bose-Einstein factor)
%====================================================================%
 \item[3.] \textit{Heat transfer from body 1 to body 2:}
 $$H_1^{(2)}=\frac{2}{\pi}
   \int \,\Theta(\omega) \mathcal{H}_1^{(2)}(\omega)\,d\omega
 $$
 $$\mathcal{H}_1^{(2)}=\text{Im Tr }
   \bigg\{ (1 + \GG \TT_2) \frac{1}{1-\GG \TT_1 \GG \TT_2}
           \GG 
           \Big[ \im \TT_1 - \TT_1 \big(\im \GG\big) \TT_1^\dagger \Big] 
           \GG^\dagger
           \frac{1}{1-\TT_2^\dagger \GG^\dagger \TT_1^\dagger \GG}\TT_2^\dagger
   \bigg\}
 $$
%====================================================================%
 \item[4.] \textit{Momentum transfer (Nonequilibrium Casimir force) from body 1 to body 2:}
 $$F_1^{(2)}=\frac{2}{\pi}
   \int \,\Theta(\omega) \mathcal{F}_1^{(2)}(\omega)\,d\omega
 $$
 $$\mathcal{F}_1^{(2)}=\text{Re Tr }
   \bigg\{ \nabla (1 + \GG \TT_2) 
           \frac{1}{1-\GG \TT_1 \GG \TT_2}
           \GG 
           \Big[ \im \TT_1 - \TT_1 \big(\im \GG\big) \TT_1^\dagger \Big] 
           \GG^\dagger
           \frac{1}{1-\TT_2^\dagger \GG^\dagger \TT_1^\dagger \GG}
           \TT_2^\dagger
   \bigg\}
 $$
\end{description}
%====================================================================%

Although the formulas are expressed as traces of 
certain linear-algebraic combinations of the operators 
and are are thus---in principle---basis-independent, in practice the 
$\mathbb{T}$ operator is so unwieldy that the formulas have only
ever been applied to certain high-symmetry geometries, in which
the known special forms of the Maxwell solutions allow 
analytical evaluation of $\mathbb{T}$-matrix elements.
In effect, this amounts to working in a basis of globally-defined
wave functions, which are typically highly specific to the 
particular shapes of the objects in question and thus not 
amenable to the treatment of more general geometries.
In particular, to date there has been no obvious way to use
$\mathbb{T}/\mathbb{G}$ formulas to handle \textit{inhomogeneous}
bodies with continuously varying material properties, since 
the $\mathbb{T}$-matrices cannot be computed analytically
for such bodies even if their shapes are highly symmetric.

Here I show that in fact the restriction to special-function
bases is in fact unnecessary. Instead, it is possible to compute 
$\mathbb{T}$-matrices \textit{numerically} for arbitrary 
inhomogeneous objects using a basis of \textit{localized} 
functions. This insight leads to a general-purpose approach
that is capable, in principle, of predicting Casimir and 
heat-transfer phenomena for bodies of arbitrary shapes and
arbitrary anisotropic continuously-varying material properties. 


%%%%%%%%%%%%%%%%%%%%%%%%%%%%%%%%%%%%%%%%%%%%%%%%%%%%%%%%%%%%%%%%%%%%%%
%%%%%%%%%%%%%%%%%%%%%%%%%%%%%%%%%%%%%%%%%%%%%%%%%%%%%%%%%%%%%%%%%%%%%%
%%%%%%%%%%%%%%%%%%%%%%%%%%%%%%%%%%%%%%%%%%%%%%%%%%%%%%%%%%%%%%%%%%%%%%
\appendix 

%%%%%%%%%%%%%%%%%%%%%%%%%%%%%%%%%%%%%%%%%%%%%%%%%%%%%%%%%%%%%%%%%%%%%%
%%%%%%%%%%%%%%%%%%%%%%%%%%%%%%%%%%%%%%%%%%%%%%%%%%%%%%%%%%%%%%%%%%%%%%
%%%%%%%%%%%%%%%%%%%%%%%%%%%%%%%%%%%%%%%%%%%%%%%%%%%%%%%%%%%%%%%%%%%%%%
\section{Volume integrals involving SWG basis functions}

Integrals over the support of SWG basis functions take the form
\numeq{SWGVolumeIntegral}
{
 \int_{\sup \vb b_\alpha} \mathcal{I}(\vb x, \vb b_\alpha) \, d\vb x
= 
  \int_{\mc P_\alpha^+} \mathcal{I}(\vb x, \vb b_\alpha) \, d\vb x
 +\int_{\mc P_\alpha^-} \mathcal{I}(\vb x, \vb b_\alpha) \, d\vb x
}
%%%%%%%%%%%%%%%%%%%%%%%%%%%%%%%%%%%%%%%%%%%%%%%%%%%%%%%%%%%%%%%%%%%%%%
Consider a tetrahedron with vertices 
$\{\vb Q, \vb V_1, \vb V_2, \vb V_3.\}$ 
A general
integral over this region takes the form
%====================================================================%
\begin{align*}
 \int_{\mc P} \mathcal{I}(\vb x, \vb b ) \, d\vb x
&=\mc{J} \int_0^1 \, du \int_0^{1-u} \, dv \int_0^{1-u-v} dw \,
  \mathcal{I}\Big( \vb x(u,v,w), \vb b(u,v,w) \Big)
\end{align*}
%====================================================================%
where
%====================================================================%
\begin{align*}
 \vb x(u,v,w) 
&= 
 \vb Q + u\vb L_1 + v\vb L_2 + w\vb L_3 
\\
 \vb b(u,v,w) 
&= 
 \pm \frac{A}{3V}\bigg\{ u \vb L_1 
                        +v \vb L_2 
                        +w \vb L_3 
                 \bigg\}
\end{align*}
%====================================================================%
and the Jacobian of the transformation is 
%====================================================================%
$$\mc J=\frac{d(x,y,z)}{d(u,v,w)}
   =\det\left|\begin{array}{ccc} 
     \\
     \vb L_1 & \vb L_2 & \vb L_3 \\
     \\
     \end{array}\right|
  =6V
$$

\subsection*{Dipole and Quadrupole Moments}

The dipole moment of the current distribution described by
a single unit-strength SWG basis function is 

$$ \vb p_\alpha 
   \equiv 
   \frac{i}{\omega} 
   \underbrace{\int_{\sup \vb b_\alpha} \vb b_\alpha(\vb x) \, d\vb x}
             _{\bmc J_\alpha(\vb x)}
$$
%====================================================================%
The Cartesian components of $\bmc J_\alpha(\vb x)$ may be worked out 
in closed form:
%====================================================================%
\begin{align*}
 \mc J_{\alpha i}(\vb x)
&=2A\int_0^1 \, du \, \int_0^{1-u} \, dv \, \int_0^{1-u-v}  \, dw \,
    (u+v+w) (\vb Q^- - \vb Q^+)_i
\\
&=\frac{A}{4} (\vb Q^- - \vb Q^+)_i.
\end{align*}
%====================================================================%
Similarly, the quadrupole moments are related to the quantity
\begin{align*}
 \mc Q_{\alpha ij}(\vb x)
&= \int_{\sup \vb b_\alpha} b_{\alpha i} (\vb x-\vb x_0)_j\, d\vb x
\intertext{where $\vb x_0=\frac{1}{3}(\vb V_1 + \vb V_2 + \vb V_3)$
           is the centroid of the basis function (which we take to 
           be the centroid of the triangle that constitutes the 
           common face).}
&=
2A\int_0^1 \, du \, \int_0^{1-u} \, dv \, \int_0^{1-u-v}  \, dw \,
 \bigg\{
\\
&\qquad\qquad 
 \hphantom{-}\,
  \bigg[ u\vb L_1^+ + v\vb L_2^+ + w\vb L_3+ \bigg]_i
  \left[ \left(u-\frac{1}{3}\right)\vb L_1^+
        +\left(v-\frac{1}{3}\right)\vb L_1^+
        +\left(w-\frac{1}{3}\right)\vb L_1^+
  \right]_j
\\
&\qquad\qquad 
 -\bigg[ u\vb L_1^- + v\vb L_2^- + w\vb L_3- \bigg]_i
  \left[ \left(u-\frac{1}{3}\right)\vb L_1^-
        +\left(v-\frac{1}{3}\right)\vb L_2^-
        +\left(w-\frac{1}{3}\right)\vb L_3^-
  \right]_j
 \bigg\}
\\
&=\frac{A}{20}
  \bigg\{ 
          Q^-_i \Big[ \vb Q^- - \vb x_0 \Big]_j
         -Q^+_i \Big[ \vb Q^+ - \vb x_0 \Big]_j
         +x_{0i}\Big[ \vb Q^+ - \vb x_0 \Big]_j
  \bigg\}
\end{align*}

\end{document}
