\documentclass[letterpaper]{article}

\input{physcmds}
\usepackage{cite}

\graphicspath{{figures/}}
\newcommand{\TInv}{\rotatebox[origin=c]{180}{$T$}}
\newcommand{\vbTInv}{\rotatebox[origin=c]{180}{$\vb T$}}
\newcommand{\vbTInvSmall}{\rotatebox[origin=c]{180}{\scriptsize{$\vb T$}}}
\newcommand{\bbTInv}{\rotatebox[origin=c]{180}{$\mathbb{T}$}}
\newcommand{\bbVInv}{\rotatebox[origin=c]{180}{$\mathbb{V}$}}
\newcommand{\vbSigma}{\boldsymbol{\Sigma}}
\newcommand{\fd}{^{\text{\tiny{F}}\dagger}}

% 'IEM' stands for 'imaginary part of epsilon matrix'
\newcommand{\vbIEM}{\boldsymbol{\Sigma}}
\newcommand{\IEM}{\Sigma}

\newcommand{\citeasnoun}[1]{Ref.~\citen{#1}}
\newcommand{\citeasnouns}[1]{Refs.~\citen{#1}}

%------------------------------------------------------------
%------------------------------------------------------------
%- Special commands for this document -----------------------
%------------------------------------------------------------
%------------------------------------------------------------
\newcommand{\vbxi}{\boldsymbol{\xi}}
\newcommand{\TT}{\mathbb{T}}
\newcommand{\GG}{\mathbb{G}}
\newcommand{\im}{\text{Im }}
\newcommand{\vbeps}{\boldsymbol{\epsilon}}

%------------------------------------------------------------
%------------------------------------------------------------
%- Document header  -----------------------------------------
%------------------------------------------------------------
%------------------------------------------------------------
\title { {\sc buff-em}: A Volume-Integral-Equation
         Solver Suite for Computational Fluctuation Physics}
\author {Homer Reid}
\date {June 3, 2014}

%------------------------------------------------------------
%------------------------------------------------------------
%- Start of actual document
%------------------------------------------------------------
%------------------------------------------------------------

\begin{document}
\pagestyle{myheadings}
\markright{Homer Reid: \texttt{buff-em} }
\maketitle

\tableofcontents

%%%%%%%%%%%%%%%%%%%%%%%%%%%%%%%%%%%%%%%%%%%%%%%%%%%%%%%%%%%%%%%%%%%%%%
%%%%%%%%%%%%%%%%%%%%%%%%%%%%%%%%%%%%%%%%%%%%%%%%%%%%%%%%%%%%%%%%%%%%%%
%%%%%%%%%%%%%%%%%%%%%%%%%%%%%%%%%%%%%%%%%%%%%%%%%%%%%%%%%%%%%%%%%%%%%%
\newpage
\section{Overview}
\label{OverviewSection}

The MIT Casimir theory group, led by Professors Kardar and 
Jaffe, has derived a number of formulas expressing
Casimir forces (both equilibrium and non-equilibrium) and
rates of heat transfer among a collection of material bodies 
in terms of the $\TT$ and $\GG$ operators of the bodies
\cite{Rahi2009, Kruger2012}.
Although the formulas are expressed as traces of 
certain linear-algebraic combinations of the operators 
and are are thus---in principle---basis-independent, in 
practice the definition of the $\mathbb{T}$ operator is so
unwieldy in the position-space basis that to date the formulas 
have only ever been applied to certain high-symmetry geometries, 
in which the known special forms of the Maxwell solutions allow
analytical evaluation of $\mathbb{T}$-matrix elements in
special bases (such as spherical wave functions for spherical
scatterers).
In effect, this amounts to working in a basis of globally-defined
wave functions, which are typically specific to the 
particular shapes of the objects in question and thus not 
amenable to the treatment of more general geometries.
In particular, to date there has been no obvious way to use
$\mathbb{T}/\mathbb{G}$ formulas to handle \textit{inhomogeneous}
bodies with continuously varying material properties, since 
the $\mathbb{T}$-matrices cannot be computed analytically
for such bodies even if their shapes are highly symmetric.

Here I show that in fact the restriction to special-function
bases is in fact unnecessary, and that, indeed, it is possible
to compute $\mathbb{T}$-matrices \textit{numerically} for
arbitrary inhomogeneous objects using a basis of \textit{localized}
functions. The key observation is that, although it is difficult 
to work with the $\mathbb{T}$ operator in real space, the
$\textit{inverse}$ of this operator has a convenient real-space
description.
This insight leads to a general-purpose approach
that is capable, in principle, of predicting Casimir and
heat-transfer phenomena for bodies of arbitrary shapes and
arbitrary anisotropic continuously-varying material properties.

As it turns out, the inverse $\mathbb{T}$ operator is 
closely related to the matrix that enters the classical
volume-integral-equation (VIE) method of computational 
electromagnetism~\cite{SWG1984}. Thus, the implementation
of a numerical $\mathbb{T}-$matrix-based computation scheme
for fluctuation physics amounts in practice to the 
implementation of a VIE solver. This is the motivation
for {\sc buff-em}.

%%%%%%%%%%%%%%%%%%%%%%%%%%%%%%%%%%%%%%%%%%%%%%%%%%%%%%%%%%%%%%%%%%%%%%
%%%%%%%%%%%%%%%%%%%%%%%%%%%%%%%%%%%%%%%%%%%%%%%%%%%%%%%%%%%%%%%%%%%%%%
%%%%%%%%%%%%%%%%%%%%%%%%%%%%%%%%%%%%%%%%%%%%%%%%%%%%%%%%%%%%%%%%%%%%%%
\newpage
\addcontentsline{toc}{section}{\large Part I: Theory}
\section{VIE formulation of classical EM scattering}

In this section I rederive the volume-integral-equation (VIE)
approach to classical electromagnetic scattering. This formulation
is standard, but I here describe it using terminology and
symbols that emphasize the connection to the $\mathbb{T}$-matrix
scattering approach used by the Kardar-Jaffe group.

\subsection*{Continuous VIE formulation}

Consider a material body with relative permittivity tensor
$\vbeps(\vb x)$ lying in vacuum and irradiated by monochromatic
sources which may lie inside or outside the body. 
Let the electric field due to the sources be $\vb E\sups{inc}$ 
and let $\vb J\supt{I}(\vb x)$ be the induced volume current
density throughout the bulk of the body. 
(We work at frequency $\omega$ and assume time dependence
of all field and currents $\propto e^{-i\omega t}$.)
The total electric field at any point is a sum of
``incident'' and ``scattered'' contributions:\footnote{We
put the terms ``incident'' and ``scattered'' in quotes
to remind readers that the ``incident''-field sources may
in fact lie \textit{inside} the body; in this case it is 
not quite right to refer to the field they produce as 
being ``incident'' on the body, but the terminology is 
convenient nonetheless.}
%====================================================================%
\begin{align}
 \vb E\sups{tot} &= \vb E\sups{inc} + \vb E\sups{scat}
\\
                 &= \vb E\sups{inc} + ikZ_0 \mathbb{G}*\vb J\supt{I}
\label{ETot}
\end{align}
%====================================================================%
where $k=\omega/c$ is the (free-space) wavenumber,
$Z_0=\sqrt{\mu_0/\epsilon_0}$ is the impedance of free 
space\footnote{Here and throughout we consistently eliminate
all reference to the free-space permittivity and
permeability constants $\epsilon_0,\mu_0$ in favor of
$Z_0=\sqrt{\mu_0/\epsilon_0}$ and $c=1/\sqrt{\epsilon_0 \mu_0}.$
For example, we write the combinations 
$\{\epsilon_0\omega, \mu_0\omega\}$ respectively in the form 
$\{\frac{k}{Z_0}, kZ_0\}$ where
$k=\omega/c$ is the free space wavelength.},
$*$ denotes convolution, and $\mathbb{G}$ is the free-space
dyadic Green's function:
%====================================================================%
$$ \mathbb{G}(\vb x, \vb x^\prime) = 
   \Big(\vb 1 - \frac{1}{k^2} \nabla \times \nabla \Big)
   \frac{e^{ikr}}{4\pi r}
   \qquad (r\equiv |\vb x-\vb x^\prime|).
$$
%====================================================================%
On the other hand, the induced current is related to the total field
according to
%====================================================================%
\begin{align*}
 \vb J\supt{I}(\vb x) 
&= -i\frac{k}{Z_0} \big[\vbeps(\vb x) - \vb 1\big]\cdot \vb E\sups{tot}(\vb x)
\\
&\equiv -\frac{1}{ikZ_0} \mathbb{V}(\vb x) \cdot \vb E\sups{tot}(\vb x)
\end{align*}
%====================================================================%
where $\mathbb{V}\equiv k^2\big[\vb 1 - \vbeps(\vb x)\big]$ is
sometimes~\cite{Rahi2009} known as the ``potential.''
At points not in free space, i.e. points at which 
$\vbeps(\vb x)\ne \vb 1$, we can invert this equation to read
%====================================================================%
\begin{align}
-ikZ_0 \bbVInv(\vb x) \cdot \vb J\supt{I}(\vb x) &= \vb E\sups{tot}(\vb x)
\nonumber
\intertext{where $\bbVInv\equiv \mathbb{V}^{-1}.$ Now insert (\ref{ETot}):}
-ikZ_0 \bbVInv(\vb x) \cdot \vb J\supt{I}(\vb x) 
&= \vb E\sups{inc} + ik Z_0 \mathbb{G} \star \vb J\supt{I} \\
\intertext{Rearranging and writing out the convolution, 
           we obtain a volume-integral equation for $\vb J\supt{I}$:} 
  -ikZ_0\left[   \bbVInv(\vb x) \cdot \vb J\supt{I}(\vb x)
              + \int \mathbb{G}(\vb x, \vb x^\prime) 
                \cdot \vb J\supt{I}(\vb x^\prime) \,d\vb x^\prime
        \right]
&= \vb E\sups{inc}(\vb x).
\\
\intertext{In what follows it will be convenient to think of the RHS 
           here as the convolution of just a single operator with $\vb J\supt{I}$:}
 -ikZ_0
  \int \bbTInv(\vb x, \vb x^\prime) \cdot \vb J\supt{I}(\vb x^\prime) \,d\vb x^\prime
&= \vb E\sups{inc}(\vb x).
\label{ContinuousVIE}
\end{align}
where
\numeq{TOperator}
{
  \bbTInv(\vb x, \vb x^\prime) 
  =
  \bbVInv(\vb x) \delta(\vb x-\vb x^\prime)+ \mathbb{G}(\vb x,\vb x^\prime).
}
(The symbol $\bbTInv$ is pronounced ``tee-inverse'' or ``eet''.)

%=================================================
%=================================================
%=================================================
\subsection*{Discretized VIE formulation}

Now let $\vb b_\alpha$ be some convenient set of $N$ vector-valued 
basis functions and approximate the induced current in the form
%====================================================================%
\numeq{JExpansion}
 { \vb J\supt{I}(\vb x) \approx \sum_{\alpha} j\supt{I}_\alpha \vb b_\alpha(\vb x). }
%====================================================================%
Insert into (\ref{ContinuousVIE}) and ``test'' both sides
with the elements of the set $\{\vb b_\alpha\}$ to obtain a
discretized version of (\ref{ContinuousVIE}) in the form of an
$N\times N$ linear system:
%====================================================================%
\numeq{VIESystem}
{
 \vbTInv \cdot \vb j\supt{I} = \vb v
}
%====================================================================%
where the elements of the vector $\vb j\supt{I}$ are the expansion 
coefficients in (\ref{JExpansion}) and the elements of 
$\vbTInv$ and $\vb v$ are 
%====================================================================%
\numeq{Elements}
{
 \TInv_{\alpha\beta}=\exptwoB{\vb b_\alpha}{\bbTInv}{\vb b_\beta},
 \qquad
 v_{\alpha}=-\frac{1}{ikZ_0} \inpB{\vb E\sups{inc}}{\vb b_\beta}.
}
%====================================================================%

\subsubsection*{VIE matrix for geometries containing multiple bodies}

Equation (\ref{VIESystem}) applies to the case in which we 
have only a single material body. For a geometry involving
$N$ separate bodies, this is generalized to read
%====================================================================%
\numeq{VIESystem2}
{
 \underbrace{
 \left(\begin{array}{cccc}
 \vbTInv_1  & \vb G_{12}  & \cdots & \vb G_{1N} \\
 \vb G_{21} & \vbTInv_{2} & \cdots & \vb G_{2N} \\ 
 \vdots     & \vdots      & \ddots & \vdots     \\ 
 \vb G_{N1} & \vb G_{N2}  & \cdots & \vbTInv_N 
 \end{array}\right)
            }_{\vb M}
%--------------------------------------------------------------------%
\left(\begin{array}{c} 
 \vphantom{\vbTInv_1} \vb j_1\supt{I} \\
 \vphantom{\vbTInv_1} \vb j_2\supt{I} \\
 \vphantom{\ddots}    \vdots  \\
 \vphantom{\vbTInv_1} \vb j_N\supt{I} \\
\end{array}\right)
%--------------------------------------------------------------------%
 =
\left(\begin{array}{c} 
 \vphantom{\vbTInv_1} \vb v_1 \\
 \vphantom{\vbTInv_1} \vb v_2 \\
 \vphantom{\ddots}    \vdots  \\
 \vphantom{\vbTInv_1} \vb v_N \\
\end{array}\right).
}
%====================================================================%
where $\vb M$ is the VIE matrix for the composite system.
Here the diagonal blocks involve just the matrix elements
of the inverse $\mathbb{T}$-operator [defined by equation (\ref{TOperator})
with the $\bbVInv$ operator appropriate for the $n$th body]
while the off-diagonal blocks involve only the matrix elements of 
the $\mathbb{G}$ operator. 

\subsection*{Vector subblocks and projection matrices}

In problems involving two or more bodies it is convenient 
to define projection matrices that operate on full vectors,
like those in (\ref{VIESystem2}), to extract only the 
portions corresponding to body $n$, i.e. we define the
$n$th projection matrix $\vb P_n$ according to 
%====================================================================%
\numeq{ProjectionMatrix}
{
\underbrace{
  \left(\begin{array}{ccccc} 
         0 & \cdots & 0     & \cdots & 0 \\
   \vdots  & \ddots & 0     & \vdots & 0 \\
         0 &      0 & \vb 1 & \cdots & 0 \\
   \vdots  & \vdots & 0     & \ddots & 0 \\
         0 &      0 & 0     & \cdots & 0 \\
  \end{array}\right)
           }_{\vb P_n}
\left(\begin{array}{c} 
 \vb j_1 \\
 \vdots  \\
 \vb j_n \\
 \vdots  \\
 \vb j_N
 \end{array}\right)
=
\left(\begin{array}{c} 
 0       \\
 \vdots  \\
 \vb j_n \\
 \vdots  \\
  0       
 \end{array}\right)
}

\subsection*{Symmetries of the VIE matrix}

\subsubsection*{Symmetries of the $\bbVInv$ matrix} 

%====================================================================%
\begin{align}
 \exptwoB{\vb b_\alpha}{\bbVInv}{\vb b_\beta}
&=\frac{1}{k^2} 
   \int \, d\vb x \,
   b_{\alpha i}(\vb x_\alpha) 
   \Big[\vb {1} - \vbeps(\vb x)\Big]^{-1}_{ij}
   b_{\beta j}(\vb x_\beta)
\nn
&=\exptwoB{\vb b_\beta}{\bbVInv}{\vb b_\alpha}
\label{VInvSymmetric}
\intertext{\textbf{if} we have that}
\vbeps &= \vbeps^T \qquad (\textit{not } \vbeps=\vbeps^\dagger!).
\label{EpsSymmetric}
\end{align}

\subsubsection*{Symmetries of the $\mathbb{G}$ matrix} 
%====================================================================%
\begin{align*}
 \exptwoB{\vb b_\alpha}{\mathbb{G}}{\vb b_\beta}
&=\int \, d\vb x_\alpha \, d\vb x_\beta \,
   b_{\alpha i}(\vb x_\alpha) 
   \mathbb{G}_{ij}(\vb x_\alpha-\vb x_\beta)
   b_{\beta j}(\vb x_\beta)
\\
\intertext{Now invoke the relation 
           $\mathbb{G}_{ij}(\vb r) = \mathbb{G}_{ji}(-\vb r):$
          }
&=\int \, d\vb x_\alpha \, d\vb x_\beta \,
   b_{\beta j}(\vb x_\beta)
   \mathbb{G}_{ji}(\vb x_\beta - \vb x_\alpha)
   b_{\alpha i}(\vb x_\alpha) 
\\
&=
 \exptwoB{\vb b_\beta}{\mathbb{G}}{\vb b_\alpha}
\end{align*}
%====================================================================%

%%%%%%%%%%%%%%%%%%%%%%%%%%%%%%%%%%%%%%%%%%%%%%%%%%%%%%%%%%%%%%%%%%%%%%
%%%%%%%%%%%%%%%%%%%%%%%%%%%%%%%%%%%%%%%%%%%%%%%%%%%%%%%%%%%%%%%%%%%%%%
%%%%%%%%%%%%%%%%%%%%%%%%%%%%%%%%%%%%%%%%%%%%%%%%%%%%%%%%%%%%%%%%%%%%%%
\newpage
\section{FVC approach to fluctuation-induced phenomena}

In this section I consider a collection of one or 
more material bodies $\{\mc B_n\}$, at various temperatures
$\{T_n\}$ and embedded in an environment at
temperature $T\subs{env}$, and derive a sequence of concise
matrix-trace formulas expressing thermally and quantum-mechanically 
averaged heat-transfer rates, forces, and torques on the bodies 
in terms of the $\vb T$ and $\vb G$ matrices discussed in the 
previous section. Because the resulting energy and momentum
transfers may be viewed as arising from fluctuations in volume 
currents in the bodies, I term this the 
``fluctuating volume-current'' (FVC) approach to fluctuation
physics.

The derivation proceeds in two steps.

\begin{enumerate}
%--------------------------------------------------------------------%
\item I first consider a fixed, deterministic volume electric 
current distribution $\vb J\supt{F}(\vb x)$---confined to the interiors
of our material bodies but otherwise arbitrary---and use
the VIE formalism of the previous section to derive compact 
expressions for the rates of energy and momentum absorption 
by the bodies. These expressions will be quadratic (bilinear) 
functions of $\vb J\supt{F}$. (The F superscript stands
for ``free''; it distinguishes the fixed, externally-imposed
current $\vb J\supt{F}$ from the \textit{induced} current
$\vb J\supt{I}$ to which it gives rise.)
%--------------------------------------------------------------------%
\item I then average over thermal and quantum-mechanical
fluctuations of $\vb J\supt{F}$ to derive temperature-dependent
mean heat-transfer rates and forces on the bodies.
%--------------------------------------------------------------------%
\end{enumerate}
In what follows I will go back and forth somewhat freely between
continuous operator/field notation 
[involving symbols like $\mathbb{G}$ and $\vb E(\vb x)$] and 
discretized matrix/vector notation 
(involving symbols like $\vb G$ and $\vb e$). For a precise
dictionary of the correspondence, see Appendix \ref{CorrespondenceAppendix}.

%%%%%%%%%%%%%%%%%%%%%%%%%%%%%%%%%%%%%%%%%%%%%%%%%%%%%%%%%%%%%%%%%%%%%%
%%%%%%%%%%%%%%%%%%%%%%%%%%%%%%%%%%%%%%%%%%%%%%%%%%%%%%%%%%%%%%%%%%%%%%
%%%%%%%%%%%%%%%%%%%%%%%%%%%%%%%%%%%%%%%%%%%%%%%%%%%%%%%%%%%%%%%%%%%%%%
\subsection{Energy and momentum transfer from volume-current bilinears}

Consider a collection of material bodies $\{\mc B_n\}$ and a 
fixed, deterministic volume current distribution $\vb J\supt{F}(\vb x)$
that is nonzero only inside the bodies.
(We work at a fixed frequency $\omega$ with all fields and currents 
varying in time like $e^{-i\omega t}.$)
In this section we derive formulas expressing time-average rates 
of energy and momentum absorption by the bodies as bilinear
functions of $\vb J\supt{F}.$

\subsubsection*{Induced currents from free currents}

The free current distribution $\vb J\supt{F}(\vb x)$ excites an 
induced current distribution $\vb J\supt{I}(\vb x)$ which we can 
determine using the VIE techniques of the previous section. 
Indeed, taking $\vb J\supt{F}$ as the source of the incident field 
in a scattering problem, we have
%====================================================================%
\begin{align}
 \vb E\sups{inc}(\vb r) &= ikZ_0 \mathbb{G} \star \vb J\supt{F}
\intertext{and the RHS vector of the discretized VIE system, equation
           (\ref{VIESystem2}), reads}
 \vb v                  &= -\vb G \, \vb j\supt{F}
\label{ScatProbRHS}
\end{align}
%====================================================================%
where, for a geometry consisting of $N$ bodies, the vectors and
matrices have an $N$-fold block structure:
%====================================================================%
\renewcommand{\arraystretch}{1.5}
$$ \vb v 
   = 
   \left(\begin{array}{c} 
     \vphantom{\vb j\supt{F}_1} \vb v_1 \\
     \vphantom{\vb j\supt{F}_2} \vb v_2 \\
     \vdots \\
     \vphantom{\vb j\supt{F}_N} \vb v_N
   \end{array}\right),
%--------------------------------------------------------------------%
\qquad  
   \vb G = \left(\begin{array}{cccc}
    \vphantom{\vb j\supt{F}_1} \vb G_{11} & \vb G_{12}  & \cdots & \vb G_{1N} \\
    \vphantom{\vb j\supt{F}_1} \vb G_{21} & \vb G_{22}  & \cdots & \vb G_{2N} \\
    \vdots                          & \vdots      & \ddots & \vdots     \\ 
    \vphantom{\vb j\supt{F}_1} \vb G_{N1} & \vb G_{N2}  & \cdots & \vb G_{NN}
   \end{array}\right), 
\qquad
%--------------------------------------------------------------------%
   \vb j\supt{F} 
   = 
   \left(\begin{array}{c} 
     \vb j\supt{F}_1 \\
     \vb j\supt{F}_2 \\ 
     \vdots    \\
     \vb j\supt{F}_N
   \end{array}\right).
$$
\renewcommand{\arraystretch}{1.0}
%====================================================================%
In particular, the $n$th subblock of $\vb j\supt{F}$ is the
projection of the free current distribution\footnote{Note that 
$\vb J_n\supt{F}(\vb x)$ is just the restriction of $\vb J\supt{F}$ 
to the interior of $\mc B_n$.} in body $\mc B_n$, 
$\vb J_n\supt{F}(\vb x)$, onto the subset of basis functions whose 
support lies in body $n$:
%====================================================================%
\numeq{jnF}
{ j_{n\alpha}\supt{F} 
  = 
  \int \vb b_{n\alpha}(\vb r) \cdot \vb J_n\supt{F}(\vb r) \, d\vb r.
}
%====================================================================%
Now taking equation (\ref{ScatProbRHS}) to be the RHS of the VIE scattering
problem (\ref{VIESystem2}), we obtain an expression for the induced
currents in terms of the free currents, 
%====================================================================%
\begin{align}
  \vb M \, \vb j\supt{I} &= - \vb G \vb j\supt{F} 
\\
\intertext {or}
  \vb j\supt{I} &= -\vb W \vb G \vb j\supt{F} 
\label{jiFromjf}
\end{align}
%====================================================================%
where $\vb W=\vb M^{-1}$ is the inverse of the VIE matrix 
defined by (\ref{VIESystem2}).

The \textit{total} current is 
\begin{align}
 \vb j &= \vb j\supt{F} + \vb j\supt{I}
\nn
  &= \Big[ \vb 1 - \vb W \vb G\Big] \vb j\supt{F}.
\label{jTot}
\end{align}

\subsubsection*{Fields from free currents}

The $\vb E$-field at an arbitrary point in space (either
inside or outside a body) is then simply the sum of 
contributions from fixed and induced currents:
%====================================================================%
\begin{align}
 \vb E &= ikZ_0 \mathbb{G} \star (\vb J\supt{F} + \vb J\supt{I})
\label{EConvolution} \\
\intertext{or, in discretized form (Appendix \ref{CorrespondenceAppendix}),}
 \vb e &= ikZ_0 \vb G (\vb j\supt{F} + \vb j\supt{I}) 
\nn
       &= ikZ_0 \vb G \Big( \vb 1 - \vb W \vb G \Big) \vb j\supt{F}
\label{efromjf}
\end{align}
%====================================================================%
where in going to the last line I used (\ref{jiFromjf}).

In what follows I will also need the the quantity 
$\partial_i \vb E(\vb r),$ i.e. the derivative of $\vb E$ with
respect to the evaluation point. Differentiating both sides of
(\ref{EConvolution}), we see that the derivative operates
on the first argument of $\mathbb{G}(\vb r, \vb r^\prime)$
and leaves everything else on the RHS untouched; thus we find
simply 
%====================================================================%
\numeq{diefromfj}
{\partial_i \vb e = ikZ_0 \Big[\partial_i \vb G\Big] (\vb j\supt{F} + \vb j\supt{I})}
%====================================================================%
where the matrix elements of the quantity in square brackets are
$$ \Big[ \partial_i \vb G\Big]_{\alpha\beta}
  = \int \int \vb b_\alpha(\vb r) 
              \left[\pard{}{\vb r_i} \mathbb{G}(\vb r, \vb r^\prime)\right]
              \vb b_\beta(\vb r^\prime) \, d\vb r \, d\vb r^\prime.
$$

\subsubsection*{Power absorption}

The time-average rate at which body $\mc B_n$ absorbs power from
the source distribution $\vb J\supt{F}$ is obtained by integrating
the quantity $\frac{1}{2}\text{Re }\vb J^* \cdot \vb E$ over the 
interior of $\mc B_n$; here $\vb J$ is the \textit{total} current 
in $\mc B_n$, consisting of both free and induced contributions:
%====================================================================%
\begin{align}
 P_n(\omega) 
 &= 
 \frac{1}{2}\text{Re }\int_{\mc B_n} \vb J^*(\vb r) \cdot \vb E(\vb r) \, d\vb r
\label{PowerVolumeIntegral} \\
 &= \frac{1}{2}\text{Re }\vb j_n^\dagger \cdot \vb e_n \nonumber
\intertext{where we used equation (\ref{VolIntToDotProd}). 
           The $n$ subscript on vectors picks out the subblock
           corresponding to body $n$; using the projection matrices
           $\vb P_n$ defined by (\ref{ProjectionMatrix}), we 
           could equivalently write this in the form}
 &= \frac{1}{2}\text{Re }\vb j^\dagger \vb P_n \vb e \nonumber
\intertext{Now insert equations (\ref{jTot}) and (\ref{efromjf}):}
 &= \frac{1}{2}\text{Re }\left\{ ikZ_0 \, 
    \vb j\fd \Big[\vb 1-\vb G^\dagger \vb W^\dagger \Big]
    \vb P_n
    \vb G \Big[\vb 1 - \vb W \vb G\Big] \vb j\supt{F} \right\}
\nn
 &= -\frac{kZ_0}{2}\text{ Im Tr }\left\{ 
    \Big[\vb 1-\vb G^\dagger \vb W^\dagger \Big]
    \vb P_n
    \vb G \Big[\vb 1 - \vb W \vb G\Big] 
    \Big[\vb j\supt{F} \vb j\fd \Big]
    \right\}.
\label{DeterministicPowerTraceFormula}
\end{align}
%====================================================================%
As advertised, this expression depends quadratically on
$\vb J\supt{F}$, as witness the appearance of the outer matrix
product $\vb j\supt{F} \vb j\fd$.

%=================================================
%=================================================
%=================================================
\subsubsection*{Momentum absorption}

The time-average rate at which body $\mc B_n$ absorbs $i$-directed
\textit{momentum} from the source distribution $\vb J\supt{F}$---that
is, the $i$-directed force on the body---may be expressed as a
volume-integral expression very similar to that of
(\ref{PowerVolumeIntegral})
but with 
 \textbf{(i)} ``Re'' replaced by ``Im,''
 \textbf{(ii)} $\vb E$ replaced by $\partial_i \vb E$, 
and 
 \textbf{(iii)} an extra factor of $\omega$ in the denominator:
%====================================================================%
\begin{align*}
 T_n(\omega) 
 &= 
 \frac{1}{2\omega}
 \text{Im }\int_{\mc B_n} \vb J^*(\vb r) \cdot \partial_i \vb E(\vb r) d\vb r.
\intertext{(This expression is quoted in~\citeasnoun{Kruger2012}; I
            also provide a quick derivation in
            Appendix \ref{ForceFormulaAppendix}.)
            Going over to the discrete world, we have}
 &= 
 \frac{1}{2\omega}
 \text{Im }\vb j^\dagger \vb P_n (\partial_i \vb e)
\intertext{Insert (\ref{jTot}) and (\ref{efromjf}):}
 &= 
\frac{1}{2\omega}\text{Im }
    \left\{ ikZ_0 \, 
    \vb j\fd \Big[\vb 1-\vb G^\dagger \vb W^\dagger \Big]
    \vb P_n
    (\partial_i \vb G) \Big[\vb 1 - \vb W \vb G\Big] \vb j\supt{F} \right\}
\nn
&= \frac{kZ_0}{2\omega}\text{Re Tr}
    \left\{ 
    \Big[\vb 1-\vb G^\dagger \vb W^\dagger \Big]
    \vb P_n
    (\partial_i \vb G) \Big[\vb 1 - \vb W \vb G\Big] \vb j\supt{F} \vb j\fd
   \right\}
\end{align*}
%====================================================================%

%%%%%%%%%%%%%%%%%%%%%%%%%%%%%%%%%%%%%%%%%%%%%%%%%%%%%%%%%%%%%%%%%%%%%%
%%%%%%%%%%%%%%%%%%%%%%%%%%%%%%%%%%%%%%%%%%%%%%%%%%%%%%%%%%%%%%%%%%%%%%
%%%%%%%%%%%%%%%%%%%%%%%%%%%%%%%%%%%%%%%%%%%%%%%%%%%%%%%%%%%%%%%%%%%%%%
\subsection{Statistical averages of volume-current bilinears}

The classical, deterministic expressions derived above for 
time-average quantities $Q$ (where $Q$ is a power, force,
or torque) all take the form
%====================================================================%
\numeq{QofW}
{Q(w) 
   \propto 
   \Tr \Big\{ \vb Q(\omega) \cdot \big(\vb j\supt{F} \vb j\fd\big) \Big\}
}
%====================================================================%
where $\vb Q(\omega)$ is a frequency-dependent matrix. The
statistical \textit{average} of such quantities is performed
by averaging over all possible free current distributions
$\vb J\supt{F}(\vb r)$, which amounts to computing the statistical
average of the matrix $\vb j\supt{F} \vb j\fd$:
%====================================================================%
\begin{align*}
 \big\langle Q(\omega) \big\rangle
   \propto 
   \Tr \Big\{ \vb Q(\omega) \cdot 
              \big\langle \vb j\supt{F} \vb j\fd\big\rangle_{\omega} 
       \Big\}.
\end{align*}
This quantity represents just the contribution of frequency-$\omega$
fluctuations to the average power, force or torque (PFT); the  
\textit{total} PFT is given by integrating over all frequencies,
$ Q=\int_{0}^\infty \, \langle Q(\omega) \rangle \,d\omega.$

%====================================================================%
The elements of the matrix $\vb j\supt{F}\vb j\fd$ are 
%====================================================================%
\begin{align}
 \Big[ \vb j\supt{F} \vb j\fd \Big]_{\alpha\beta}
&=\int \int     b_{\alpha i}(\vb r) 
                J\supt{F}_i(\vb r) J^{\text{\tiny{F}}*}_j(\vb r^\prime)
                b_{\beta j}(\vb r^\prime)
  \, d\vb r \, d\vb r^\prime
%--------------------------------------------------------------------%
\intertext{Now perform the statistical average. The only quantities
on the RHS that experience averaging are the factors of $J$ in the
integrand:}
%--------------------------------------------------------------------%
 \Big\langle \vb j\supt{F} \vb j\fd \Big\rangle_{\alpha\beta}
&=\int \int     b_{\alpha i}(\vb r) 
 \Big\langle
                J\supt{F}_i(\vb r) J^{\text{\tiny{F}}*}_j(\vb r^\prime)
 \Big\rangle
                b_{\beta j}(\vb r^\prime)
  \, d\vb r \, d\vb r^\prime.
\label{jjd1}
\end{align}
We now make use of the 
fluctuation-dissipation theorem in the form of the Rytov correlation
function,\footnote{How do the \textit{units} of this equation work?
To answer this question, I think it's easiest to multiply both
sides by $Z_0$ to bring a factor of impedance to the LHS.
Since $J$ is the Fourier transform of a volume current density, it
has units of 
$\frac{\text{current}}{\text{length}^2}\cdot \frac{1}{\text{frequency}}$.
and the LHS then has units of 
$\frac{\text{impedance} \cdot \text{current}^2 \cdot \text{time}^2}
      {\text{length}^4}
 =\frac{\text{energy}\cdot\text{time}}
      {\text{length}^4}
$ where we used that $\text{impedance}\cdot\text{current}^2=\text{power}$
(for example, recall the $\sim I^2 R$ dependence of Joule heating)
and $\text{power}\cdot\text{time}=\text{energy}$. Meanwhile, on
the RHS, the dimensionful factors remaining after multiplying by 
$Z_0$ are $\Theta$ (energy) and $k\delta(\vb r-\vb r^\prime)$ 
(length$^{-4}$) so everything works out modulo a factor of inverse 
frequency on the RHS, which I think must be coming from a factor
like $\delta(\omega-\omega^\prime)$ that is implicit somewhere.}
%====================================================================%
\numeq{Rytov}
{ \big\langle J\supt{F}_i(\vb r) J^{\text{\tiny{F}}*}_j(\vb r^\prime)
  \big\rangle
  = \frac{2k}{\pi Z_0} \Theta(T) \delta(\vb r-\vb r^\prime) 
    \text{Im }\epsilon_{ij}(\vb r).
}
%====================================================================%
Inserting this into (\ref{jjd1}), we find
%====================================================================%
$$
 \Big\langle \vb j\supt{F} \vb j\fd \Big\rangle_{\alpha\beta}
 = \frac{2k}{\pi Z_0}
     \int \Theta\big[T(\vb r)\big]
          b_{\alpha i}(\vb r) 
          \Big[\text{Im }\epsilon_{ij}(\vb r) \Big]
          \vb b_{\beta  j}(\vb r)
  \, d\vb r. 
$$
%====================================================================%
For the situation we consider here---involving $N$ material bodies,
throughout the interior of which the temperature is constant---the
full matrix takes the form 
%====================================================================%
\numeq{jjdMatrix}
{
 \Big\langle \vb j\supt{F} \vb j\fd \Big\rangle
 =\frac{2k}{\pi Z_0}
  \left(\begin{array}{cccc}
  \Theta(T_1) \vbIEM_1 & 0 & \cdots & 0 \\ 
  0 & \Theta(T_2) \vbIEM_2 & \cdots & 0 \\ 
  \vdots & \vdots & \ddots & \vdots   \\
  0 & 0 & \cdots & \Theta(T_N) \vbIEM_N \\
  \end{array}\right)
}
%====================================================================%
where $\vbIEM_n$ is just the matrix of basis-function overlaps
with the imaginary part of the relative dielectric function
for the $n$th body, i.e.
%====================================================================%
\numeq{IEM}
{ \Big[\vbIEM_n\Big]_{\alpha\beta} = 
   \int b_{\alpha i}(\vb r) \Big[ \text{Im } \vbeps_{n}(\vb r)\Big]_{ij}
        b_{\beta j}(\vb r) d\vb r.
}
%====================================================================%
Here $\vbeps_n$ is the dielectric tensor for body $\mc B_n.$
For a basis of localized functions, this matrix is highly sparse
(for the SWG basis discussed below it contains just 7 nonzero elements 
per row). Moreover, numerical evaluation of the matrix elements of 
$\vbIEM$ is essentially costless, particularly compared to the cost of 
computing matrix elements of $\mathbb{G}$;
it involves just a single three-dimensional numerical cubature
and may be carried out simultaneously with computation of the 
matrix elements of the $\bbVInv$ operator needed to assemble the 
VIE matrix.

Inserting (\ref{jjdMatrix}) into ... 

The total heat transfer to, and the total $i$-directed 
force and torque on, a destination body $\mc B_d$ are given by
%====================================================================%
\begin{align*}
 H_d    = \int_0^\infty \mc H_d(\omega) d\Omega,         \qquad 
 F_{di} = \int_0^\infty \mc F_{di}(\omega) d\Omega, \qquad 
 T_{di} = \int_0^\infty \mc T_{di}(\omega) d\Omega
\end{align*}
%====================================================================%
where the contribution of each frequency $\omega$ may be written
as the sum of equilibrium contributions plus non-equilibrium
contributions from all other bodies acting as sources, with 
each contribution expressed as a thermal/statistical factor 
times a generalized flux: 
%====================================================================%
\begin{align*}
 \mc H_d(\omega)
 &= \sum_{s} 
    \Big[ \Theta\big(\omega,T_s) - \Theta\big(\omega, T\subs{env}\big) \Big]
    \Phi\sups{energy}_{s\to d}(\omega)
\\
 \mc F_{di}(\omega)
 &= F_{di}\sups{eq}\big( T\subs{env} \big)
   +\sum_{s} 
    \Big[ \Theta\big(\omega,T_s) - \Theta\big(\omega, T\subs{env}\big) \Big]
    \Phi\sups{lin mom}_{s\to d}(\omega)
\\
 \mc T_{di}(\omega)
 &= T_{di}\sups{eq}\big( T\subs{env} \big)
   +\sum_{s} 
    \Big[ \Theta\big(\omega,T_s) - \Theta\big(\omega, T\subs{env}\big) \Big]
    \Phi\sups{ang mom}_{s\to d}(\omega)
\end{align*}
%====================================================================%
(For the heat transfer there is no equilibrium contribution, as there
is no net exchange of energy between equal-temperature bodies. There
\textit{is} a net transfer of \textit{momentum}, whereupon the force and
torque expressions do contain equilibrium contributions.)
%====================================================================%
\begin{align*}
 \Phi\sups{energy}_{s\to d}(\omega)
&=-\frac{k^2}{\pi}\text{Im Tr }
   \left\{ 
           \Big[ \vb G (\vb 1 - \vb W \vb G)\Big]_{ds}
           \vbSigma_s
           \Big[\vb 1 - \vb G^\dagger \vb W^\dagger\Big]_{sd}
   \right\}
\\
 \Phi\sups{lin mom}_{s\to d}(\omega)
&= \frac{k^2}{\pi \omega}\text{Re Tr }
   \left\{ 
           \Big[ \big(\partial_i \vb G\big) (\vb 1 - \vb W \vb G)\Big]_{ds}
           \vbSigma_s
           \Big[\vb 1 - \vb G^\dagger \vb W^\dagger\Big]_{sd}
   \right\}
\\
 \Phi\sups{ang mom}_{s\to d}(\omega)
&= \frac{k^2}{\pi \omega}\text{Re Tr }
   \left\{ 
           \Big[ \big(\partial_\theta \vb G\big) (\vb 1 - \vb W \vb G)\Big]_{ds}
           \vbSigma_s
           \Big[\vb 1 - \vb G^\dagger \vb W^\dagger\Big]_{sd}
   \right\}
\end{align*}
%====================================================================%
The trace we compute in these expressions has the form
%====================================================================%
\begin{align*}
\text{Tr }\Big[ \left(\vb X \vb A\right)_{ds}
                \vbSigma_s
                \big(\vb A^\dagger\big)_{sd} 
          \Big]
&= \sum_{ijk} \left(\vb X \vb A\right)_{ds, ij}
              \Sigma_{s,jk}
              \Big[\vb A^\dagger_{sd}\Big]_{ki}
\\
&= \sum_{ijk} \big(\vb X \vb A\big)_{di; sj}
              \Sigma_{s,jk}
              A^*_{di;sk}
\end{align*}
%====================================================================%
where 
%====================================================================%
$$\vb A = \Big( \vb 1 - \vb W \vb G\Big)$$
%====================================================================%
and
%====================================================================%
$$ \vb X=\{\vb G, \partial_i \vb G, \partial_\theta \vb G\}.$$
%====================================================================%
%For example, suppose in a four-object geometry we want the 
%force on object 2 due to sources in object 3.
%%====================================================================%
%$$ \left(\begin{array}{cccc}
%   \vb A^\dagger_{11} & \vb A^\dagger_{12} & \vb A^\dagger_{13} & \vb A^\dagger_{14} \\
%   \vb A^\dagger_{21} & \vb A^\dagger_{22} & \vb A^\dagger_{23} & \vb A^\dagger_{24} \\
%   \vb A^\dagger_{31} & \vb A^\dagger_{32} & \vb A^\dagger_{33} & \vb A^\dagger_{34} \\
%   \vb A^\dagger_{41} & \vb A^\dagger_{42} & \vb A^\dagger_{43} & \vb A^\dagger_{44}
%   \end{array}\right)
%%--------------------------------------------------------------------%
%   \left(\begin{array}{cccc}
%   \vb X_{11} & \vb X_{12} & \vb X_{13} & \vb X_{14} \\
%   \red{\vb X_{21}} & \red{\vb X_{22}} & \red{\vb X_{23}} & \red{\vb X_{24}} \\
%   \vb X_{31} & \vb X_{32} & \vb X_{33} & \vb X_{34} \\
%   \vb X_{41} & \vb X_{42} & \vb X_{43} & \vb X_{44}
%   \end{array}\right)
%%--------------------------------------------------------------------%
%   \left(\begin{array}{cccc}
%   \vb A_{11} & \vb A_{12} & \red{\vb A_{13}} & \vb A_{14} \\
%   \vb A_{21} & \vb A_{22} & \red{\vb A_{23}} & \vb A_{24} \\
%   \vb A_{31} & \vb A_{32} & \red{\vb A_{33}} & \vb A_{34} \\
%   \vb A_{41} & \vb A_{42} & \red{\vb A_{43}} & \vb A_{44}
%   \end{array}\right)
%$$
%%====================================================================%

%%%%%%%%%%%%%%%%%%%%%%%%%%%%%%%%%%%%%%%%%%%%%%%%%%%%%%%%%%%%%%%%%%%%%%
%%%%%%%%%%%%%%%%%%%%%%%%%%%%%%%%%%%%%%%%%%%%%%%%%%%%%%%%%%%%%%%%%%%%%%
%%%%%%%%%%%%%%%%%%%%%%%%%%%%%%%%%%%%%%%%%%%%%%%%%%%%%%%%%%%%%%%%%%%%%%
\newpage
\section{Comparison to the Kardar/Jaffe 
         $\mathbb{T}\mathbb{G}$ Formulas}

\paragraph{Notation}
In the seminal~\citeasnouns{Rahi2009, Kruger2012}, the 
formulas
are presented in a somewhat private language that elides the 
distinction between matrices of different dimensions. In particular,
although the quantities $\mathbb{T}_1$ and $\mathbb{T}_2$
in those formulas always correspond to matrices of dimension 
$N_1\times N_1$ and $N_2\times N_2$ (where $N_1$ and $N_2$ 
are the numbers of basis functions allocated to objects 1 and 2), 
the symbol ``1'' refers ambiguously to $N_1\times N_1$ or 
$N_2\times N_2$-dimensional unit matrices, while the symbol
$\mathbb{G}_0$ refers ambiguously to matrices of dimension
$N_1\times N_1$, $N_1\times N_2$, $N_2\times N_1$, or
$N_2\times N_2$. In what follows we attempt to dispel this 
confusion through the use of symbols such as $\vb 1_{N}$ 
for $N\times N$ unit matrices and $\vb G_{ij}$ for 
$N_i\times N_j$ matrix representations of the $\mathbb{G}$ operator.

We also assign shorthand notation to two matrices that appear 
frequently in these formulas:
%====================================================================%
\begin{align*}
\text{``}
 \text{Im }\mathbb{T}_i - 
           \mathbb{T}_i \text{Im}[\mathbb{G}_0]\mathbb{T}_i^*
 \text{''}
&\,\,\equiv\,\, \vb A_i 
 = \text{Im }\vb{T}_i - \vb{T}_i \text{Im}[\vb G_{ii}]\vb{T}_i^\dagger
\\
\text{``}
\frac{1}{1-\mathbb{G}_0 \mathbb{T}_i \mathbb{G}_0 \mathbb{T}_j}
\text{''}
&\,\,\equiv\,\, \vb B_{ij}
= \Big[ \mathbf{1}_{N_j}
         -\vb G_{ji} \vb{T}_i \vb{U}_{ij} \vb T_j
   \Big]^{-1}
\end{align*}
%====================================================================%
Matrices $\vb A_i$ and $\vb B_{ij}$ have dimensions $N_i\times N_i$ 
and $N_j \times N_j$ respectively.

Examples of the Kr\"uger/Kardar/Jaffe formulas that we will use
include
%====================================================================%
\begin{description}
 \item[1.] \textit{Equilibrium Casimir energy between two bodies:}
 \numeq{KJCasimirEnergy}
  {
   \mathcal{E}=\frac{\hbar c}{2\pi}\int
   \log\det\Big(\vb{1} - \vb G_{12}\vb T_2\vb G_{21}\vb T_1\Big)
   d\xi
  }
(where $\xi$ indicates imaginary frequency)
%====================================================================%
 \item[2.] \textit{Rate of heat radiation from a single body:}
%====================================================================%
 $$H=\frac{2}{\pi}
   \int \,\Theta(\omega) \mc{H}(\omega) d\omega
 $$
where $\Theta(\omega)\equiv \frac{\hbar\omega}{e^{\hbar\omega/kT} - 1}$
is the Bose-Einstein factor and $\mc H$ reads, in Kr\"uger language, 
%====================================================================%
$$
  \mc H=
   \text{Tr }\Big[  \text{Im} \big[ \mathbb{G}_0 \big] 
                    \text{Im} \big[ \mathbb{T}_1 \big]
                   -\text{Im} \big[ \mathbb{G}_0 \big]
                    \mathbb{T}_i
                    \text{Im} \big[ \mathbb{G}_0 \big]
                    \mathbb{T}_i^\dagger
             \Big]
 $$
%====================================================================%
or, in our language,
%====================================================================%
$$ \mc H = \text{Tr} \Big[ \text{Im}\big[\vb G_{11}\big] \vb A_{1}\Big].$$
%====================================================================%
 \item[3.] \textit{Heat transfer from body 1 to body 2:}
 $$H_1^{(2)}=\frac{2}{\pi}
   \int \,\Theta(\omega) \mathcal{H}_1^{(2)}(\omega)\,d\omega
 $$
%====================================================================%
where, in Kr\"uger language,
%====================================================================%
\begin{align}
\mathcal{H}_1^{(2)}
&=\text{Im Tr }
   \bigg\{ (1 + \GG \TT_2) \frac{1}{1-\GG \TT_1 \GG \TT_2}
           \GG 
           \Big[ \im \TT_1 - \TT_1 \big(\im \GG\big) \TT_1^\dagger \Big] 
           \GG^\dagger
           \frac{1}{1-\TT_2^\dagger \GG^\dagger \TT_1^\dagger \GG^\dagger}\TT_2^\dagger
   \bigg\}
\nn
%--------------------------------------------------------------------%
\intertext{or, in our language,}
%--------------------------------------------------------------------%
\mathcal{H}_1^{(2)}
&=\text{Im Tr }
   \bigg\{ (\vb 1_{N_2} + \vb G_{22} \vb T_2) \vb B_{12}
           \vb G_{21} \vb A_{1} \vb G_{21}^\dagger
           \vb B_{12}^\dagger \vb T_2^\dagger
   \bigg\}
\end{align}
%====================================================================%
 \item[4.] \textit{ Momentum transfer (Nonequilibrium Casimir force) 
                    from body 1 to body 1 or body 1 to body 2} 
%====================================================================%
 \begin{align*}
   F_1^{(1)}&=\frac{2}{\pi}
   \int \,\Theta(\omega) \mathcal{F}_1^{(1)}(\omega)\,\frac{d\omega}{\omega}
\\
   F_1^{(2)}&=\frac{2}{\pi}
   \int \,\Theta(\omega) \mathcal{F}_1^{(2)}(\omega)\,\frac{d\omega}{\omega}
 \end{align*}
%====================================================================%
In Kr\"uger language,
%====================================================================%
\begin{align*}
 \mathcal{F}_1^{(1)}&=\text{Re Tr }
   \bigg\{ \nabla (1 + \GG \TT_2) 
           \frac{1}{1-\GG \TT_1 \GG \TT_2}
           \GG 
           \Big[ \im \TT_1 - \TT_1 \big(\im \GG\big) \TT_1^\dagger \Big] 
           \frac{1}{1-\GG_0^* \TT_2^* \GG_0^* \TT_1^*}
   \bigg\}
\\
 \mathcal{F}_1^{(2)}&=\text{Re Tr }
   \bigg\{ \nabla (1 + \GG \TT_2) 
           \frac{1}{1-\GG \TT_1 \GG \TT_2}
           \GG 
           \Big[ \im \TT_1 - \TT_1 \big(\im \GG\big) \TT_1^\dagger \Big] 
           \GG^\dagger
           \frac{1}{1-\TT_2^\dagger \GG^\dagger \TT_1^\dagger \GG}
           \TT_2^\dagger
   \bigg\}
\\
\end{align*}
%====================================================================%
\end{description}
%====================================================================%

\subsection*{Equilibrium Casimir force}

Equation (\ref{KJCasimirEnergy}) for the equilibrium Casimir energy 
is the two-body special case of the more general formula
%====================================================================%
\numeq{KJCasimirEnergy2}
{
   \mathcal{E}=\frac{\hbar c}{2\pi}\int
   \log\frac{\det \vb M(\xi)}{\det \vb M_0(\xi)} \, d\xi 
}
where $\vb M(\xi)$ is just the full VIE matrix and $\vb M_0$
is that matrix with the $\vb G$ blocks zeroed out, corresponding
to the case in which all objects are infinitely separated:
%====================================================================%
$$
\vb M(\xi)=
   \left(\begin{array}{cccc}
   \vbTInv_1  & \vb G_{12} & \cdots & \vb G_{1N} \\
   \vb G_{21} & \vbTInv_2  & \cdots & \vb G_{2N} \\
   \vdots     & \vdots     & \ddots & \vdots     \\
   \vb G_{N1} & \vb G_{N2} & \cdots & \vbTInv_{N} 
   \end{array}\right), 
\qquad
\vb M_0(\xi)=
   \left(\begin{array}{cccc}
   \vbTInv_1  & 0          & \cdots & 0          \\
   0          & \vbTInv_2  & \cdots & 0          \\
   \vdots     & \vdots     & \ddots & \vdots     \\
   0          & 0          & \cdots & \vbTInv_{N} 
   \end{array}\right).
$$
%====================================================================%
%====================================================================%
where the matrix is just the full VIE matrix 
Differentiating, we obtain a expression for the Casimir force on 
body $1$:
%====================================================================%
\numeq{KJCasimirForce}
{
   \mathcal{F}_1=\frac{\hbar c}{2\pi}\int
   \text{Tr}\Big\{ \vb M^{-1} 
            \frac{ \partial \vb M}{\partial \vb r_i} 
            \Big\} \, d\xi
}
%====================================================================%
where the derivative is taken with respect to infinitesimal
displacement of object 1 in the $i$th cartesian direction:
%====================================================================%
$$
\frac{ \partial \vb M}{\partial \vb r_i} 
 =
   \left(\begin{array}{cccc}
   0                     & \partial_i \vb G_{12} & \cdots & \partial_i \vb G_{1N} \\
   \partial_i \vb G_{21} & 0                     & \cdots & 0          \\
   \vdots                & \vdots                & \ddots & \vdots     \\
   \partial_i \vb G_{N1} & 0                     & \cdots & 0           
   \end{array}\right),
$$

%====================================================================%

%%%%%%%%%%%%%%%%%%%%%%%%%%%%%%%%%%%%%%%%%%%%%%%%%%%%%%%%%%%%%%%%%%%%%%
%%%%%%%%%%%%%%%%%%%%%%%%%%%%%%%%%%%%%%%%%%%%%%%%%%%%%%%%%%%%%%%%%%%%%%
%%%%%%%%%%%%%%%%%%%%%%%%%%%%%%%%%%%%%%%%%%%%%%%%%%%%%%%%%%%%%%%%%%%%%%
\addcontentsline{toc}{section}{\large Part II: Implementation}
\newpage
\section{SWG Basis Functions}

SWG basis functions are the three-dimensional analog of
RWG basis functions. They are defined on pairs of adjacent tetrahedra:
%====================================================================%
$$ \vb b_\alpha(\vb x) = 
  \begin{cases}
   \displaystyle{
    +\frac{A_\alpha}{3V_\alpha^+}(\vb x - \vb Q_\alpha^+), 
                }
    \qquad &\vb x\in \mc P_\alpha^+
\\[5pt]
   \displaystyle{
  -\frac{A\alpha}{3V_\alpha^-}(\vb x - \vb Q_\alpha^-), 
                } 
    \qquad &\vb x\in \mc P_\alpha^-
  \end{cases}
$$
%====================================================================%
where $\mc P_\alpha^\pm$ are the two tetrahedra associated with basis
function $\alpha$, $V_\alpha^\pm$ are their volumes, $\vb Q_\alpha\pm$
are the source/sink vertices, and $A_\alpha$ is the area of the 
triangular face shared by $\mc P_\alpha^\pm$. (I denote tetrahedra 
by $\mc P$, which stands for ``pyramid,'' to avoid confusion with
the symbol $\mc T$, which stands for ``triangle'' in my memos on
RWG basis functions.)

The divergence of the SWG basis function is 
%====================================================================%
$$ \nabla \cdot \vb b_\alpha(\vb x) = 
    \pm \frac{A_\alpha}{V_\alpha^\pm}, \qquad \vb x\in \mc P_\alpha^\pm.
$$


%%%%%%%%%%%%%%%%%%%%%%%%%%%%%%%%%%%%%%%%%%%%%%%%%%%%%%%%%%%%%%%%%%%%%%
%%%%%%%%%%%%%%%%%%%%%%%%%%%%%%%%%%%%%%%%%%%%%%%%%%%%%%%%%%%%%%%%%%%%%%
%%%%%%%%%%%%%%%%%%%%%%%%%%%%%%%%%%%%%%%%%%%%%%%%%%%%%%%%%%%%%%%%%%%%%%
\newpage
\section{SWG Matrix Elements of the $\mathbb{G}$ operator}

%=================================================
%=================================================
%=================================================
\subsection{Distant case: Volume-integral method}

The $\mathbb{G}$-matrix element between two SWG basis functions
is
%====================================================================%
\numeq{bGb}
{
 \exptwoB{\vb b_\alpha}{\mathbb G}{\vb b_\beta}
 =\int_{\sup \vb b_\alpha} \, d\vb x_\alpha \,
  \int_{\sup \vb b_\beta} \, d\vb x_\beta\,
   b_{\alpha i}(\vb x_\alpha)
   \mathbb G_{ij}(\vb R_0 + \overline{\vb x}_\alpha - \overline{\vb x}_\beta)
   b_{\beta j}(\vb x_\beta)
}
%====================================================================%
where 
%====================================================================%
$$ \overline{\vb x}_\alpha \equiv \vb x_\alpha - \vb x_{\alpha 0}, 
   \qquad 
   \overline{\vb x}_\beta \equiv \vb x_\beta - \vb x_{\beta 0}, 
   \qquad 
   \vb R_0 = \vb x_{\alpha 0}-\vb x_{\beta 0}
$$
%====================================================================%
and $\vb x_{\alpha 0}, \vb x_{\beta 0}$ are the centroids of
the basis functions.

When the two basis functions are well separated 
(i.e. $|\vb R_0| \gg |\overline{\vb x}_\alpha|, |\overline{\vb x}_\beta|$),
we may compute (\ref{bGb}) to sufficient accuracy using a volume-integral
method: 
%====================================================================%
\numeq{bGb}
{
 \exptwoB{\vb b_\alpha}{\mathbb G}{\vb b_\beta}
 =\sum \pm \int_{\mc P_\alpha^\pm } d\vb x_\alpha
           \int_{\mc P_\beta^\pm  } d\vb x_\beta
           \left[ \vb b_\alpha \cdot \vb b_\beta
                 -\frac{9}{k^2}
           \right] \Phi\Big( \big|\vb R_0 + \overline{\vb x}_\alpha
                                - \overline{\vb x}_\beta
                             \big|\Big)
}
%====================================================================%
where $\Phi(r) = \frac{e^{ik|\vb r|}}{4\pi |\vb r|}.$
in which the 6-dimensional integration over each of the
four pairs of tetrahedra is carried out by simple low-order
numerical cubature, as discussed in Appendix \ref{VolumeIntegralAppendix}.
 
%
%Expand $\mathbb G_{ij}$ about $\vb R_0$:
%%====================================================================%
%\numeq{TaylorExpansion}
%{ \mathbb{G}_{ij}(\vb R_0 + \overline{\vb x}_\alpha - \overline{\vb x}_\beta)
%  \,\approx\, \mathbb{G}_{ij}^0
%         +\overline{x}_{\alpha k} \mathbb{G}_{ijk}^0
%         -\overline{x}_{\beta  k} \mathbb{G}_{ijk}^0
%         +\cdots
%}
%%====================================================================%
%where $\mathbb{G}^0_{ij}\equiv \mathbb{G}_{ij}(\vb R_0)$ 
%and $\mathbb{G}_{ijk}^0\equiv \partial_k \mathbb{G}_{ij}(\vb R_0).$ 
%Equation (\ref{bGb}) then reads (with a summation convention in force)
%%====================================================================%
%\numeq{DQApprox}
%{
% \exptwoB{\vb b_\alpha}{\mathbb G}{\vb b_\beta}
%\approx 
%  \mc J_{\alpha i} \mathbb{G}^0_{ij} \mc J_{\beta j}
% +\mc Q_{\alpha ik} \mathbb{G}^0_{ijk} \mc J_{\beta j}
% -\mc J_{\alpha i} \mathbb{G}^0_{ijk} \mc Q_{\beta jk}
%}
%%====================================================================%
%Here we have introduced the first and second moments
%of the current distributions of the SWG functions:
%\begin{align*}
% \mc J_{\alpha i}
%\equiv \int_{\sup \vb b_\alpha} b_{\alpha i}(\vb x)\,d\vb x
%\\
% \mc Q_{\alpha ij}
%\equiv \int_{\sup \vb b_\alpha} b_{\alpha i}(\vb x) x_j\,d\vb x
%\end{align*}

%=================================================
%=================================================
%=================================================
\subsection{Nearby case: Surface-integral method}
When the basis functions are not well separated, the
dipole/quadrupole expansion method of the previous 
subsection is inaccurate. In this case it is convenient 
to follow the work of Bleszynski et al.~\cite{Bleszynski2013}
by reducing the volume integral (\ref{bGb}) to a nonsingular 
\textit{surface} integral:
%====================================================================%
\numeq{GSurfaceIntegral}
{
 \exptwoB{\vb b_\alpha}{\mathbb G}{\vb b_\beta}
=-\frac{1}{4\pi ik} 
  \int_{\partial \sup \vb b_\alpha} \, \!\!\!\!\! d\vb x_\alpha \,
  \int_{\partial \sup \vb b_\alpha} \, \!\!\!\!\! d\vb x_\beta
  \Big(\vbhat{n}_\alpha \cdot \vbhat{n}_\beta\Big)
  \Big[   T_1(\vb x_\alpha, \vb x_\beta)
        + T_2(\vb x_\alpha, \vb x_\beta)
  \Big]
}
%====================================================================%
with
\begin{align*} 
     T_1(\vb x_\alpha, \vb x_\beta) 
  &= \vb b_\alpha(\vb x_\alpha ) \cdot \vb b_\beta(\vb x_\beta)
     h(ik R)
\\
     T_2(\vb x_\alpha, \vb x_\beta) 
  &= -\frac{9A_\alpha A_\beta}{V_\alpha V_\beta k^2}
      \Big\{ 9h(ikR) + w(ikR) 
             +k^2\big[(\vb Q_\alpha - \vb Q_\beta)\cdot \vb R\big]\,p(ikR)
      \Big\}
\end{align*} 
where the scalar functions $h,w,p$ are 
%====================================================================%
\begin{align*}
  h(x) &= \frac{1}{x}\texttt{ExpRelBar}(x,2) \\
  w(x) &= \texttt{ExpRelBar}(x,3) \\
  p(x) &= \frac{1}{x^3}\Big[ x\texttt{ExpRelBar}(x,2) 
                            - \texttt{ExpRelBar}(x,3)\Big] - \frac{1}{3}
\\
       &= \frac{h(x)}{x} - \frac{w(x)}{x^3} - \frac{1}{3}.
\end{align*}
Here the function \texttt{ExpRelBar} is defined by 
\begin{align*}
 \texttt{ExpRelBar}(x,N) 
&= e^{x} - \sum_{n=0}^{N-1} \frac{x^{n}}{n!}
\\
&= \sum_{n=N}^\infty \frac{x^n}{n!}
\end{align*}
I call this \texttt{ExpRelBar} because it is similar to
a similar function known as \texttt{ExpRel} (the two
functions differ only in their normalization). 

Note that the function \texttt{ExpRelBar} satisfies the 
derivative relationship
\begin{align*}
 \frac{d}{dx} \texttt{ExpRelBar}(x,N)
&= \sum_{n=N}^\infty \frac{n x^{n-1}}{n!} \\
&= \sum_{n=N}^\infty \frac{x^{n-1}}{(n-1)!} \\
&= \sum_{n=N-1}^\infty \frac{x^{n}}{n!}
&= \texttt{ExpRelBar}(x,N-1).
\end{align*}

Series expansions:
%====================================================================%
\begin{align*}
  h(x)        &= \sum_{n=1}^\infty \frac{x^n}{(n+1)!}   
  h^\prime(x) &= \sum_{n=1}^\infty \frac{x^n}{(n+1)!}   
\end{align*}
%====================================================================%

%=================================================
%=================================================
%=================================================
\subsection{Derivatives with respect to translations}

For computation of non-equilibrium Casimir forces and torques,
we require the matrix elements of derivatives of the
$\mathbb{G}$ operator with respect to its first argument.

\subsubsection*{Distant case}
%====================================================================%
\numeq{bdiGb}
{
 \exptwoB{\vb b_\alpha}{\partial_i \mathbb G}{\vb b_\beta}
 =\sum \pm \int_{\mc P_\alpha^\pm } d\vb x_\alpha
           \int_{\mc P_\beta^\pm  } d\vb x_\beta
           \left[ \vb b_\alpha \cdot \vb b_\beta
                 -\frac{9}{k^2}
           \right] 
           r_i \Psi(|\vb r|)
}
%====================================================================%
where
%====================================================================%
$$ \vb r = \vb R_0 - \vb x_\alpha - \vb x_\beta, \qquad
   \Psi(r) = (ikr-1)\frac{e^{ikr}}{4\pi r^3}.
$$
%====================================================================%
For calculations of torques we need also matrix elements
of the derivative of $\mathbb{G}(\vb x_\alpha, \vb x_\beta)$ with respect
to rotation of $\vb x_\alpha$ about a coordinate axis pointing in the
direction $\vbhat{a}$ and centered at an origin $\vb x_0$.
Under rotation through an infinitesimal angle $\Delta \theta$
about this axis, the point $\vb x_\alpha$ goes to 
$\vb x_\alpha + \Delta \vb x_\alpha$ with 
%====================================================================%
$$\Delta \vb x_\alpha 
  = 
  \Big[\vbhat{a}\times (\vb x_\alpha-\vb x_0)\big] \, \Delta \theta
$$
%====================================================================%
and the angular derivative of $\mathbb{G}$ is thus
%====================================================================%
$$
 \partial_\theta\mathbb{G}
 =
 \varepsilon_{ijk} \vbhat{a}_j (\vb x_\alpha - \vb x_0)_k \partial_i \mathbb{G}
$$
%====================================================================%
with matrix elements
%====================================================================%
\begin{align*}
 \exptwoB{\vb b_\alpha}{\partial_\theta \mathbb G}{\vb b_\beta}
&=\sum \pm \int_{\mc P_\alpha^\pm } d\vb x_\alpha
           \int_{\mc P_\beta^\pm  } d\vb x_\beta
           \left[ \vb b_\alpha \cdot \vb b_\beta
                 -\frac{9}{k^2}
           \right]
           \varepsilon_{ijk} \vbhat{a}_j(\vb x_\alpha - \vb x_0)_k
           r_i \Psi(|\vb r|)
\\
&=\sum \pm \int_{\mc P_\alpha^\pm } d\vb x_\alpha
           \int_{\mc P_\beta^\pm  } d\vb x_\beta
           \left[ \vb b_\alpha \cdot \vb b_\beta
                 -\frac{9}{k^2}
           \right]
 e         \left\{ \vb r \cdot \Big[\vbhat{a} \times (\vb x_\alpha-\vb x_0)\Big]\right\}
           \Psi(|\vb r|).
\label{bdtGb}
\end{align*}
%====================================================================%
\subsubsection*{Nearby case}
 
For the nearby case we use the surface-integral formula
(\ref{GSurfaceIntegral}), but with the quantities $T_1, T_2$ 
replaced by their derivatives. These are 
%====================================================================%
\begin{align*}
 \partial_i T_1 
&= 
 ik \frac{r_i}{R} \vb b_\alpha \cdot \vb b_\beta h^\prime(ikR)
\\[5pt]
 \partial_i T_2 
&= -ik\frac{r_i}{R}\frac{9A_\alpha A_\beta }{V_\alpha V_\beta k^2}
    \Big[  9h^\prime(ikR) 
          + w^\prime(ikR) 
          + k^2\big[(\vb Q_\alpha - \vb Q_\beta)\cdot \vb R\big]p^\prime(ikR)
    \Big]
\\
&\qquad \qquad 
   -\frac{9A_\alpha A_\beta}{V_\alpha V_\beta}
    \big( Q_{\alpha i}- Q_{\beta i} \big) p(ikR)
\end{align*}
%====================================================================%
\begin{align*}
 h^\prime(ikR) \\
\end{align*}

%%%%%%%%%%%%%%%%%%%%%%%%%%%%%%%%%%%%%%%%%%%%%%%%%%%%%%%%%%%%%%%%%%%%%%
%%%%%%%%%%%%%%%%%%%%%%%%%%%%%%%%%%%%%%%%%%%%%%%%%%%%%%%%%%%%%%%%%%%%%%
%%%%%%%%%%%%%%%%%%%%%%%%%%%%%%%%%%%%%%%%%%%%%%%%%%%%%%%%%%%%%%%%%%%%%%
\newpage
\section{PFT symmetries}

%%%%%%%%%%%%%%%%%%%%%%%%%%%%%%%%%%%%%%%%%%%%%%%%%%%%%%%%%%%%%%%%%%%%%%
\begin{align*}
 P\sups{abs} 
   &= \frac{1}{2}\text{Re }\int \vb j^* \cdot \vb E\sups{tot} dV 
\\
   &= \underbrace{\frac{1}{2}\text{Re }\int \vb j^* \cdot \vb E\sups{scat} dV}
                _{P_1} 
     \,+\,
      \underbrace{\frac{1}{2}\text{Re }\int \vb j^* \cdot \vb E\sups{inc} dV}
                _{P_2} 
\\
P_1
  &= \frac{1}{2}\text{Re }\int \vb j^* \cdot \vb E\sups{scat} dV 
\\ 
  &= \frac{1}{2}\text{Re }\sum_{\alpha,\beta=1}^{N\subt{BF}}
     \Big[ j_\alpha^* j_\beta \cdot i\omega \mu_0 G_{\alpha\beta}\Big],
     \qquad
     G_{\alpha\beta}\equiv \exptwoB{\vb b_\alpha}{\vb G}{\vb b_\beta}
\\
  &= \frac{1}{2}\text{Re }\sum_{\alpha}
     \Big[ i\omega \mu_0 |j_\alpha|^2 G_{\alpha\alpha} \Big]
    +\frac{1}{2}\text{Re }\sum_{\beta > \alpha}
     \Big[ i\omega \mu_0
           \big( j_\alpha^* j_\beta + j_\alpha j^*_\beta\big)
           G_{\alpha\beta}
     \Big]
\intertext{(where I used $G_{\alpha\beta}=G_{\beta\alpha}$)}
  &= -\omega \mu_0
      \primedsum_{\beta \ge \alpha}
        \Big[ \text{Re }\big(j^*_\alpha j_\beta \big)
              \cdot \text{Im } G_{\alpha\beta}
        \Big]
\end{align*}
%%%%%%%%%%%%%%%%%%%%%%%%%%%%%%%%%%%%%%%%%%%%%%%%%%%%%%%%%%%%%%%%%%%%%%
where the primed sum means the $\beta=\alpha$ term is to 
be weighted with a factor of 1/2.

Next,
%%%%%%%%%%%%%%%%%%%%%%%%%%%%%%%%%%%%%%%%%%%%%%%%%%%%%%%%%%%%%%%%%%%%%%
\begin{align*}
 F_i
   &= \frac{1}{2\omega}\text{Im }\int \vb j^* \cdot \partial_i \vb E\sups{tot} dV 
\\
   &= \underbrace{\frac{1}{2\omega}\text{Im }\int \vb j^* \cdot \partial_i \vb E\sups{scat} dV}
                _{F_1} 
     \,+\,
      \underbrace{\frac{1}{2\omega}\text{Im }\int \vb j^* \cdot \partial_i \vb E\sups{inc} dV}
                _{F_2} 
\\
F_1
  &= \frac{1}{2\omega}\text{Im }\int \vb j^* \cdot \partial_i \vb E\sups{scat} dV 
\\ 
  &= \frac{1}{2\omega}\text{Im }\sum_{\alpha,\beta=1}^{N\subt{BF}}
     \Big[ j_\alpha^* j_\beta \cdot i\omega \mu_0 (\partial_i G)_{\alpha\beta}\Big],
     \qquad
     (\partial_i G)_{\alpha\beta}
    \equiv \exptwoB{\vb b_\alpha}{\partial_i \vb G}{\vb b_\beta}
\\
  &= \frac{1}{2\omega}\text{Im }\sum_{\beta>\alpha}
     \Big[ i\omega \mu_0 (j_\alpha^* j_\beta - j_\alpha j_\beta^*) 
           (\partial_i G)_{\alpha\beta}\Big],
     \qquad
\intertext{[where I used $(\partial_i G)_{\alpha\alpha}=0$ and $(\partial_i G)_{\alpha\beta}=-(\partial_i G)_{\beta\alpha}$]}
  &= -\mu_0 \sum_{\beta >\alpha}
        \Big[ \text{Im }\big(j^*_\alpha j_\beta \big)
              \cdot \text{Im } (\partial_i G)_{\alpha\beta}
        \Big]
\end{align*}
The expression for the torque is the similar.

%%%%%%%%%%%%%%%%%%%%%%%%%%%%%%%%%%%%%%%%%%%%%%%%%%%%%%%%%%%%%%%%%%%%%%
%%%%%%%%%%%%%%%%%%%%%%%%%%%%%%%%%%%%%%%%%%%%%%%%%%%%%%%%%%%%%%%%%%%%%%
%%%%%%%%%%%%%%%%%%%%%%%%%%%%%%%%%%%%%%%%%%%%%%%%%%%%%%%%%%%%%%%%%%%%%%
\newpage
\section{Fields of Individual SWG Basis Functions}

Once we have obtained the solution vector $\vb j$ to the 
linear system (\ref{VIESystem}), we can compute the fields
at arbitrary points in space according to 
%====================================================================%
$$ \left\{ \begin{array}{c} \vb E(\vb x) \\ 
                            \vb H(\vb x) 
           \end{array} 
   \right\}
   = 
   \left\{ \begin{array}{c} \vb E\sups{inc}(\vb x) \\ 
                            \vb H\sups{inc}(\vb x) 
           \end{array} 
   \right\}
  + 
  \sum_\alpha j_\alpha
   \left\{ \begin{array}{c} \mathbb{E}_\alpha(\vb x) \\ 
                            \mathbb{H}_\alpha(\vb x)
           \end{array} 
   \right\}
$$
%====================================================================%
where $\{\mathbb{E}_\alpha, \mathbb{H}_\alpha\}$ are the
electric and magnetic fields produced by the current distribution
of basis function $\vb b_\alpha$ populated with unit strength:
%====================================================================%
\begin{subequations}
\begin{align}
 \mathbb{E}_\alpha(\vb x)
&= ikZ_0 
   \int_{\sup \vb b_\alpha} \mathbb{G}(\vb x, \vb x_\alpha) \cdot \vb b_\alpha
   d\vb x_\alpha
\\
 \mathbb{H}_\alpha(\vb x)
&= 
\end{align}
\label{EHAlpha}
\end{subequations}
%====================================================================%

\subsection*{Distant case: Volume-integral technique}

When the two basis functions are well separated, 
we may compute (\ref{EHAlpha}) to sufficient accuracy using a volume-integral
method in which the 3-dimensional integration over each of the
two tetrahedra is carried out by simple low-order
numerical cubature, as discussed in Appendix \ref{VolumeIntegralAppendix}.

\subsection*{Nearby case: Surface-integral formulation}

The vector and scalar potentials due to a current distribution
$\vb J$ are 
%====================================================================%
\begin{align*}
 \vb A(\vb x) 
&= 
 \mu_0 \int_{\mc V} \vb J(\vb x^\prime) G_0(\vb r) dV
\\
 \Phi(\vb x)
&= 
 \frac{1}{\epsilon_0} \int_{\mc V} \rho(\vb x^\prime) G_0(\vb r) dV
\\
&= 
 \frac{1}{i\omega \epsilon_0} 
 \int_{\mc V} \Big[\nabla \cdot \vb J\Big] G_0(\vb r) dV
\end{align*}
%====================================================================%
where in the last line we use $\rho = \frac{1}{i\omega}\nabla \cdot \vb J$
and 
$$ G_0(\vb r) = \frac{e^{ik|\vb r|}}{4\pi |\vb r|}
   =\frac{1}{ik}\nabla^2 h(\vb r),
   \qquad 
   \vb 
   r=\vb x-\vb x^\prime.
$$
%====================================================================%
The $\vb E$-field is 
%====================================================================%
$$ \vb E = i\omega\vb A - \nabla \Phi.$$
%====================================================================%
Now putting $\vb J=\vb b_\alpha(\vb x)$ and using $\omega \mu_0 =k Z_0$
and $\omega\epsilon_0=k/Z_0$ we find, for the $\vb E$-field of a
single SWG function
%====================================================================%
\begin{align}
 \mathbb{E}_{\alpha}(\vb x)
&= ikZ_0\left[ \int \vb b_\alpha(\vb x^\prime) G_0(\vb r) d V
              +\frac{1}{k^2}\nabla 
              \int \Big[\nabla \cdot \vb b_\alpha\Big]
                    G_0(\vb r) d\vb x^\prime
        \right].
\label{EAlpha}
\end{align}
%====================================================================%
First integral:
%====================================================================%
\begin{align*}
\int_{\mc V} 
 \vb b_\alpha(\vb x^\prime) G_0(\vb r) dV
&=\sum \pm \frac{A}{3ik V} 
  \int_{\mc V} (\vb x^\prime - \vb Q) 
               \Big[\nabla_{\vb x^\prime}^2 h(\vb r)\Big]
  dV
\intertext{Use (\ref{GreensTheorem}a):}
&= \sum \pm \frac{A}{3ik V} 
    \int_{\partial \mc V} \Big[ (\vb x^\prime - \vb Q)
                       \big[\vbhat{n} \cdot \nabla_{\vb x^\prime} h(\vb r)\big]
                       - \vbhat{n} h(\vb r) 
                 \Big] \, dA
\intertext{Use $\nabla_{\vb x^\prime} h(\vb r) = +k^2 q(\vb r)\,\vb r$:}
&=\sum \pm \frac{A}{3ik V}
    \int_{\partial \mc V} 
    \Big[ (\vb x^\prime - \vb Q)
          k^2 (\vb{r} \cdot \vbhat{n}) q(\vb r)
                        - \vbhat{n} h(\vb r) 
    \Big] \, dA
\end{align*}
%====================================================================%
Second integral:
%====================================================================%
\begin{align*}
\int_{\mc V} \Big[ \nabla \cdot \vb b_\alpha \Big]
             G_0(\vb r) \, dV
&=\sum \pm \frac{A}{ik V} 
   \int_{\mc V} \nabla_{\vb x^\prime}^2 h(\vb r) \, dV
\\
&=\sum \pm \frac{A}{ik V} 
   \int_{\partial \mc V} 
   \nabla_{\vb x \prime} h(\vb r) \cdot \vbhat{n} 
   \, dA
\\
&=-ik\sum \pm \frac{A}{V}
   \int_{\partial \mc V} q(\vb r) (\vb r\cdot \vbhat{n})
   \, dA
\end{align*}
%====================================================================%
The quantity that enters (\ref{EAlpha}) is 
%====================================================================%
\begin{align*}
\frac{1}{k^2} \nabla 
\int_{\mc V} \Big[ \nabla \cdot \vb b_\alpha \Big]
             G_0(\vb r) \, dV
&=\sum \pm \frac{A}{ikV}
  \int_{\partial \mc V}
  \Big[ q(\vb r)\vbhat{n} - k^2 (\vb r\cdot \vbhat{n})t(\vb r) \vb r\Big]
  dA
\end{align*}
%====================================================================%
Putting it all together, the surface integral for the $\vb E$ field
of an individual SWG function is 
%====================================================================%
\begin{align}
 \mathbb{E}_{\alpha}(\vb x)
&=Z_0 \sum \pm \frac{A}{3V} \int_{\partial \mc V}
   \Bigg\{ k^2 (\vb r \cdot \vbhat{n}) q(\vb r) (\vb x^\prime - \vb Q)
          +\big[3q(\vb r) - h(\vb r)\big ]\vbhat{n} 
\nn
&\hspace{2in}
          -3k^2(\vb r \cdot \vbhat{n}) t(\vb r)\vb r
   \Bigg\}dA.
\end{align}
%====================================================================%


%%%%%%%%%%%%%%%%%%%%%%%%%%%%%%%%%%%%%%%%%%%%%%%%%%%%%%%%%%%%%%%%%%%%%%
%%%%%%%%%%%%%%%%%%%%%%%%%%%%%%%%%%%%%%%%%%%%%%%%%%%%%%%%%%%%%%%%%%%%%%
%%%%%%%%%%%%%%%%%%%%%%%%%%%%%%%%%%%%%%%%%%%%%%%%%%%%%%%%%%%%%%%%%%%%%%
\appendix 

%%%%%%%%%%%%%%%%%%%%%%%%%%%%%%%%%%%%%%%%%%%%%%%%%%%%%%%%%%%%%%%%%%%%%%
%%%%%%%%%%%%%%%%%%%%%%%%%%%%%%%%%%%%%%%%%%%%%%%%%%%%%%%%%%%%%%%%%%%%%%
%%%%%%%%%%%%%%%%%%%%%%%%%%%%%%%%%%%%%%%%%%%%%%%%%%%%%%%%%%%%%%%%%%%%%%
\newpage
\section{Dictionary of the operator--matrix correspondence}
\label{CorrespondenceAppendix}

Infinite-dimensional position-space basis 
$\Longleftrightarrow$ 
$N\subt{B}$-dimensional basis of discrete expansion functions
$\{\vb b_\alpha(\vb x)\}.$ 

\subsection*{Notation}

\begin{itemize}
  \item We use blackboard-bold symbols for position-space 
        operators: $\mathbb{G}, \mathbb{T}$.
  \item We use upper-case bold letters for 
        discrete-basis matrices: $\vb T, \vb G, \vb W.$
        The elements of these matrices are, e.g.
        %====================================================================%
        $$ G_{\alpha\beta} \equiv \int \int 
           \vb b_\alpha(\vb r) \mathbb{G}(\vb r, \vb r^\prime)
           \vb b_\beta(\vb r^\prime) \, d\vb r \, d\vb r^\prime
        $$
        %====================================================================%
  \item (At the risk of confusion with the previous item) 
        We use upper-case bold letters for position-space
        vectors: $\vb J(\vb x), \vb E(\vb x).$
  \item We use lower-case bold letters for
        discrete-basis vectors: $\vb j, \vb e.$ The elements
        of these vectors are e.g.
        %====================================================================%
        $$ e_{\alpha} 
           \equiv 
           \int \vb b_\alpha(\vb r) \cdot \vb E(\vb r) d\vb r.
        $$
        %====================================================================%
\end{itemize}

%====================================================================%
\renewcommand{\arraystretch}{1.5}
$$\begin{array}{|c|c|}\hline
   \textbf{Continuous} 
 & \textbf{Discrete} 
\\\hline
%--------------------------------------------------------------------%
  \mathbb{G}(\vb r, \vb r^\prime)
& 
  \vb G\text{ matrix, with elements }
  U_{\alpha\beta} 
  = \int \int \vb b_\alpha(\vb r) \mathbb{G}(\vb r, \vb r^\prime)
              \vb b_\beta(\vb r^\prime) \, d\vb r \, d\vb r^\prime
\\\hline
%--------------------------------------------------------------------%
  \vb{E}(\vb r)
&
  \vb e\text{ vector, with elements }
  e_{\alpha} 
  = \int \vb b_\alpha(\vb r) \cdot \vb {E}(\vb r) \, d\vb r
\\\hline
%--------------------------------------------------------------------%
\end{array}$$
%====================================================================%


\subsection*{Approximate completeness relation}

One way to conceptualize the transition from the continuous
to the discrete is to suppose that the basis functions 
$\{\vb b_\alpha(\vb x)\}$ satisfy an approximate completeness relation
of the form 
%====================================================================%
\begin{subequations}
\begin{align}
\sum_{\alpha} b_{\alpha i}(\vb r) b_{\alpha j}(\vb r^\prime)
&\approx \delta_{ij} \delta(\vb r-\vb r^\prime)
\intertext{or}
\sum_\alpha \int b_{\alpha i}(\vb r) b_{\alpha j}(\vb r^\prime) \, d\vb r^\prime
&=\delta_{ij}
\end{align}
\label{SWGCompleteness}
\end{subequations}
%====================================================================%

\subsection*{Convolutions becomes matrix-vector products}
For example, consider the continuous version of the equation
relating the total current to the total field:
%====================================================================%
\begin{align}
  \vb E &= ikZ_0 \mathbb{G} \star \vb J 
\intertext{or}
  E_i(\vb r) &= ikZ_0 \int \mathbb{G}_{ij}(\vb r, \vb r^\prime) 
                J_j(\vb r^\prime) 
                d\vb r^\prime.
\intertext{Insert (\ref{SWGCompleteness}a) [in the form 
           $\sum b_{\beta j}(\vb r^\prime) 
                 b_{\beta k}(\vb r^{\prime\prime})
                 =\delta_{jk}\delta(\vb r^\prime-\vb r^{\prime\prime})
           $]between $\mathbb{G}$ and $\vb J$ on the RHS:}
  E_i(\vb r) &= ikZ_0 \sum_{\beta }
                \left[
                \int \mathbb{G}_{ij}(\vb r, \vb r^\prime) 
                     b_{\beta j}(\vb r^\prime)
                     d\vb r^\prime
                \right]
                \underbrace{\left[
                \int b_{\beta k}(\vb r^{\prime\prime}) 
                     J_k(\vb r^{\prime\prime})
                     d\vb r^{\prime\prime}
                            \right]}_{\vb j_\beta}
\intertext{As it stands this equation exists in a sort of 
           hybrid continuous-discrete form. 
           Now multiply both sides by $\vb b_{\alpha}(\vb r)$ 
           and integrate over $\vb r$ to find}
 e_\alpha &= ikZ_0 G_{\alpha\beta} j_\beta
\intertext{or}
 \vb e &= ikZ_0 \vb G\, \vb j.
\end{align}
%====================================================================%

%=================================================
%=================================================
%=================================================
\subsection*{Volume integrals become dot products}

Consider, for example, the integral
%====================================================================%
\begin{align}
\Big\langle \vb J \cdot \vb E \Big\rangle_{\mc B_n}
 &\equiv
 \int_{\mc B_n} \vb J^*(\vb r) \cdot \vb E(\vb r) \, d\vb r
\nonumber
\intertext{Rewrite this in the seemingly pedantic form}
 &= 
 \int_{\mc B_n} \int_{\mc B_n} 
 J_i^*(\vb r) \delta_{ij} \delta(\vb r-\vb r^\prime)
                E_j(\vb r^\prime) \, d\vb r \, d\vb r^\prime
\nonumber
\intertext{Now insert (\ref{SWGCompleteness}a):}
 &= 
 \int_{\mc B_n} \int_{\mc B_n} 
  J_i^*(\vb r) 
  \left[ \sum_{\alpha} b_{\alpha i}(\vb r) b_{\alpha j}(\vb r^\prime)\right]
  E_j(\vb r^\prime) \, d\vb r \, d\vb r^\prime
\nn
 &= 
 \sum_{\alpha}
 \underbrace{\Big[
 \int_{\mc B_n} b_{\alpha i}(\vb r)J_i^*(\vb r)  \, d\vb r 
             \Big]
            }_{j_{n\alpha}^*}
 \underbrace{\Big[
 \int_{\mc B_n} b_{\alpha j}(\vb r^\prime) E_j(\vb r^\prime) \, d\vb r^\prime 
             \Big]
            }_{e_{n\alpha}}
 &= \vb j_n^\dagger \vb e_n.
\label{VolIntToDotProd}
\end{align}
%====================================================================%

%%%%%%%%%%%%%%%%%%%%%%%%%%%%%%%%%%%%%%%%%%%%%%%%%%%%%%%%%%%%%%%%%%%%%%
%%%%%%%%%%%%%%%%%%%%%%%%%%%%%%%%%%%%%%%%%%%%%%%%%%%%%%%%%%%%%%%%%%%%%%
%%%%%%%%%%%%%%%%%%%%%%%%%%%%%%%%%%%%%%%%%%%%%%%%%%%%%%%%%%%%%%%%%%%%%%
\newpage
\section{Derivation of volume integrals for the force and torque}
\label{ForceFormulaAppendix}

Consider a body in which exists both a (deterministic) 
total volume current distribution $\vb J(\vb x)$ and
electric and magnetic fields $\{\vb E, \vb H\}(\vb x).$
The time-average force experienced by the currents in 
an infinitesimal volume $dV$ is
%====================================================================%
\begin{align}
d\vb F &= \frac{1}{2}\text{Re }
 \Big[ \rho^* \vb E + \mu_0 \vb J^* \times \vb H \Big] \, dV
\nn
\intertext{Use $\rho=\frac{1}{i\omega}(\nabla \cdot \vb J)$
           and $\vb H=\frac{1}{i\omega\mu_0}\nabla \times \vb E$:}
 &=
 \frac{1}{2}\text{Re }\left\{
 \frac{1}{i\omega}
 \Big[ -(\nabla \cdot \vb J^*) \vb E 
       + \vb J^* \times (\nabla \times \vb E) 
 \Big] \,
                      \right\} dV
\nonumber
\intertext{or}
dF_i &= -\frac{1}{2\omega}\text{Im }
 \Big[
 (\partial_j J^*_j) E_i - 
  \underbrace{ \varepsilon_{ijk}
               \varepsilon_{k\ell m}
             }_{\delta_{i\ell}\delta_{jm} - \delta_{im}\delta_{j\ell}}
  J^*_j \partial_\ell E_m
 \Big] \, dV
\nn
&= -\frac{1}{2\omega}\text{Im }
 \Big[ (\partial_j J^*_j) E_i
       - J^*_j \partial_i E_j
       + J^*_j \partial_j E_i
 \Big] \, dV.
\label{dFi}
\intertext{The total force is given by integrating over the volume:}
F_i
&= -\frac{1}{2\omega}\text{Im } \int_{\mc B_n}
 \Big[ (\partial_j J^*_j) E_i
       - J^*_j \partial_i E_j
       + J^*_j \partial_j E_i
 \Big] dV
\label{Fi}
\end{align}
%====================================================================%
The first and third terms here together read
%====================================================================%
\numeq{Argument}
{
 \int \partial_j \big(J_j^* E_i\big ) dV 
 = \int \nabla \cdot (E_i \, \vb J^*) \, dV
 = \oint E_i \vb J^* \cdot d\vb A = 0
}
%====================================================================%
because $\vb J\cdot \vbhat{n}=0$ at the surface of the object 
(no current flows from the body into space). 
Thus only the middle term in (\ref{Fi}) is nonvanishing,
and we find simply
%====================================================================%
\numeq{ForceVolumeIntegral}
{ F_i = \frac{1}{2\omega}\text{Im } \int_{\mc B_n}
         J^*_j \partial_i E_j \, dV 
      = \frac{1}{2\omega}\text{Im } 
         \int_{\mc B_n} \vb J^* \cdot \partial_i \vb E \, dV.
}
%====================================================================%

\subsubsection*{Torque}

The contribution of currents in $dV$ to the \textit{torque} 
about an origin $\vb r_0$ is given by
%====================================================================%
\begin{align*} 
 d\bmc T &= (\vb r - \vb r_0) \times d \vb F
\intertext{or, in components,}
 d\mc T_i &= \varepsilon_{ijk} (\vb r - \vb r_0)_j d \vb F_k.
\intertext{Insert (\ref{dFi}):}
&= -\frac{1}{2\omega}\text{Im}
 \left\{
 \varepsilon_{ijk} (\vb r - \vb r_0)_j
 \Big[ (\partial_\ell J^*_\ell) E_k
       - J^*_\ell \partial_k E_\ell
       + J^*_\ell \partial_\ell E_k
 \Big]\right\} \, dV.
\end{align*} 
The first and third terms here integrate to zero by an 
argument similar to (\ref{Argument}), and we find
%====================================================================%
\numeq{TorqueVolumeIntegral}
{  \mc T_i 
 = \frac{1}{2\omega}\text{Im } \int_{\mc B_n}
   \epsilon_{ijk} (\vb r-\vb r_0)_j J_\ell^* \partial_k E_\ell
   \,dV
 = \frac{1}{2\omega}\text{Im } \int_{\mc B_n}
    \vb J^* \cdot \partial_\theta \vb E \, dV
}
where the symbol $\partial_\theta \vb E$ denotes the derivative
of $\vb E(\vb r)$ with respect to an infinitesimal rotation of the
point $\vb r$ about the $i$th coordinate axis with origin $\vb r_0.$

%%%%%%%%%%%%%%%%%%%%%%%%%%%%%%%%%%%%%%%%%%%%%%%%%%%%%%%%%%%%%%%%%%%%%% 
%%%%%%%%%%%%%%%%%%%%%%%%%%%%%%%%%%%%%%%%%%%%%%%%%%%%%%%%%%%%%%%%%%%%%%
%%%%%%%%%%%%%%%%%%%%%%%%%%%%%%%%%%%%%%%%%%%%%%%%%%%%%%%%%%%%%%%%%%%%%%
\newpage
\section{From Rytov to Johnson-Nyquist}
\label{RytovToJohnson}

For those of us who learned about noise in resistors
before learning about Casimir forces and radiative heat transfer
in nanoparticles,
it's useful to relate the abstract and possibly mysterious 
notion of the Rytov correlation function to the concrete and 
familiar concept of Johnson-Nyquist noise.
(Even for those who need no help with fluctuation-dissipation 
ideas, this exercise is useful for pinning down factors of $2\pi$ and
other normalization effluvia.)

\subsection*{Johnson-Nyquist Noise}

In elementary circuit theory we are taught that, at temperature
$T$, a resistor exhibits a mean-square power of 
%====================================================================%
$$ \big\langle P \big\rangle = 4 kT \Delta f$$
%====================================================================%
(where $\Delta f$ is the effective measurement bandwidth in Hertz,
usually determined by low-pass and high-pass filters in the circuit).
If the resistance of the resistor is $R$, then the mean-square voltage 
across its terminals and the mean-square current flowing through it are
%====================================================================%
\begin{align}
 \big \langle V^2 \big\rangle 
 &= \big\langle P \big \rangle R = 4k T R\, \Delta f, \qquad 
\nn
 \big \langle I^2 \big\rangle 
 &= \frac{1}{R} \big\langle P \big \rangle 
  = \frac{4k T}{R} \Delta f
\label{MeanSquareCurrent}
\end{align}
%====================================================================%
We would now like to understand equation (\ref{MeanSquareCurrent})
on the basis of the Rytov correlation function.

%=================================================
%=================================================
%=================================================
\subsection*{Macroscopic current noise from microscopic current-density 
             fluctuations}

To this end, consider a resistor consisting of a homogeneous cylinder of
length $L$ and cross-sectional area $A$ with relative dielectric
function
%====================================================================%
\numeq{EpsilonSigma}
{
 \epsilon(\omega)
 \quad=\quad
 \epsilon^\prime(\omega) + i\epsilon^{\prime\prime}(\omega)
 \quad=\quad
 \epsilon^\prime(\omega) + i\frac{\sigma}{\epsilon_0 \omega}
}
%====================================================================%
where $\sigma$ is the microscopic conductivity in units of
mho$\cdot$meters [one mho = 1 inverse ohm  (1 $\Omega^{-1}$) = 
1 siemen].\footnote{The \textit{absolute} permittivity
of the object is $\epsilon_0\epsilon^\prime + i\frac{\sigma}{\omega}$. 
To check that the imaginary part of (\ref{EpsilonSigma}) is indeed
dimensionless, note that $\epsilon_0=\frac{1}{Z_0 c}$ where
$Z_0\approx 377\, \Omega$ is the impedance of free space
and $c$ is the speed of light; thus the units of the imaginary
part of (\ref{EpsilonSigma}) are 
$$ \Big[\frac{\sigma}{\epsilon_0 \omega}\Big] 
   =\frac{[\text{mhos$\cdot$meters}]}
         {[\text{mhos$\cdot$seconds$\cdot$meters}][\text{seconds}^{-1}]}
   =\text{dimensionless.}
$$}
The microscopic resistivity, with units of ohms/meter, is
$\rho=1/\sigma$. The total resistance of the resistor is
%====================================================================%
$$ R=\frac{L}{A}\rho = \frac{L}{A\sigma}.$$
%====================================================================%
We align the cylinder axis with the $z$-axis and break up 
coordinates into transverse and longitudinal components according to
$\vb x=(\vbrho, z)$.

The instantaneous current density at a point in the cylinder
may be expressed as a Fourier synthesis:
%====================================================================%
\begin{align}
 \vb J(\vb x, t) 
&= \int \vb J(\vb x,\omega) e^{-i\omega t} \, d\omega
\label{FourierSynthesis}
\intertext{where $\vb J(\vb x,\omega)$ is the Fourier transform of 
           the instantaneous current density $\vb J(\vb x,t)$:}
 \vb J(\vb x, \omega) 
&= \frac{1}{2\pi} \int \vb J(\vb x,t) e^{+i\omega t} \, dt.
\label{FourierAnalysis}
\end{align}
%====================================================================%
The instantaneous \textit{current} passing through a cross-sectional
plane at height $z$ is 
%====================================================================%
$$ I(z,t)=\int I(z,\omega) e^{-i\omega t} \, d\omega$$
%====================================================================%
where $I(z,\omega)$ is given by integrating $\vb J$ over the cross-sectional
plane:
%====================================================================%
\begin{align*}
 I(z,\omega)&=\int_A \vb J(\vbrho, z,\omega)\cdot \vbhat{n} \, d\vbrho
\intertext{or, in our specific geometry in which the cross section is
everywhere normal to the $z$ direction,}
 I(z,\omega)&=\int_A J_z(\vbrho, z,\omega) \, d\vbrho.
\end{align*}
%====================================================================%
Over a time interval of length $\tau$, the average of the product
of $I(t,z)$ and $I(t,z^\prime)$ is
%====================================================================%
\begin{align} 
\big\langle I(z)I(z^\prime)\big\rangle
&=\frac{1}{\tau}\int_0^\tau I(z,t) I(z^\prime, t) \, dt
\nn
&=\frac{1}{\tau}\int_0^\tau \, dt \,
  \int \, d\omega \, 
  \int \, d\omega^\prime \, 
  I(z,\omega) I(z,\omega^\prime)
  e^{-i(\omega + \omega^\prime)t}
\nn
&=\frac{1}{\tau}\int_0^\tau \, dt \,
  \int \, d\omega \,
  \int \, d\omega^\prime \,
  \int_A \, d\vbrho \,
  \int_A \, d\vbrho^\prime \,
  J_z(\vbrho,z,\omega) J_z(\vbrho^\prime,z^\prime, \omega^\prime)
  e^{-i(\omega + \omega^\prime)t}.
\label{IZZP}
\end{align}
%====================================================================%
To proceed we need to address a question about which we haven't said
anything thus far---namely, where the current \textit{comes from}. 
In a deterministic circuit problem
we would have some fixed, known, externally applied voltage $V$
across the resistor, which would induce a current equal to $I=V/R$.
Here, on the other hand, there is no external voltage, and instead
the current arises from thermal and quantum-mechanical 
\textit{fluctuations} in the microscopic current density.
Although we can't track the instantaneous progress of these
fluctuations in time---and thus, for example, we can't write
down an expression for the instantaneous current density 
$\vb J(\vb x, t)$---we can make precise statements about
certain statistical \textit{averages} over these fluctuating
quantities. One particularly obvious statement is that the 
time-average value of any Cartesian component of $\vb J$ vanishes, 
%====================================================================%
$$\Big\langle J_i(\vb x,t)\Big\rangle=0 \qquad \text{for any } i.$$
%====================================================================%
A less obvious but even more important statement is that
the fluctuation-dissipation theorem allows us to make a
very definite prediction about the time-average value of the
\textit{product} of two cartesian components of $\vb J$. 
This equation---sometimes known as the \textit{Rytov correlation
function}---is easiest to write in the frequency domain, where
it takes the form
%====================================================================%
\begin{align}
 \Big\langle  J_i(\vb x, \omega) J_j(\vb x^\prime, \omega^\prime) \Big\rangle
&= \delta(\omega+\omega^\prime)
   \Big\langle J_i(\vb x), J_j(\vb x^\prime) \Big\rangle_\omega
\label{Rytov1}
\\[5pt]
 \Big\langle  J_i(\vb x) J_k(\vb x^\prime) \Big\rangle_{\omega}
&= \frac{2 \omega \epsilon_0}{\pi} \Theta(\omega, T) \delta(\vb r-\vb r^\prime)
   \text{Im }\epsilon_{ij}(\vb r, \omega)
\label{Rytov2}
\end{align}
%====================================================================%
Equation (\ref{Rytov1}) here is essentially the same\footnote{
My formula differs from that of Landau and Lifshitz (LL) by a 
factor of $2\pi$, which arises because of our different conventions
for the Fourier analysis and synthesis of time-domain functions:
I like to put a factor of $\frac{1}{2\pi}$ in Fourier-analysis 
equations like (\ref{FourierAnalysis}), and to omit this factor 
in Fourier-synthesis equations like (\ref{FourierSynthesis}), while
LL make the opposite choice.} as equation 
(122.4) in Landau and Lifshitz (LL), \textit{Statistical Physics Volume 1}.
In equation (\ref{Rytov2}), $\Theta(\omega, T)$ is the Bose-Einstein 
statistical factor at the local temperature near $\vb r$ 
(about which we will have more to say shortly)
and $\epsilon_{ij}(\vb r, \omega)$ is the
$(i,j)$ component of the relative permittivity tensor of the material
at point $\vb r$ and frequency $\omega$.
Inserting (\ref{Rytov1}) into (\ref{IZZP}) and using
the $\delta$ functions to perform the $\vbrho^\prime$ and $\omega^\prime$
integrations, we have 
%%%%%%%%%%%%%%%%%%%%%%%%%%%%%%%%%%%%%%%%%%%%%%%%%%%%%%%%%%%%%%%%%%%%%%
\begin{align*}
&\hspace{-0.1in}
\big\langle I(z)I(z^\prime)\big\rangle\\
&=\frac{1}{\tau}\int_0^\tau \, dt \,
  \int \, d\omega \,
  \int \, d\omega^\prime \,
  \int_A \, d\vbrho \,
  \int_A \, d\vbrho^\prime \,
  \Big\langle
  J_z(\vbrho,z,\omega) J_z(\vbrho^\prime,z^\prime, \omega^\prime)
  \Big\rangle
  e^{-i(\omega + \omega^\prime)t}
\\
&=\frac{2\epsilon_0 }{\pi \tau}\delta(z-z^\prime)
  \int_0^\tau \, dt \,
  \int \, d\omega \, \omega \, \Theta(\omega, T)
  \int_A \, d\vbrho \,
  \text{Im }\epsilon_{zz}(\vbrho, z, \omega)
\intertext{
In the present case [cf. equation (\ref{EpsilonSigma})] we have
$\text{Im }\epsilon_{ij}(\vb r,\omega)=\delta_{ij} \frac{\sigma}{\epsilon_0 \omega}$
(independent of $\vb r$), whereupon we find}
&=\frac{2\sigma }{\pi}\cdot \delta(z-z^\prime) 
  \cdot \underbrace{\frac{1}{\tau} \int_0^\tau \, dt \, \int\, d\omega \,
                    \Theta(\omega, T) 
                   }_{kT \Delta \omega}
  \underbrace{\int_A \, d\vbrho \,}_{A}
\end{align*}
%%%%%%%%%%%%%%%%%%%%%%%%%%%%%%%%%%%%%%%%%%%%%%%%%%%%%%%%%%%%%%%%%%%%%%
Here I used the high-temperature limit $\Theta(\omega, T)\approx kT$ 
(see below), Finally, averaging over the length of the resistor yields
\begin{align*}
 \big\langle I^2(z) \big\rangle
&= \frac{1}{L} \int_0^L \big\langle I(z)I(z^\prime)\big\rangle \, dz^\prime 
\\
&=\frac{2kT\Delta \omega}{\pi} \cdot \underbrace{\frac{A\sigma}{L}}_{1/R}
\\
&= \frac{4kT}{R}\Delta f
\end{align*}
where I used $\Delta \omega=2\pi\Delta f.$ This is equation 
(\ref{MeanSquareCurrent}).
%%%%%%%%%%%%%%%%%%%%%%%%%%%%%%%%%%%%%%%%%%%%%%%%%%%%%%%%%%%%%%%%%%%%%%

%=================================================
%=================================================
%=================================================
\subsection*{Limiting behavior of $\Theta(\omega, T)$}

The Bose-Einstein statistical factor $\Theta(\omega , T)$, 
which describes the average energy contained in an 
electromagnetic\footnote{Or otherwise bosonic.} 
mode of frequency $\omega$, is 
%%%%%%%%%%%%%%%%%%%%%%%%%%%%%%%%%%%%%%%%%%%%%%%%%%%%%%%%%%%%%%%%%%%%%%
\numeq{ThetaDef}
{
 \Theta(\omega, t)
 =\hbar \omega
        \left[ \frac{1}{e^{\frac{\hbar\omega}{kT}} - 1} + \frac{1}{2}\right]
=\frac{\hbar \omega}{2} \coth\pf{\hbar \omega}{2k T}
}
%%%%%%%%%%%%%%%%%%%%%%%%%%%%%%%%%%%%%%%%%%%%%%%%%%%%%%%%%%%%%%%%%%%%%%
In the high- and low-temperature limits (equivalently, the low- and 
high-frequency) limits, the statistical factor of equation (\ref{ThetaDef}) 
becomes
%%%%%%%%%%%%%%%%%%%%%%%%%%%%%%%%%%%%%%%%%%%%%%%%%%%%%%%%%%%%%%%%%%%%%%
\numeq{ThetaLimits}
{
  \Theta(\omega, t)\xrightarrow{\frac{\hbar\omega}{kT}\to 0} kT, 
  \qquad
  \Theta(\omega, t)\xrightarrow{\frac{\hbar\omega}{kT}\to \infty} \frac{\hbar \omega}{2}.
}
%%%%%%%%%%%%%%%%%%%%%%%%%%%%%%%%%%%%%%%%%%%%%%%%%%%%%%%%%%%%%%%%%%%%%%
The first case here corresponds to classical equipartition of energy: 
we have roughly $kT$ worth of energy in each mode, independent of 
frequency. The second case corresponds to quantum-mechanical zero-point
energy; we have some energy in each mode even at zero temperature.

To estimate the crossover between the high- and low-temperature
regimes, recall that room temperature ($T$=300 K) corresponds to 
an energy of $kT \approx $ 26 meV (milli-electron-volts), while 
$\hbar$ has the numerical value
%%%%%%%%%%%%%%%%%%%%%%%%%%%%%%%%%%%%%%%%%%%%%%%%%%%%%%%%%%%%%%%%%%%%%%
$$ \hbar 
   \approx 7\cdot 10^{-16} \text{eV}\cdot \text{s}
  = 7\cdot{10^{-16}} \frac{\text{eV}}{\text{rad/s}}.
$$
Thus, for a circuit at frequency $f=1$ GHz at a temperature of $T$=300 K,
we have 
%%%%%%%%%%%%%%%%%%%%%%%%%%%%%%%%%%%%%%%%%%%%%%%%%%%%%%%%%%%%%%%%%%%%%%
$$ kT = 0.026 \text{ eV } \quad \ll \quad \hbar\omega 
      = \left( 7\cdot 10^{-16} \frac{\text{eV}}{\text{rad/s}} \right)
        \left( 2\pi \cdot 10^9 \text{rad/s} \right)
      = 4 \cdot 10^{-6}\text{ eV}
$$
and thus, for ordinary circuits at ordinary temperatures, we are 
well in the high-temperature (low-frequency) regime in which
$\Theta(\omega,T)\approx kT$.

%%%%%%%%%%%%%%%%%%%%%%%%%%%%%%%%%%%%%%%%%%%%%%%%%%%%%%%%%%%%%%%%%%%%%%
%%%%%%%%%%%%%%%%%%%%%%%%%%%%%%%%%%%%%%%%%%%%%%%%%%%%%%%%%%%%%%%%%%%%%%
%%%%%%%%%%%%%%%%%%%%%%%%%%%%%%%%%%%%%%%%%%%%%%%%%%%%%%%%%%%%%%%%%%%%%%
\newpage
\section{Volume integrals involving SWG basis functions}
\label{VolumeIntegralAppendix}

Integrals over the support of SWG basis functions take the form
\numeq{SWGVolumeIntegral}
{
 \int_{\sup \vb b_\alpha} \mathcal{I}(\vb x, \vb b_\alpha) \, d\vb x
= 
  \int_{\mc P_\alpha^+} \mathcal{I}(\vb x, \vb b_\alpha) \, d\vb x
 +\int_{\mc P_\alpha^-} \mathcal{I}(\vb x, \vb b_\alpha) \, d\vb x
}
%%%%%%%%%%%%%%%%%%%%%%%%%%%%%%%%%%%%%%%%%%%%%%%%%%%%%%%%%%%%%%%%%%%%%%
Consider a tetrahedron with vertices 
$\{\vb Q, \vb V_1, \vb V_2, \vb V_3.\}$ 
A general
integral over this region takes the form 
%====================================================================%
\begin{align*}
 \int_{\mc P} \mathcal{I}(\vb x, \vb b ) \, d\vb x
&=\mc{J} \int_0^1 \, du \int_0^{1-u} \, dv \int_0^{1-u-v} dw \,
  \mathcal{I}\Big( \vb x(u,v,w), \vb b(u,v,w) \Big)
\end{align*}
%====================================================================%
%====================================================================%
\begin{align*}
 \vb x(u,v,w) 
&= 
 \vb Q + u\vb L_1 + v\vb L_2 + w\vb L_3 
\\
 \vb b(u,v,w) 
&= 
 \pm \frac{A}{3V}\bigg\{ u \vb L_1 
                        +v \vb L_2 
                        +w \vb L_3 
                 \bigg\}
\end{align*}
%====================================================================%
where
%====================================================================%
$$ \vb L_i \equiv \vb V_i - \vb Q_i $$
%====================================================================%
and the Jacobian of the transformation is 
%====================================================================%
$$\mc J=\frac{d(x,y,z)}{d(u,v,w)}
   =\det\left|\begin{array}{ccc} 
     \\
     \vb L_1 & \vb L_2 & \vb L_3 \\
     \\
     \end{array}\right|
  =6V
$$

%=================================================
%=================================================
%=================================================
\subsection*{Overlap Matrix Elements} 

\begin{align*} 
  \inpB{\vb b_\alpha}{\vb b_\beta}
&=\sum \pm \frac{A_\alpha A_\beta}{9V^2}
  \int_{\mc V} 
  \Big(\vb x-\vb Q_\alpha\Big)
  \cdot
  \Big(\vb x-\vb Q_\beta\Big)
  d\vb x
\intertext{(where the sum is over the 0, 1, or 2 tetrahedra in the 
            common support of $\{\vb b_\alpha,\vb b_\beta\}$)}
&=\sum \pm \frac{2A_\alpha A_\beta}{3V}
  \int_0^1 \, du \int_0^{1-u} \, dv \, \int_0^{1-u-v}\, dw\,
  \bigg\{ 
  \Big[u\vb L_{1\alpha} + v\vb L_{2\alpha} + w\vb L_{3\alpha}\Big]
\\
&\hspace{2in}
  \cdot \Big[u\vb L_{1\alpha} + v\vb L_{2\alpha} + w\vb L_{3\alpha} 
            + \vb Q_\alpha - \vb Q_\beta\Big]
  \bigg\}
\\
&=\sum \pm \frac{2A_\alpha A_\beta}{3V}
  \bigg\{ \frac{1}{120}
          \Big|\vb L_{1\alpha} + \vb L_{2\alpha} + \vb L_{3\alpha}
          \Big|^2
         +\frac{1}{120}\Big( |\vb L_{1\alpha}|^2 
                            +|\vb L_{2\alpha}|^2 
                            +|\vb L_{3\alpha}|^2 \Big)
\\
&\hspace{2in}
         +\frac{1}{24}
         \Big(\vb L_{1\alpha} + \vb L_{2\alpha} + \vb L_{3\alpha}\Big)
          \cdot (\vb Q_\alpha - \vb Q_\beta)
  \bigg\}
\intertext{This can be simplified by noting that 
           $\vb L_{1\alpha} + \vb L_{2\alpha} + \vb L_{3\alpha}
           =3(\vb X_{0\alpha} - \vb Q_\alpha)$
           where $\vb X_{0\alpha}$ is the centroid of 
           basis function $\vb b_\alpha$:}
&=\sum \pm \frac{2A_\alpha A_\beta}{3V}
  \Big[ \frac{3}{40}(\vb X_{0\alpha}-\vb Q_\alpha)^2 
        +\frac{1}{120}\Big( |\vb L_{1\alpha}|^2 
                           +|\vb L_{2\alpha}|^2 
                           +|\vb L_{3\alpha}|^2 \Big)
\\
&\hspace{1.5in}
         +\frac{1}{8}(\vb X_{0\alpha} - \vb Q_\alpha) 
               \cdot (\vb Q_\alpha-\vb Q_\beta)
  \Big].
\end{align*}

%=================================================
%=================================================
%=================================================
\subsection*{Dipole and Quadrupole Moments}

The dipole moment of the current distribution described by
a single unit-strength SWG basis function is 

\numeq{pAlpha}
{
   \vb p_\alpha 
   \equiv 
   \frac{i}{\omega} 
   \underbrace{\int_{\sup \vb b_\alpha} \vb b_\alpha(\vb x) \, d\vb x}
             _{\bmc J_\alpha(\vb x)}
}
%====================================================================%
The Cartesian components of $\bmc J_\alpha(\vb x)$ may be worked out 
in closed form:
%====================================================================%
\begin{align}
 \mc J_{\alpha i}(\vb x)
&=2A\int_0^1 \, du \, \int_0^{1-u} \, dv \, \int_0^{1-u-v}  \, dw \,
    (u+v+w) (\vb Q^- - \vb Q^+)_i
\nn
&=\frac{A}{4} (\vb Q^- - \vb Q^+)_i.
\label{mcJAlpha}
\end{align}
%====================================================================%
Similarly, the quadrupole moments are related to the quantity
\begin{align*}
 \mc Q_{\alpha ij}(\vb x)
&= \int_{\sup \vb b_\alpha} b_{\alpha i} (\vb x-\vb x_0)_j\, d\vb x
\intertext{where $\vb x_0=\frac{1}{3}(\vb V_1 + \vb V_2 + \vb V_3)$
           is the centroid of the basis function (which we take to 
           be the centroid of the triangle that constitutes the 
           common face).}
&=
2A\int_0^1 \, du \, \int_0^{1-u} \, dv \, \int_0^{1-u-v}  \, dw \,
 \bigg\{
\\
&\qquad\qquad 
 \hphantom{-}\,
  \bigg[ u\vb L_1^+ + v\vb L_2^+ + w\vb L_3+ \bigg]_i
  \left[ \left(u-\frac{1}{3}\right)\vb L_1^+
        +\left(v-\frac{1}{3}\right)\vb L_1^+
        +\left(w-\frac{1}{3}\right)\vb L_1^+
  \right]_j
\\
&\qquad\qquad 
 -\bigg[ u\vb L_1^- + v\vb L_2^- + w\vb L_3- \bigg]_i
  \left[ \left(u-\frac{1}{3}\right)\vb L_1^-
        +\left(v-\frac{1}{3}\right)\vb L_2^-
        +\left(w-\frac{1}{3}\right)\vb L_3^-
  \right]_j
 \bigg\}
\\
&=\frac{A}{20}
  \bigg\{ 
          Q^-_i \Big[ \vb Q^- - \vb x_0 \Big]_j
         -Q^+_i \Big[ \vb Q^+ - \vb x_0 \Big]_j
         +x_{0i}\Big[ \vb Q^+ - \vb x_0 \Big]_j
  \bigg\}
\end{align*}

%%%%%%%%%%%%%%%%%%%%%%%%%%%%%%%%%%%%%%%%%%%%%%%%%%%%%%%%%%%%%%%%%%%%%%
%%%%%%%%%%%%%%%%%%%%%%%%%%%%%%%%%%%%%%%%%%%%%%%%%%%%%%%%%%%%%%%%%%%%%%
%%%%%%%%%%%%%%%%%%%%%%%%%%%%%%%%%%%%%%%%%%%%%%%%%%%%%%%%%%%%%%%%%%%%%%
\section{Derivation of Equation (\ref{GSurfaceIntegral})}

Begin by introducing the function
%====================================================================%
\begin{align}
 h(r) &\equiv \frac{e^{ikr} - 1 - ikr}{4\pi ikr}
\intertext{with gradient}
 \nabla_{\vb r} h 
 &= \frac{1+e^{ikr}(-1+ikr)}{4ik\pi r^3} \vb {r}
\intertext{and Laplacian}
 \nabla_{\vb r}^2 h &= 
 ik \cdot \frac{e^{ikr}}{4\pi r}.
\label{DelDelh}
\end{align}
It is convenient to write the gradient of $h$ in the form
%====================================================================%
\begin{align*}
 \nabla_{\vb r} h 
 = -k^2 q(r) \, \vb r, \, 
   \qquad q(r)&\equiv\frac{1+e^{ikr}(-1+ikr)}{4\pi(ikr)^3}.
\\
              &\equiv\frac{e^{ikr}}{4\pi(ikr)^3}\cdot \texttt{ExpRelBar}(-ikr,2)
\end{align*}
and to write the gradient of $q$ in the form
%====================================================================%
\begin{align}
 \nabla_{\vb r} q
 = -k^2 t(r) \, \vb r, \,
   \qquad t(r)&\equiv\frac{-3+e^{ikr}(3-3ikr+(ikr)^2)}{4\pi(ikr)^5}.
\\
              &\equiv-\frac{e^{ikr}}{4\pi(ikr)^5}
               \Big\{     \texttt{ExpRelBar}(-ikr,2) 
                       + 2\texttt{ExpRelBar}(-ikr,3)
               \Big\}.
\end{align}
%====================================================================%
Also useful is
$$ \frac{e^{ik|\vb r|}}{4\pi|\vb r|}
   =\frac{1}{4\pi i k} \nabla_{\vb r} \cdot \vb F(\vb r),
   \qquad
   \vb F(\vb r)= \frac{e^{ikr}(ikr-1) + 1}{r^2} \vb r.
$$

%====================================================================%
We will also need Green's theorem in the forms 
%====================================================================%
\begin{subequations}
\begin{align}
\int_{\mc V} \Big[ \phi\nabla^2 \psi - \psi\nabla^2 \phi\Big] dV
&=
\int_{\partial \mc V} \left(\phi \nabla \psi - \psi\nabla\phi\right)
     \cdot \, d\vb A
\intertext{and}
\int_{\mc V} (\nabla \cdot \vb a) \, dV 
&=
\int_{\partial \mc V} \vb a \cdot \, d\vb A 
\end{align}
\label{GreensTheorem}
\end{subequations}
%====================================================================%
The matrix element of the $\mathbb{G}$ operator between
SWG basis functions is 
%====================================================================%
\begin{align*}
  \exptwoB{\vb b_\alpha}{\mathbb{G}}{\vb b_\beta}
&=
 \int_{\mc V_\alpha}\,d\vb x_\alpha\,
 \int_{\mc V_\beta}\,d\vb x_\beta\,
 \Big[\vb b_\alpha \cdot \vb b_\beta 
          -\frac{ (\nabla \cdot \vb b_\alpha)
                  (\nabla \cdot \vb b_\beta)
                }{k^2}
 \Big] \frac{e^{ik|\vb r|}}{4\pi |\vb r|}
\\
%--------------------------------------------------------------------%
&=-\pf{A_\alpha A_\beta}{9V_\alpha V_\beta k^2} \mathcal{I}_1
  -\pf{A_\alpha A_\beta}{V_\alpha V_\beta k^4} \mathcal{I}_2
\intertext{where}
%--------------------------------------------------------------------%
\mathcal{I}_1
&=
 \int_{\mc V_\alpha}\,d\vb x_\alpha\,
 \int_{\mc V_\beta}\,d\vb x_\beta\,
     P(\vb x_\alpha, \vb x_\beta) 
     \nabla^2_\alpha h(\vb r)
\\
%--------------------------------------------------------------------%
\mathcal{I}_2
&=
 \int_{\mc V_\alpha}\,d\vb x_\alpha\,
 \int_{\mc V_\beta}\,d\vb x_\beta\,
 \nabla_\alpha \nabla_\beta h(\vb r)
\end{align*}
with

$$P(\vb x_\alpha, \vb x_\beta) = 
   (\vb x_\alpha-\vb Q_\alpha)\cdot (\vb x_\beta-\vb Q_\beta),
   \qquad
   \nabla_\alpha \equiv \nabla_{\vb x_\alpha},  
   \qquad
   \nabla_\beta \equiv \nabla_{\vb x_\beta}.
$$
%====================================================================%
First transform the integral in $\mathcal{I}_1$
by using (\ref{GreensTheorem}a) with $\phi=P$, $\psi=h$
and noting that $\nabla_\alpha^2 P=0:$ 
%====================================================================%
\begin{align}
\mathcal{I}_1 
&=
 \int_{\mc V_\beta} \, d\vb x_\beta\,
 \left\{ 
   \int_{\mc V_\alpha}
         P(\vb x_\alpha, \vb x_\beta)
         \nabla_\alpha^2 h(\vb r)
        \, d\vb x_\alpha
 \right\}
\nn
&=
 \int_{\mc V_\beta} \, d\vb x_\beta\,
 \left\{ 
   \int_{\partial \mc V_\alpha}
         \Big[
         P(\vb x_\alpha, \vb x_\beta) \nabla_\alpha h
         - (\vb x_\beta-\vb Q_\beta)h
         \Big] \cdot \vbhat{n_\alpha} \,d\vb x_\alpha
 \right\}
\nn
&=-
 \int_{\mc V_\beta} \, d\vb x_\beta\,
 \left\{ 
   \int_{\partial \mc V_\alpha}
         \Big[
         P(\vb x_\alpha, \vb x_\beta) \nabla_\beta h
         + (\vb x_\beta-\vb Q_\beta)h
         \Big] \cdot \vbhat{n_\alpha} \,d\vb x_\alpha
 \right\}
\intertext{Use $\nabla_\beta\Big[Ph\Big] = P\nabla_\beta h
               +h(\vb x_\alpha-\vb Q_\alpha):$}
&=-
 \int_{\mc V_\beta} \, d\vb x_\beta\,
 \left\{ 
   \int_{\partial \mc V_\alpha}
         \Big[
         \nabla_\beta \Big[Ph\Big] 
         - (\vb x_\alpha-\vb Q_\alpha) h
         + (\vb x_\beta-\vb Q_\beta)h
         \Big] \cdot \vbhat{n_\alpha} \,d\vb x_\alpha
 \right\}
\\
&=-
 \int_{\mc V_\beta} \, d\vb x_\beta\,
 \left\{ 
   \int_{\partial \mc V_\alpha}
         \Big[
         \nabla_\beta \Big[Ph\Big] 
         + (  \vb x_\beta - \vb x_\alpha 
            - \vb Q_\beta - \vb Q_\alpha) h
         \Big] \cdot \vbhat{n_\alpha} \,d\vb x_\alpha
 \right\}
\end{align}
Now use (\ref{GreensTheorem}b) twice to 
transform the second integral in $\mathcal{I}_2$:
\begin{align*}
\mc I_2=
 \int_{\mc V_\alpha} \, d\vb x_\alpha\,
 \nabla_\alpha \cdot 
 \left\{ \int_{\mc V_\beta}  \, d\vb x_\beta\,
         \nabla_\beta h(\vb r)
 \right\}
&=
 \int_{\mc V_\alpha} \, d\vb x_\alpha\,
 \nabla_\alpha \cdot 
 \left\{ \int_{\partial \mc V_\beta} 
         h(\vb r)\vbhat{n}_\beta\, d\vb x_\beta
 \right\}
\\
&=
 \int_{\partial \mc V_\beta} d\vb x_\beta
 \vbhat{n}_\beta \cdot 
 \left\{
 \int_{\mc V_\alpha} 
 \nabla_\alpha h(\vb r)
 \, d\vb x_\alpha\,
 \right\}     
\\
&=
 \int_{\partial \mc V_\alpha} d\vb x_\alpha
 \int_{\partial \mc V_\beta} d\vb x_\beta
 \Big[\vbhat{n}_\alpha \cdot \vbhat{n}_\beta\Big] h(\vb r).
\end{align*}

%%%%%%%%%%%%%%%%%%%%%%%%%%%%%%%%%%%%%%%%%%%%%%%%%%%%%%%%%%%%%%%%%%%%%%
%%%%%%%%%%%%%%%%%%%%%%%%%%%%%%%%%%%%%%%%%%%%%%%%%%%%%%%%%%%%%%%%%%%%%%
%%%%%%%%%%%%%%%%%%%%%%%%%%%%%%%%%%%%%%%%%%%%%%%%%%%%%%%%%%%%%%%%%%%%%%
\section{Diagonal Elements of the $\mathbb{G}$ Matrix}

The diagonal element of the $\mathbb{G}$ matrix is 
%====================================================================%
\numeq{DiagonalElement}
{
   \exptwob{\vb b_\alpha}{\mathbb{G}}{\vb b_\alpha}
= \frac{1}{ikZ_0}\int \vb J^*(\vb x) \cdot \vb E(\vb x) \, d\vb x
}
%====================================================================%
where $\vb J(\vb x)=\vb b_\alpha(\vb x)$ is the current distribution
of a single unit-strength SWG function (note we have $\vb J^*=\vb J$ 
as the SWG basis is real-valued) and
%====================================================================%
$$ \vb E(\vb x) =
   ikZ_0 \int \mathbb{G}(\vb x, \vb x^\prime) 
   \vb b_\alpha(\vb x^\prime) d\vb x^\prime
$$ 
%====================================================================%
is the electric field produced by this same current distribution.

Using Poynting's theorem [see, for example, Jackson equation (6.134)],
we may equate the real part of the RHS of equation 
(\ref{DiagonalElement}) to the negative of the power radiated 
to infinity by the localized current distribution $\vb J(\vb x)$:
%====================================================================%
\begin{align}
 -\frac{1}{2}\text{Re }\int \vb J^* \cdot \vb E \, dV 
&= \text{Re }\oint \vb S \cdot d\vb A 
\nn
&\approx \frac{ c^2 Z_0 k^4}{12\pi} |\vb p|^2
\label{PoyntingTheorem}
\end{align}
%====================================================================%
where on the RHS we have inserted the expression for the total power 
radiated by a point dipole; we expect the approximation to improve as 
the size of the basis function shrinks (so that its current distribution
more closely resembles that of a point dipole). Using equations 
(\ref{pAlpha}), (\ref{DiagonalElement}), and (\ref{PoyntingTheorem})
then allows us to derive the following expected relationship
between the diagonal $\mathbb{G}$-element and the first 
spatial moment of the SWG basis function:
%====================================================================%
\begin{align*}
 \text{Im } \exptwob{\vb b_\alpha}{\mathbb{G}}{\vb b_\alpha}
= \frac{k}{6\pi} \Big|\bmc J_\alpha\Big|^2.
\end{align*} 

%====================================================================%
%====================================================================%
%====================================================================%
\newpage
\section{Bleszynski method}

The Bleszynski method for integrating a scalar function $\Psi_V$
over a tetrahedron-product domain is based on the integral identity
%====================================================================%
\numeq{BleszynskiMaster}
{ \int_{\mc P} \int_{\mc P^\prime}
    \Psi_V(\vb x, \vb x^\prime) \, dV^\prime \, dV
  = \oint_{\partial \mc P}
    \oint_{\partial \mc P^\prime}
    \Psi_A(\vb x, \vb x^\prime) \vbhat{n}\cdot \vbhat{n^\prime}
    \, dA \, dA^\prime\,
}
%====================================================================%
where 
%====================================================================%
\numeq{PsiVPsiA}
{ \nabla_{\vb x} \nabla_{\vb x^\prime} \Psi_A(\vb x-\vb x^\prime)
   =\Psi_V(\vb x, \vb x^\prime).
}
%====================================================================%
For SWG problems we need to apply (\ref{BleszynskiMaster}) to
two different families of $\Psi_V$ functions, namely functions of the
form
%====================================================================%
\begin{subequations}
\begin{align}
 \Psi_{V1}(\vb x-\vb x^\prime) 
 &= 
 g_{V1}(r)
\\
 \Psi_{V2}(\vb x-\vb x^\prime) 
 &= 
 (\vb x-\vb Q)\cdot (\vb x^\prime-\vb Q^\prime) g_{V2}(r)
\end{align}
\label{PsiVForms}
\end{subequations}
%====================================================================%
where $g(r)$ is a scalar kernel function of the quantity
$r=|\vb x-\vb x^\prime|$.
%====================================================================%
To satisfy (\ref{PsiVPsiA}) for these types of $\Psi_V$ functions,
I will need to consider three different families of functions 
$\Psi_A(\vb x-\vb x^\prime)$:
%====================================================================%
\begin{align*}
 \Psi_{A1}(\vb x-\vb x^\prime)
 &\equiv(\vb x-\vb Q)\cdot (\vb x^\prime-\vb Q^\prime) h_{A1}(r)
\\[5pt]
 \Psi_{A2}(\vb x-\vb x^\prime)
 &\equiv(\vb x-\vb x^\prime)\cdot (\vb Q-\vb Q^\prime) h_{A2}(r)
\\[5pt]
 \Psi_{A3}(\vb x-\vb x^\prime)
 &\equiv h_{A3}(r)
\end{align*}
%====================================================================%
where, in each case, $h(r)$ is a scalar function of $r$.

Differentiating these three classes of $\Psi_A$ functions, one finds
%====================================================================%
\begin{align*}
 \nabla_{\vb x} \nabla_{\vb x^\prime}
 \Psi_{A1}(\vb x-\vb x^\prime)
&=  3h_{A1}(r) + rh_{A1}^\prime(r) 
     -  \Big(\vb r \cdot \Delta \vb Q \Big)
       \left[\frac{1}{r} h_{A1}^\prime(r)\right]
\\
&\qquad
     - \big(\vb x-\vb Q\big)\cdot\big(\vb x^\prime - \vb Q^\prime\big)
       \left[ \frac{2}{r}h_{A1}^\prime(r) + h_{A1}^{\prime\prime}(r)\right]
\\
%--------------------------------------------------------------------%
 \nabla_{\vb x} \nabla_{\vb x^\prime}
 \Psi_{A2}(\vb x-\vb x^\prime)
&=-(\vb r \cdot \Delta \vb Q)  
   \left[ \frac{4}{r}h_{A2}^\prime(r)
          +h_{A2}^{\prime\prime}(r) 
   \right]
\\
%--------------------------------------------------------------------%
 \nabla_{\vb x} \nabla_{\vb x^\prime}
 \Psi_{A3}(\vb x-\vb x^\prime)
&= -\frac{2}{r}h_{A3}^\prime(r) - h_{A3}^{\prime\prime}(r)
\end{align*}
%====================================================================%
Summing, I have
%====================================================================%
\begin{align*}
 \nabla_{\vb x} \nabla_{\vb x^\prime}
\Big( \Psi_{A1} + \Psi_{A2} + \Psi_{A3} \Big)
&=\big(\vb x-\vb Q\big) \cdot \big(\vb x^\prime-\vb Q^\prime\big)
  \bigg[ - \frac{2}{r}h^\prime_{A1}
         -h^{\prime\prime}_{A1} 
  \bigg]
\\
&\qquad - \vb r \cdot \Delta \vb Q
          \bigg[\frac{1}{r} h^\prime_{A1} + \frac{4}{r}h^\prime_{A2}
               +h^{\prime\prime}_{A2}\bigg]
\\
&\qquad
 + 3h_{A1} + r h_{A1}^\prime
          - \frac{2}{r}h^\prime_{A3} - h^{\prime\prime}_{A3}.
\end{align*}
%====================================================================%
To satisfy equation (\ref{PsiVPsiA}) for a $\Psi_V$ function
of the form (\ref{PsiVForms}a), it suffices to take $h_{A1}=h_{A2}=0$
and to choose $h_{A3}$ to satisfy the differential equation
%====================================================================%
$$ - \frac{2}{r}h^\prime_{A3}(r) - h^{\prime\prime}_{A3}(r) = g_{V1}(r).$$
%====================================================================%
Examples of solutions include
%====================================================================%
$$\begin{array}{lclclcl}
 \displaystyle{ g_{V1}(r) }
 &=& 
 \displaystyle{ r^p  }
 \qquad &\Longrightarrow& \qquad
 \displaystyle{ h_{A3}(r) }
 &=& 
 \displaystyle{-\frac{1}{(p+2)(p+3)} r^{p+2}}
\\[12pt]
%--------------------------------------------------------------------%
 \displaystyle{ g_{V1}(r) }
 &=& 
 \displaystyle{ \frac{e^{ikr}}{4\pi r} }
 \qquad &\Longrightarrow& \qquad
 \displaystyle{ h_{A3}(r) }
 &=& 
 \displaystyle{-\frac{\texttt{ExpRel}(ikr,2)}{4\pi ikr}}
\end{array}
$$
%====================================================================%
On the other hand, to satisfy equation (\ref{PsiVPsiA}) for a $\Psi_V$ 
function of the form (\ref{PsiVForms}b), I must satisfy the simultaneous
equations
%====================================================================%
\begin{align*}
  - \frac{2}{r}h^\prime_{A1} -h^{\prime\prime}_{A1} 
&= g_{V2}(r) 
\\
 \frac{1}{r} h^\prime_{A1} + \frac{4}{r}h^\prime_{A2}
 +h^{\prime\prime}_{A2}
&=0
\\
 +3h_{A1} + r h_{A1}^\prime
          - \frac{2}{r}h^\prime_{A3} - h^{\prime\prime}_{A3}
&=0.
\end{align*}
%====================================================================%
Examples of solutions include
%====================================================================%
%====================================================================%
$$
 g_{V2}(r) 
 = 
 \displaystyle{ r^p  }
 \qquad \Longrightarrow \qquad
 \left\{
 \begin{array}{ccl}
 \displaystyle{ h_{A1}(r) }
 &=& 
 \displaystyle{-\frac{1}{(p+2)(p+3)} r^{p+2}}
\\[8pt]
 \displaystyle{ h_{A2}(r) }
 &=& 
 \displaystyle{+\frac{1}{(p+2)(p+3)(p+5)} r^{p+2}}
\\[8pt]
 \displaystyle{ h_{A3}(r) }
 &=& 
 \displaystyle{-\frac{1}{(p+2)(p+3)(p+4)} r^{p+4}}
\\ 
 \end{array}
 \right.
\\[12pt]
$$
%====================================================================%
and
%====================================================================%
$$
 g_{V2}(r) 
 = 
 \displaystyle{ \frac{e^{ikr}}{4\pi r}  }
 \qquad \Longrightarrow \qquad
 \left\{
 \begin{array}{ccl}
 \displaystyle{ h_{A1}(r) }
 &=& 
 \displaystyle{-\frac{\texttt{ExpRel}(ikr,2)}{4\pi (ik)^2 r}}
\\[8pt]
 \displaystyle{ h_{A2}(r) }
 &=& 
 \displaystyle{+\frac{\texttt{ExpRel}(ikr,3)}{4\pi (ik)^3 r^2}
               -\frac{\texttt{ExpRel}(ikr,4)}{4\pi (ik)^4 r^3}
              }
\\[8pt]
 \displaystyle{ h_{A3}(r) }
 &=& 
 \displaystyle{-\frac{\texttt{ExpRel}(ikr,3)}{4\pi (ik)^3}}
\\ 
 \end{array}
 \right.
\\[12pt]
$$

%====================================================================%
%====================================================================%
%====================================================================%
\newpage
\bibliographystyle{ieeetr}
\bibliography{buff-em}

\end{document}
