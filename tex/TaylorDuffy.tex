\documentclass[letterpaper]{article}

\input{physcmds}
\usepackage{cite}

\graphicspath{{figures/}}

%------------------------------------------------------------
%------------------------------------------------------------
%- Special commands for this document -----------------------
%------------------------------------------------------------
%------------------------------------------------------------
\newcommand{\vbXi}{\boldsymbol{\xi}}
\newcommand{\vbEta}{\boldsymbol{\eta}}

%------------------------------------------------------------
%------------------------------------------------------------
%- Document header  -----------------------------------------
%------------------------------------------------------------
%------------------------------------------------------------
\title{Taylor-Duffy Method for Tetrahedra}
\author {Homer Reid}
\date {December 29, 2014}

%------------------------------------------------------------
%------------------------------------------------------------
%- Start of actual document
%------------------------------------------------------------
%------------------------------------------------------------

\begin{document}
\pagestyle{myheadings}
\markright{Homer Reid: Taylor-Duffy Method for Tetrahedra}
\maketitle

\tableofcontents

%%%%%%%%%%%%%%%%%%%%%%%%%%%%%%%%%%%%%%%%%%%%%%%%%%%%%%%%%%%%%%%%%%%%%%
%%%%%%%%%%%%%%%%%%%%%%%%%%%%%%%%%%%%%%%%%%%%%%%%%%%%%%%%%%%%%%%%%%%%%%
%%%%%%%%%%%%%%%%%%%%%%%%%%%%%%%%%%%%%%%%%%%%%%%%%%%%%%%%%%%%%%%%%%%%%%
Let $\mc P, \mc P^\prime$ be tetrahedra with vertices
%%%%%%%%%%%%%%%%%%%%%%%%%%%%%%%%%%%%%%%%%%%%%%%%%%%%%%%%%%%%%%%%%%%%%%
$$ \mc P=\{\vb V_1, \vb V_2, \vb V_3, \vb V_4\}, \qquad 
   \mc P^\prime=\{\vb V_1, \vb V_2^\prime, \vb V_3^\prime, \vb V_4^\prime\}.
$$
%%%%%%%%%%%%%%%%%%%%%%%%%%%%%%%%%%%%%%%%%%%%%%%%%%%%%%%%%%%%%%%%%%%%%%
(Note that $\mc P, \mc P^\prime$ have at least one common vertex.)
I consider six-dimensional integrals over the product 
domain $\mc P\times \mc P^\prime$ of the form
%%%%%%%%%%%%%%%%%%%%%%%%%%%%%%%%%%%%%%%%%%%%%%%%%%%%%%%%%%%%%%%%%%%%%%
\numeq{OriginalIntegral}
{
 \mathcal{I}
=\int_{\mc P} d\vb x \, \int_{\mc P^\prime} d\vb x^\prime \, 
  P(\vb x, \vb x^\prime) K(|\vb x-\vb x^\prime|)
}
%%%%%%%%%%%%%%%%%%%%%%%%%%%%%%%%%%%%%%%%%%%%%%%%%%%%%%%%%%%%%%%%%%%%%%
where $P$ is a polynomial in the cartesian components of 
$\vb x, \vb x^\prime$ and $K(r)$ is a scalar kernel function that
may be singular at $r=0$. A convenient parameterization of the 
domain of integration is 
%%%%%%%%%%%%%%%%%%%%%%%%%%%%%%%%%%%%%%%%%%%%%%%%%%%%%%%%%%%%%%%%%%%%%%
\begin{align*}
 \vb x &= V_1 + \xi_1 \vb A
              + \xi_2 \vb B
              + \xi_3 \vb C
\\
 \vb x^\prime &= V_1 + \eta_1 \vb A^\prime
                     + \eta_2 \vb B^\prime
                     + \eta_3 \vb C^\prime
\end{align*}
%%%%%%%%%%%%%%%%%%%%%%%%%%%%%%%%%%%%%%%%%%%%%%%%%%%%%%%%%%%%%%%%%%%%%%
\begin{align*}
 \vb A&=(\vb V_2-\vb V_1), \qquad 
 \vb B=(\vb V_3-\vb V_2),  \qquad
 \vb C=(\vb V_4-\vb V_3)
\\
 \vb A^\prime&=(\vb V_2^\prime-\vb V_1),       \qquad 
 \vb B^\prime=(\vb V_3^\prime-\vb V_2^\prime), \qquad
 \vb C^\prime=(\vb V_4^\prime-\vb V_3^\prime)
\end{align*}
%%%%%%%%%%%%%%%%%%%%%%%%%%%%%%%%%%%%%%%%%%%%%%%%%%%%%%%%%%%%%%%%%%%%%%
$$
      0 \le \xi_1, \eta_1 \le 1,
\quad 0 \le \xi_2 \le \xi_1,
\quad 0 \le \xi_3 \le \xi_2,
\quad 0 \le \eta_2 \le \eta_1,
\quad 0 \le \eta_3 \le \eta_2.
$$
The integral (\ref{OriginalIntegral}) becomes
%%%%%%%%%%%%%%%%%%%%%%%%%%%%%%%%%%%%%%%%%%%%%%%%%%%%%%%%%%%%%%%%%%%%%%
\numeq{OriginalIntegral2}
{
 \mathcal{I} =
\int_0^1 d\xi_1 \, \int_0^{\xi_1} d\xi_2 \, \int_0^{\xi_2} d\xi_3 \,
\int_0^1 d\eta_1 \, \int_0^{\eta_1} d\eta_2 \, \int_0^{\eta_2} d\eta_3 \,
P(\vbXi, \vbEta) K(\vbXi, \vbEta)
}
%%%%%%%%%%%%%%%%%%%%%%%%%%%%%%%%%%%%%%%%%%%%%%%%%%%%%%%%%%%%%%%%%%%%%%
Following~\cite{TaylorDuffy}, we now change variables from 
$\{\vbXi, \vbEta\}$ to $\{\vbXi, \vb u\}$, where $\vb u$ are 
the relative coordinates
%%%%%%%%%%%%%%%%%%%%%%%%%%%%%%%%%%%%%%%%%%%%%%%%%%%%%%%%%%%%%%%%%%%%%%
$$ u_i=\eta_i - \xi_i, $$
%%%%%%%%%%%%%%%%%%%%%%%%%%%%%%%%%%%%%%%%%%%%%%%%%%%%%%%%%%%%%%%%%%%%%%
and decompose the domain of integration in (\ref{OriginalIntegral2})
into $D$ subdomains with the following properties:
\textbf{(a)} In the $d$th subdomain, the range of the $\{u_i\}$ 
coordinates is a tetrahedron $\mc T_d^{\vb u}$ with one vertex at 
the origin, while  
\textbf{(a)} In the $d$th subdomain, the range of the $\{\xi_i\}$
coordinates is a $\vb u$-dependent tetrahedron 
$\mc T_d^{\vbXi}(\vb u)$:
%%%%%%%%%%%%%%%%%%%%%%%%%%%%%%%%%%%%%%%%%%%%%%%%%%%%%%%%%%%%%%%%%%%%%%
\numeq{TDIntegral1}
{
\mathcal{I}=\sum_{d=1}^D
 \int_{\mc T^{\vb u}_d} d\vb u
 \int_{\mc T^{\vbXi}_d(\vb u)} \, d\vbXi\,
 P(\vbXi, \vbEta + \vb u)
 K(\vbXi, \vbEta + \vb u)
}
%%%%%%%%%%%%%%%%%%%%%%%%%%%%%%%%%%%%%%%%%%%%%%%%%%%%%%%%%%%%%%%%%%%%%%
%%%%%%%%%%%%%%%%%%%%%%%%%%%%%%%%%%%%%%%%%%%%%%%%%%%%%%%%%%%%%%%%%%%%%%
\begin{align*}
 \int_{\mc T^{\vbXi}_d(\vb u)} \, d\vbXi\, F(\vbXi)
&=\int_{A_d}^{B_d}         d\xi_1 \,
  \int_{C_d}^{\xi_1 + D_d} d\xi_2 \,
  \int_{E_d}^{\xi_2 + F_d} d\xi_3 \, F(\xi_1, \xi_2, \xi_3)
\intertext{Change variables to 
           $\xi_1^\prime = \xi_1 - A_d,
           \xi_2^\prime = \xi_2 - C_d,
           \xi_3^\prime = \xi_2 - E_d$:}
&=\int_{0}^{B_d-A_d}         d\xi_1^\prime \,
  \int_{0}^{\xi_1^\prime + A_d + D_d - C_d} d\xi_2^\prime \,
  \int_{0}^{\xi_2^\prime + C_d + F_d - E_d } d\xi_3^\prime \,
   F(\xi_1^\prime + A_d, \xi_2^\prime + C_d, \xi_3^\prime + E_d)
\\
&=\int_{0}^{B_d-A_d}      d\xi_1^\prime \,
  \int_{0}^{\xi_1^\prime} d\xi_2^\prime \,
  \int_{0}^{\xi_2^\prime} d\xi_3^\prime \,
   F(\xi_1^\prime + A_d, \xi_2^\prime + C_d, \xi_3^\prime + E_d)
\intertext{Here we have used the property that 
           $A_d+D_d-C_D=C_d + F_d - E_d = 0$ for all subregions $d$.}
\end{align*}
%%%%%%%%%%%%%%%%%%%%%%%%%%%%%%%%%%%%%%%%%%%%%%%%%%%%%%%%%%%%%%%%%%%%%%

\end{document}
