\documentclass[letterpaper]{article}

\input{physcmds}
\usepackage{cite}

\graphicspath{{figures/}}

%------------------------------------------------------------
%------------------------------------------------------------
%- Special commands for this document -----------------------
%------------------------------------------------------------
%------------------------------------------------------------
\newcommand{\vbXi}{\boldsymbol{\xi}}
\newcommand{\vbEta}{\boldsymbol{\eta}}

%------------------------------------------------------------
%------------------------------------------------------------
%- Document header  -----------------------------------------
%------------------------------------------------------------
%------------------------------------------------------------
\title{Taylor-Duffy Method for Tetrahedra}
\author {Homer Reid}
\date {December 29, 2014}

%------------------------------------------------------------
%------------------------------------------------------------
%- Start of actual document
%------------------------------------------------------------
%------------------------------------------------------------

\begin{document}
\pagestyle{myheadings}
\markright{Homer Reid: Taylor-Duffy Method for Tetrahedra}
\maketitle

\tableofcontents

%%%%%%%%%%%%%%%%%%%%%%%%%%%%%%%%%%%%%%%%%%%%%%%%%%%%%%%%%%%%%%%%%%%%%%
%%%%%%%%%%%%%%%%%%%%%%%%%%%%%%%%%%%%%%%%%%%%%%%%%%%%%%%%%%%%%%%%%%%%%%
%%%%%%%%%%%%%%%%%%%%%%%%%%%%%%%%%%%%%%%%%%%%%%%%%%%%%%%%%%%%%%%%%%%%%%
Let $\mc P, \mc P^\prime$ be tetrahedra with vertices
%%%%%%%%%%%%%%%%%%%%%%%%%%%%%%%%%%%%%%%%%%%%%%%%%%%%%%%%%%%%%%%%%%%%%%
$$ \mc P=\{\vb V_1, \vb V_2, \vb V_3, \vb V_4\}, \qquad 
   \mc P^\prime=\{\vb V_1, \vb V_2^\prime, \vb V_3^\prime, \vb V_4^\prime\}.
$$
%%%%%%%%%%%%%%%%%%%%%%%%%%%%%%%%%%%%%%%%%%%%%%%%%%%%%%%%%%%%%%%%%%%%%%
(Note that $\mc P, \mc P^\prime$ have at least one common vertex.)
I consider six-dimensional integrals over the product 
domain $\mc P\times \mc P^\prime$ of the form
%%%%%%%%%%%%%%%%%%%%%%%%%%%%%%%%%%%%%%%%%%%%%%%%%%%%%%%%%%%%%%%%%%%%%%
\numeq{OriginalIntegral}
{
 \mathcal{I}
=\int_{\mc P} d\vb x \, \int_{\mc P^\prime} d\vb x^\prime \, 
  P(\vb x, \vb x^\prime) K(|\vb x-\vb x^\prime|)
}
%%%%%%%%%%%%%%%%%%%%%%%%%%%%%%%%%%%%%%%%%%%%%%%%%%%%%%%%%%%%%%%%%%%%%%
where $P$ is a polynomial in the cartesian components of 
$\vb x, \vb x^\prime$ and $K(r)$ is a scalar kernel function that
may be singular at $r=0$. A convenient parameterization of the 
domain of integration is 
%%%%%%%%%%%%%%%%%%%%%%%%%%%%%%%%%%%%%%%%%%%%%%%%%%%%%%%%%%%%%%%%%%%%%%
\begin{align*}
 \vb x &= V_1 + \xi_1 \vb A
              + \xi_2 \vb B
              + \xi_3 \vb C
\\
 \vb x^\prime &= V_1 + \eta_1 \vb A^\prime
                     + \eta_2 \vb B^\prime
                     + \eta_3 \vb C^\prime
\end{align*}
%%%%%%%%%%%%%%%%%%%%%%%%%%%%%%%%%%%%%%%%%%%%%%%%%%%%%%%%%%%%%%%%%%%%%%
\begin{align*}
 \vb A&=(\vb V_2-\vb V_1), \qquad 
 \vb B=(\vb V_3-\vb V_2),  \qquad
 \vb C=(\vb V_4-\vb V_3)
\\
 \vb A^\prime&=(\vb V_2^\prime-\vb V_1),       \qquad 
 \vb B^\prime=(\vb V_3^\prime-\vb V_2^\prime), \qquad
 vb C^\prime=(\vb V_4^\prime-\vb V_3^\prime)
\end{align*}
%%%%%%%%%%%%%%%%%%%%%%%%%%%%%%%%%%%%%%%%%%%%%%%%%%%%%%%%%%%%%%%%%%%%%%
$$
       0 \le \xi_1, \eta_1 \le 1,
\quad 0 \le \xi_2 \le \xi_1, 
\quad 0 \le \xi_3 \le \xi_2, 
\quad 0 \le \eta_2 \le \eta_1, 
\quad 0 \le \eta_3 \le \eta_2.
$$
The integral (\ref{OriginalIntegral}) becomes
%%%%%%%%%%%%%%%%%%%%%%%%%%%%%%%%%%%%%%%%%%%%%%%%%%%%%%%%%%%%%%%%%%%%%%
\numeq{OriginalIntegral2}
{
 \mathcal{I} =
\int_0^1 d\xi_1 \, \int_0^{\xi_1} d\xi_2 \, \int_0^{\xi_2} d\xi_3 \,
\int_0^1 d\eta_1 \, \int_0^{\eta_1} d\eta_2 \, \int_0^{\eta_2} d\eta_3 \,
P(\vbXi, \vbEta) K(\vbXi, \vbEta)
}
%%%%%%%%%%%%%%%%%%%%%%%%%%%%%%%%%%%%%%%%%%%%%%%%%%%%%%%%%%%%%%%%%%%%%%
Following~\cite{TaylorDuffy}, we now change variables from 
$\{\vbXi, \vbEta\}$ to $\{\vbXi, \vb u\}$, where $\vb u$ are 
the relative coordinates
%%%%%%%%%%%%%%%%%%%%%%%%%%%%%%%%%%%%%%%%%%%%%%%%%%%%%%%%%%%%%%%%%%%%%%
$$ u_i=\eta_i - \xi_i, $$
%%%%%%%%%%%%%%%%%%%%%%%%%%%%%%%%%%%%%%%%%%%%%%%%%%%%%%%%%%%%%%%%%%%%%%
and decompose the domain of integration in (\ref{OriginalIntegral2})
into subdomains with the property that, in each subdomain, the
range of the $\{\vb u\}$ coordinates is a tetrahedron in the 
$(u_1, u_2, u_3)$ space with one vertex at the origin.
In total there are 48 such tetrahedra; for example, 
the integral over one of them reads
%%%%%%%%%%%%%%%%%%%%%%%%%%%%%%%%%%%%%%%%%%%%%%%%%%%%%%%%%%%%%%%%%%%%%%
\numeq{SingleTetIntegral}
{
 \Delta \mathcal{I}
= \int_0^1    \, du_1 \,
  \int_0^{u_1} \, du_2 \,
  \int_0^{u_2} \, du_3 \,
  I(u_1, u_2, u_3),
}
%%%%%%%%%%%%%%%%%%%%%%%%%%%%%%%%%%%%%%%%%%%%%%%%%%%%%%%%%%%%%%%%%%%%%%
\begin{align*}
  I(u_1, u_2, u_3)
&=\int_{\xi_L(u_1)}^{\xi_U(u_1)} \, d\xi_1
  \int_{\xi_L(u_2)}^{\xi_U(u_2)} \, d\xi_2
  \int_{\xi_L(u_3)}^{\xi_U(u_3)} \, d\xi_3
  P(\vbXi, \vb u + \vbXi)
  K(\vbXi, \vb u + \vbXi)
\end{align*}
%%%%%%%%%%%%%%%%%%%%%%%%%%%%%%%%%%%%%%%%%%%%%%%%%%%%%%%%%%%%%%%%%%%%%%
where the upper- and lower-bound functions for the $\xi$ variables
read
$$ \xi_L(u) = \texttt{MAX}(-u, 0), \qquad 
   \xi_U(u) = \texttt{MIN}(1, 1-u).
$$
Integrals over the remaining 47 tetrahedra may be obtained
from (\ref{SingleTetIntegral}) by permuting
the $\vb u$ variables

\end{document}
